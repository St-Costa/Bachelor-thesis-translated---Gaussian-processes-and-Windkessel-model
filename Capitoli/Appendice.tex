\chapter*{Appendix}
\addcontentsline{toc}{chapter}{Appendix}




This last chapter includes all the codes not previously mentioned for generating images and being able to work with the Windkessel model as proposed in the chapter \ref{windkessel}.\\
All codes are in Python; Anaconda (version 2021.11), conda (version 4.12.0), python (version 3.9.7) were used to create them.\\
The complete codes for creating Gaussian process training are not included because they are a weak modification of Dr. Stefano Longobardi's codes, found on GitHub at the "\href{https://github.com/stelong/GPErks}{GPErks}" page.



\begin{textblock*}{0.64\textwidth}(3.5cm+0.36\textwidth,18.5cm)
   \epigraph{Appendix usually means "\textit{small outgrowth from large intestine}", but in this case it means "\textit{additional information accompanying main text}". Or are those really the same things? Think carefully before you insult this book.}{Pseudonymous Bosch}
\end{textblock*}


\newpage

%%%%%%%%%%%%%%%%%%%%%%%%%%%%%%%%%%%%%%%%
%%%%%  PROCESSI GAUSSIANI
%%%%%%%%%%%%%%%%%%%%%%%%%%%%%%%%%%%%%%%%
\section*{Gaussian processes}

\subsection*{Import and preliminary codes}

%%% Import
\lstinputlisting[language=Python, caption = Import needed for image generation of kernels introduced in the chapter \ref{gaussianProcessChapter}, firstnumber=1, stepnumber=1, label={import}]{codes/Processi gaussiani/Import.txt}

%%% Import
\lstinputlisting[language=Python, caption = Import needed for image generation on chapter prediction \ref{gaussianProcessChapter}, firstnumber=1, stepnumber=1, label={import2}]{codes/Processi gaussiani/Import2.txt}

%%% Cubed mean
\lstinputlisting[language=Python, caption = Definition of cubic mean, firstnumber=1, stepnumber=1, label={cubedMean}]{codes/Processi gaussiani/cubedMean.txt}

\newpage

%%% Esempio di processo gaussiano
\lstinputlisting[language=Python, caption=Code to generate the figure \ref{esempioProcessoGaussianoImmagine}, firstnumber=1, stepnumber=1, label={Example}]{codes/Processi gaussiani/Example.txt}



\newpage




\subsection*{Linear kernel}

%%% Plot kernel
\lstinputlisting[language=Python, label={linear kernel code}, caption={Code to generate the figure \ref{linear kernel}} ,firstnumber=1, stepnumber=1]{codes/Processi gaussiani/Linear kernel.txt}

%%% Sample
\lstinputlisting[language=Python, caption={Code to generate the figure \ref{10 sample linear kernel zero mean}} ,firstnumber=1, stepnumber=1, label={linear sample}]{codes/Processi gaussiani/Linear sample.txt}

%%% Vario c
\lstinputlisting[language=Python, caption={Code to generate the figure \ref{10 sample linear modified c}} ,firstnumber=1, stepnumber=1, label={Linear - c}]{codes/Processi gaussiani/Linear - c.txt}


%%% Vario sigmab
\lstinputlisting[language=Python, caption={Code to generate the figure \ref{10 sample linear modified sigmab}} ,firstnumber=1, stepnumber=1, label={Linear - sigmab}]{codes/Processi gaussiani/Linear - sigmab.txt}


%%% Vario sigmav
\lstinputlisting[language=Python, caption={Code to generate the figure \ref{10 sample linear modified sigmav}} ,firstnumber=1, stepnumber=1, label={Linear - sigmav}]{codes/Processi gaussiani/Linear - sigmav.txt}


%%% cubedmean
\lstinputlisting[language=Python, caption={Code to generate the figure \ref{10 sample linear kernel cubed mean}} ,firstnumber=1, stepnumber=1, label={linear cubedmean}]{codes/Processi gaussiani/Linear - cubedmean.txt}







\subsection*{Squared-exponential kernel}

%%% Plot kernel
\lstinputlisting[language=Python, caption={Code to generate the figure \ref{squared-exponential kernel}} ,firstnumber=1, stepnumber=1, label={squared-exponential}]{codes/Processi gaussiani/RBF kernel.txt}

%%% Sample
\lstinputlisting[language=Python, caption={Code to generate the figure \ref{10 sample exponential kerne zero mean}} ,firstnumber=1, stepnumber=1, label={RBF sample}]{codes/Processi gaussiani/RBF sample.txt}

\newpage

%%% Vario sigma
\lstinputlisting[language=Python, caption={Code to generate the figure \ref{10 sample exponential modified sigma}} ,firstnumber=1, stepnumber=1, label={RBF - sigma}]{codes/Processi gaussiani/RBF - sigma.txt}


%%% Vario l
\lstinputlisting[language=Python, caption={Code to generate the figure \ref{10 sample exponential modified l}} ,firstnumber=1, stepnumber=1, label={codice9}]{codes/Processi gaussiani/RBF - l.txt}

\newpage

%%% cubedmean
\lstinputlisting[language=Python, caption={Code to generate the figure \ref{10 sample exponential kernel cubed mean}} ,firstnumber=1, stepnumber=1, label={codice10}]{codes/Processi gaussiani/RBF - cubedMean.txt}




\newpage

\subsection*{Periodic kernel}


%%% kernel plot
\lstinputlisting[language=Python, caption={Code to generate the figure \ref{periodic kernel}} ,firstnumber=1, stepnumber=1, label={periodic Kernel}]{codes/Processi gaussiani/Periodic kernel.txt}

%%% sample
\lstinputlisting[language=Python, caption={Code to generate the figure \ref{3 sample periodic kerne zero mean}} ,firstnumber=1, stepnumber=1, label={periodic sample}]{codes/Processi gaussiani/Periodic sample.txt}

%%% vario sigma
\lstinputlisting[language=Python, caption={Code to generate the figure \ref{10 sample periodic modified sigma}} ,firstnumber=1, stepnumber=1, label={Periodic sigma}]{codes/Processi gaussiani/Periodic sigma.txt}

%%% vario p
\lstinputlisting[language=Python, caption={Code to generate the figure \ref{10 sample periodic modified p}} ,firstnumber=1, stepnumber=1, label={periodic p}]{codes/Processi gaussiani/Periodic p.txt}

%%% vario l
\lstinputlisting[language=Python, caption={Code to generate the figure \ref{10 sample periodic modified l}} ,firstnumber=1, stepnumber=1, label={periodic l}]{codes/Processi gaussiani/Periodic l.txt}

%%% cubed mean
\lstinputlisting[language=Python, caption={Code to generate the figure \ref{3 sample periodic kernel cubed mean}} ,firstnumber=1, stepnumber=1, label={priodic cubedmean}]{codes/Processi gaussiani/Periodic - cubedmean.txt}




\subsection*{Composition of kernels}

%%% RBF + periodic kernel
\lstinputlisting[language=Python, caption={Code to generate the figure \ref{SE + periodic kernel}} ,firstnumber=1, stepnumber=1, label={RBF + periodic kernel}]{codes/Processi gaussiani/RBF + periodic kernel.txt}

%%% RBF + periodic sample
\lstinputlisting[language=Python, caption={Code to generate the figure \ref{SE + periodic sample}} ,firstnumber=1, stepnumber=1, label={RBF + periodic sample}]{codes/Processi gaussiani/RBF + periodic sample.txt}


%%% Linear x linear kernel
\lstinputlisting[language=Python, caption={Code to generate the figure \ref{linear * linear kernel}} ,firstnumber=1, stepnumber=1, label={linear x linear}]{codes/Processi gaussiani/Linear x linear.txt}

%%% Linear x linear sample
\lstinputlisting[language=Python, caption={Code to generate the figure \ref{linear * linear sample}} ,firstnumber=1, stepnumber=1, label={linear x linear sample}]{codes/Processi gaussiani/Linear x linear sample.txt}




\newpage




\subsection*{Prediction with Gaussian processes}

%% CODICE 21
\lstinputlisting[caption={Code to generate the figure \ref{Interpolation}},language=Python, columns=fullflexible, stepnumber=1, firstnumber=1, label={interpolation code}]{codes/Processi gaussiani/Interpolazione.txt}


\newpage

%% CODICE 21
\lstinputlisting[caption={Code to generate the figure \ref{Interpolation confidence region}},language=Python, stepnumber=1, firstnumber=1, columns=fullflexible, label={interpolation confidence region code}]{codes/Processi gaussiani/InterpolazioneConfidenceRegion.txt}


\newpage

%% CODICE 21
\lstinputlisting[caption={Code to generate the figure \ref{Noisy}},language=Python, stepnumber=1, firstnumber=1, columns=fullflexible, label={Noise code}]{codes/Processi gaussiani/InterpolazioneConfidenceRegion.txt}


\newpage

%% CODICE 21
\lstinputlisting[caption={Code to generate the figure \ref{Noisy confidence region}},language=Python, stepnumber=1, firstnumber=1, columns=fullflexible, label={Noise confidence region code}]{codes/Processi gaussiani/NoiseConfidenceRegion.txt}


\newpage

%%%%%%%%%%%%%%%%%%%%%%%%%%%%%%%%%%%%%%%%
%%%%%  WINDKESSEL
%%%%%%%%%%%%%%%%%%%%%%%%%%%%%%%%%%%%%%%%
\section*{Windkessel model}
\begin{lstlisting}[language=Python,caption={Codice per importare le librerie necessarie}\label{configurazione1}, columns=fullflexible,firstnumber=1, stepnumber=1]
    import numpy as np
    import scipy as sp
    import matplotlib.pyplot as plt
    import matplotlib.pyplot as plt
    from scipy.integrate import solve_ivp
    from scipy.optimize import minimize_scalar
\end{lstlisting}

\begin{lstlisting}[language=Python,caption={Codice per il plot dei dati reali}\label{datiReali}, columns=fullflexible,firstnumber=1, stepnumber=1]
    # Pressione e flusso dai file
    flow = np.genfromtxt('stergioFlow.dat')
    pressure = np.genfromtxt('stergioPressure.dat')

    # Time grid
    M = 1000
    # array tempo
    time = np.linspace(0.,1.,M)

    # Interpolo i dati nella grid
    flow = np.interp(time,flow[:,0],flow[:,1])
    pressure = np.interp(time,pressure[:,0],pressure[:,1])

    # Plot
    fig,ax=plt.subplots()
    ln1 = ax.plot(time,flow,'b-',label='$Q_{in}$')
    ax.set_xlabel("time [s]")
    ax.set_ylabel("$Q_{in}\,[mL/s]$")
    ax2=ax.twinx()
    ln2 = ax2.plot(time,pressure,'r-',label='P')
    ax2.set_ylabel("$P\,[mmHg]$")
    lns = ln1+ln2
    labs = [l.get_label() for l in lns]
    ax.legend(lns, labs, loc=0)
\end{lstlisting}

\begin{lstlisting}[language=Python,caption={Code for calculating total peripheral resistance}\label{resistenzatotale}, columns=fullflexible,firstnumber=1, stepnumber=1]
    rd = np.average(pressure)/np.average(flow)
    print("Pressione media - %.3f mmHg" % np.average(pressure))
    print("Flusso medio - %.3f mL/s" % np.average(flow))
    print("Resistenza - %.3f mmHg/mL*s" % rd)
\end{lstlisting}

\begin{lstlisting}[language=Python,caption={Codice per la definizione della ODE}\label{ODE}, columns=fullflexible,firstnumber=1, stepnumber=1]
    # Definisco dP/dt
    def dpdt(t,p,args):
        c,r2,qFunc,pd = args
        qin = qFunc(t)
        return (qin-(p-pd)/r2)/c
\end{lstlisting}

\newpage

\begin{lstlisting}[language=Python,caption={Code to define the flow function}\label{flusso}, columns=fullflexible,firstnumber=1, stepnumber=1]
    def qFull(t):
        # Conosco i valori del tempo e del flusso
        q = np.interp(t,time,flow)
        return q
\end{lstlisting}

\begin{lstlisting}[language=Python,caption={Codice per la definizione della funzione $f_C$}\label{fC}, columns=fullflexible,firstnumber=1, stepnumber=1]
    def funC(c,args):
        """
        Objective function for CR Windkessel with compliance
        C unknown
        Input arguments:
        - c: current value for compliance
        - args: list     containing:
            1) time array where data is available
            2) pressure array where data is available (associated to time)
            3) peripehral resistance
            4) flow function, that is called qFunc(t)
            5) ODE function, that is called dydt(t,y,args)
        Output argument:
        - errnorm: the value of objective function (7) for "c"
        """
    
        time, pressure, r, qFunc, dydt= args
    
        argsIVP = [[c,r, qFunc, pd]]
    
        # ODE
        fun = dydt
    
        # Tempo iniziale e finale
        t_span = [time[0],time[-1]]
    
        # Condizione iniziale (lista di un elemento)
        y0=[pressure[0]]
    
        # Metodo (stringa con il nome)
        method='RK45'
   
        # Punti dove valutare la soluzione
        t_eval = time

        # Tolleranza per il metodo numerico
        tol = 1e-6
    
        # solve_ivp
        sol = solve_ivp(fun=fun, t_span=t_span,y0=y0,method=method, t_eval=t_eval,args=argsIVP,rtol=tol)
    
        # Vettore addendo di f_C
        errnorm = (sol.y[0,:] - pressure)**2
    
        # f_C
        errnorm = sum(errnorm)
        return errnorm
\end{lstlisting}

\newpage

\begin{lstlisting}[language=Python,caption={Plot soluzione del modello "semplice"}\label{modelloSemplice}, columns=fullflexible,firstnumber=1, stepnumber=1]
    pPred = flow*rd
    plt.plot(time,pressure,label='Data')
    plt.plot(time,pPred,label='Resistance model')
    plt.xlabel("time [s]")
    plt.ylabel("$P\,[mmHg]$")
    plt.legend()
\end{lstlisting}

\begin{lstlisting}[language=Python,caption={Codice per generare la figura \ref{plotfC}.}\label{plotfC-code}, columns=fullflexible,firstnumber=1, stepnumber=1]
    # Considero i valori di C da 0.6 a 10 per minimizzare la funzione
    M = 100
    C = np.linspace(0.6,10,M)

    function = []
    for i in range(M):
        args = [time, pressure, rd, qFull, dpdt]
        # Valuto la funzione in quel C[i]
        err = funC(C[i],args=args)
        function.append(err)

    # Plot dei valori della f_C
    plt.plot(C,function,'b--',label='$f_C$')
    plt.legend()
    plt.ylabel("$f(C)\,[mmHg]$")
    plt.xlabel("$C\,[mL/mmHg]$")
\end{lstlisting}

\begin{lstlisting}[language=Python,caption={Codice per la stima di $C$ che minimizza $f_C$}\label{stimaC}, columns=fullflexible,firstnumber=1, stepnumber=1]
    args = [time, pressure, rd, qFull, dpdt]
    
    # Trovo il valore di C che minimizza f_C
    bracket = [0.6,10.]
    # Funzione per minimizzare
    res = minimize_scalar(funC,bracket=bracket, args=args)
    
    # Non funziona sempre
    if res.success:
        C = res.x
        print("C: %.5f" % C)
    else:
        print("ATTENTION: did not succeed in finding C!")
\end{lstlisting}

\newpage

\begin{lstlisting}[language=Python,caption={Codice per generare la figura \ref{soluzioneCapprossimata}}\label{plotSoluzioneCstimata}, columns=fullflexible,firstnumber=1, stepnumber=1]
    # Imposto i valori
    args = [[C, rd, qFull, pd]]
    fun = dpdt
    t_span = [time[0],time[-1]]
    y0 = [p0]
    method='RK45'
    t_eval = time

    # Uso solve_ivp per risolvere ODE
    sol = solve_ivp(fun=fun, t_span=t_span,y0=y0,method=method,
    t_eval=t_eval,args=args,rtol=tol)
    pPred = sol.y[0,:]

    err = pressure-pPred
    print("Error inf-norm is: %.6f" % np.linalg.norm(err,np.inf))
    print("Error 2-norm is: %.6f" % np.linalg.norm(err,2))
    plt.plot(time,pressure,'g-',label='Data')
    plt.plot(time[::10],pPred[::10],'rx',label='Model')
    plt.xlabel("time [s]")
    plt.ylabel("$P\,[mmHg]$")
    plt.legend()
\end{lstlisting}

\begin{lstlisting}[language=Python,caption={Code for the definition of the function  $f_{C,\alpha}$}\label{fCa}, columns=fullflexible,firstnumber=1, stepnumber=1]
    def funCR(x,args):
    """
    Objective function for RCR Windkessel with compliance
    C and alpha unknown. Input arguments:
    - x: list with current value for compliance and alpha
    - args: list containing:
        1) time array where data is available
        2) pressure array where data is available (associated to time)
        3) peripehral resistance
        4) flow function, that is called qFunc(t)
        5) ODE function, that is called dydt(t,y,args)
    Output argument:
    - errnorm: the value of objective function (7) for "c"
    """
    c, alpha = x
    time, pressure, r, qFunc, dydt = args
    
    # Definisco "r1" e "r2" usando "r" e "alpha"
    r1= (1-alpha)*r
    r2 = alpha*r
    
    # *****Risolvo IVP*****
    # Condizionale iniziale p(t0)=pressure[0]
    args = [[c,r2,qFunc, pd]]
    fun = dydt
    t_span = [time[0],time[-1]]
    y0 = [pressure[0]]
    method='RK45'
    t_eval = time
    sol = solve_ivp(fun=fun, t_span=t_span,y0=y0,method=method, t_eval=t_eval,args=args,rtol=tol)
    
    # Valuto f_C,alpha
    pin = sol.y[0,:] + r1*flow
    errnorm = (pin - pressure)**2
    errnorm = sum(errnorm)
    return errnorm
\end{lstlisting}

\begin{lstlisting}[language=Python,caption={Code for the estimation of $C$ e $\alpha$}\label{stimaCA}, columns=fullflexible,firstnumber=1, stepnumber=1]
    args = [time, pressure, rd, qFull, dpdt]
    x0 = [1.7,1.]
    
    # Scelgo il metodo
    #method='Powell'
    #method='CG'     NON FUNZIONA
    #method='BFGS'   NON FUNZIONA
    method='Nelder-Mead'
    
    res = minimize(funCR,x0,args=args,method=method)
    print(res)
    
    if res.success:
        C = res.x[0]
        alpha = res.x[1]
        print("C: %.5f" % C)
        print("alpha: %.5f" % alpha)
    else:
        print("ATTENTION: did not succeed in finding C!")
\end{lstlisting}

\begin{lstlisting}[language=Python,caption={Codice per generare la figura \ref{soluzioneCalphaapprossimata}.}\label{soluzioneCalphastimate}, columns=fullflexible,firstnumber=1, stepnumber=1]
    # I parametri sono stati aggiornati nei codici precedenti
    r1= (1-alpha)*rd
    r2 = alpha*rd
    
    args = [[C, r2, qFull, pd]]
    fun = dpdt
    t_span = [time[0],time[-1]]
    y0 = [pressure[0]]
    method='RK45'
    t_eval = time
    tol = 1e-6
    
    # solve_ivp
    sol = solve_ivp(fun=fun, t_span=t_span,y0=y0,method=method, t_eval=t_eval,args=args,rtol=tol)
    pPred = sol.y[0,:] + r1*flow
    
    # Plot
    err = pressure-pPred
    plt.plot(time,pressure,'g-',label='Data')
    plt.plot(time[::10],pPred[::10],'rx',label='Model')
    plt.xlabel("time [s]")
    plt.ylabel("$P\,[mmHg]$")
    plt.legend()
\end{lstlisting}

\newpage

\begin{lstlisting}[language=Python,caption={Codice per il metodo delle differenze finite centrate riadattato al calcolo della sensitività locale.}\label{differenzefinite}, columns=fullflexible,firstnumber=1, stepnumber=1]
    def centredFiniteDifference(back, forward, h):
    '''
    ----- Parameters -----
    back: lista Ppred con parametro Pi-h
    forward: lista Ppred con parametro Pi+h
    h : variazione del parametro Pi
    ----- Returns -----
    lista differenze finite centrate: [map', dbp', sbp', pp']        
    '''
    
    # MAP
    mapb = np.average(back)            # MAP(Pi-hi)
    mapf = np.average(forward)         # MAP(Pi+hi)
    mapDerivative = (mapf-mapb)/(2*h)
    
    # DBP
    dbpb = np.min(back)                # DBP(Pi-hi)
    dbpf = np.min(forward)             # DBP(Pi+hi)
    dbpDerivative = (dbpf-dbpb)/(2*h)
    
    # SBP
    sbpb = np.max(back)                # SBP(Pi-hi)
    sbpf = np.max(forward)             # SBP(Pi+hi)
    sbpDerivative = (sbpf-sbpb)/(2*h)
    
    # PP
    ppb = sbpb - dbpb                  # PP(Pi-hi)
    ppf = sbpf - dbpf                  # PP(Pi+hi)
    ppDerivative = (ppf-ppb)/(2*h)
    
    return (mapDerivative, dbpDerivative, sbpDerivative, ppDerivative)
\end{lstlisting}

%\begin{lstlisting}[language=Python,caption={Code for the calculation of variables}\label{variabili}, columns=fullflexible,firstnumber=1, stepnumber=1, style=mystyle]
    map = np.average(pPred)    # MAP
    dbp = np.min(pPred)        # DBP
    sbp = np.max(pPred)        # SBP 
    pp = sbp - dbp             # PP
\end{lstlisting}

% keywordstyle=\bfseries\color{black}

\input{codes/ODE/16 - sensitività di C}

\newpage

\input{codes/ODE/17 sensitività di R1}

\input{codes/ODE/18 - sensitività R2}

\newpage

\input{codes/ODE/19 - sensitività di Pd}

\begin{lstlisting}[language=Python,caption={Code to define the flow function for the cyclic Windkessel model}\label{flusso periodico}, columns=fullflexible,firstnumber=1, stepnumber=1]
    def qFull(t):
        if type(t)==float:
            tLoc = t-int(t)*1.
        else:
            tLoc = t-t.astype(int)*1.
        q = np.interp(tLoc,time,flow)
        return q
\end{lstlisting}