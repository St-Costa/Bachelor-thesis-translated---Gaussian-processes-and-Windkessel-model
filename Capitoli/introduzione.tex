\chapter{Introduction}
In the introductory chapter, the topics covered in the paper are motivated and introduced, and the main objective is explained: the application to the Windkessel model of supervised learning with Gaussian processes. The organization of the paper in terms of chapters, sources, images and code is then shown.

\begin{textblock*}{0.64\textwidth}(3.5cm+0.36\textwidth,18.5cm)
\epigraph{\textbf{Breve dialogo con punto di circonferenza}\\
- Per il Centro, mi scusi, qual è la via?\\
- Oh non si affanni, qui è tutto periferia.}{Marco Furgeri}
\end{textblock*}

\newpage




\section{Topics covered and motivation}
\subsection{Gaussian processes}
One of the main topics covered in the paper is supervised learning, that is, the problem of learning relationships between inputs and outputs from an example dataset and then making predictions about new inputs that the machine has never seen. It is therefore clear that the problem at hand is inductive: one must define (finite) training data $D$ and a function $f$ that makes predictions for all possible input values.\\
One approach to the problem is to define a class of functions from which to draw (e.g., linear functions), but this presents a problem: the choice of class. Indeed, this choice is very delicate since it can lead, for example, to a model based on functions that fail to accurately model the target function; in that case, predictions will be inaccurate. Furthermore, increasing the size of the class of functions (e.g., by increasing their parameters in a parametric regression context) does not necessarily improve the predictions, since there is a risk of \textit{overfitting}, in which a good fit to the training data is obtained but a poor result in predictions on new data.\\
A second approach is to assign a probability to each possible function, where higher probabilities are assigned to functions that are considered more likely, for example, because they are smoother (in terms of continuity). This approach is not without problems: there are an infinite number of possible functions and one is interested in evaluating this set in finite time. Gaussian processes elegantly solve this problem by generalizing the Gaussian probability distribution. While a probability distribution describes random variables that are scalars or vectors, this type of stochastic process governs the properties of functions. Intuitively, one can think of a function as a very long (infinite) vector in which each component is the value of the function $f(x)$ for some $x$. Throughout this work, it is explained how, although a simple idea, this solves the above problem. In fact: inference in Gaussian processes is able to draw conclusions from the properties of the function on a finite number of points, thus ignoring infinitely many. To do so, the function in question is considered as a vector with $f(x_i)$ components for $i=1,...,N$ which, because of the properties of Gaussian processes, is manipulated using multivariate Gaussian random variable theory, i.e., a relatively simple theory. This is extremely powerful because it is perfectly adaptable computationally. While not a well-known approach, in reality many models commonly used in machine learning and statistics are actually special cases or limited types of Gaussian processes. 

\newpage

\subsection{Windkessel model}

Haemodynamic models based on simplified representations of the components of the cardiovascular system can contribute strongly to the study and understanding of circulatory physiology and pathology. 
These models can be derived from the Navier-Stokes equations by exploiting specific features of blood flow, such as the cylindrical morphology of vessels, and can offer a great level of detail and a potentially accurate description of relevant quantities, but their numerical discretization is very complex and requires high computational resources.

The representation and analysis of the cardiovascular system (within what are known as zero-dimensional models) began with the modeling of arterial flow using the Windkessel model. In particular, it is the two-element Windkessel model, first proposed by Stephen Hales in 1733 and later mathematically formulated by Otto Frank in 1899, that is among the simplest and best known (zerodimensional) models. It consists of a capacitor $C$, which describes the elasticity properties of large arteries, and a resistor $R$ (divided into two resistors $R1$ and $R2$), which describes the dissipative nature of small peripheral vessels, including arterioles and capillaries. The modeling was then expanded to cover the modeling of other cardiovascular components, such as the heart, heart valves, and veins to simulate the overall hemodynamics of the entire circulatory system.

%È importante sottolineare che la meccanica del sistema cardiovascolare può presentare forti non linearità. Tra queste non linearità, le equazioni costitutive dipendenti dalla pressione e le proprietà dei vasi rappresentano un esempio di grande interesse. Ad esempio nell'elaborato, nella descrizione del modello Windkessel, i valori delle componenti $C$ e $R$ ($=R_1+R_2$) sono considerati costanti mentre, poiché rappresentano parametri fisici reali, sono soggetti a non linearità. Quando il diametro del vaso cambia in base alle variazioni di pressione, la sua capacitanza cambia, così come la sua resistenza al flusso. Questi effetti sono generalmente inclusi in modelli più complessi (uno dimensionali).

%Una comprensione approfondita della propagazione delle onde di pressione e di flusso nel sistema cardiovascolare e dell'impatto delle malattie e delle variazioni anatomiche su questi modelli può fornire informazioni preziose per la diagnosi clinica e il trattamento di patologie.
However, modeling blood flow in highly complex networks can involve computationally expensive simulations. 
%L'elevato costo computazionale e il tempo di esecuzione aumentano significativamente quando si devono simulare lunghi intervalli temporali, arrivando a richiedere diversi minuti per ogni ciclo cardiaco. 
The situation worsens when, for example, one wants to integrate different mechanisms in the cerebral microcirculation, such as cerebral perfusion or solute exchange between the blood and various tissue beds.

Several papers can be found in the literature concerning models for the simulation of arterial blood flow that address the issues of runtime and topological complexity optimization. Of interest in this work, however, is to exploit the capabilities of the Windkessel model (with easy generalization to other hemodynamic models) without having to solve the differential equation that describes it. In the Windkessel model, there is only one differential equation, so the computational cost and time required to get an approximation of the solution are very low. However, exploiting supervised learning with Gaussian processes yields the same results as in the Windkessel model (within a certain region of uncertainty) without solving the differential equation, thus decreasing the running time. In this simple case, there is no noticeable improvement, but thinking about very complex models with many differential equations, which generally require long run times and even hosting on supercomputers, acceptable run times in clinical settings can be achieved with this approach.

\newpage

\subsection{Justification of the topic}
Computational Life is a company founded in 2018 by Dr. Christian Contarino (Ph.D. at the University of Trento in 2018) focusing on biomedical applications of mathematics. Most recently, he is working on Altegos™, a patient-specific predictive decision support software that leverages the predictive technology of supervised learning in Gaussian processes. 

As anticipated, complex hemodynamic models (e.g., comprehensive models that model all components of the circulation) take a long time to run, which can be up to several hours. In a research setting this is not problematic, but in a clinical setting where one interfaces with patients who need urgent care this wait is incompatible. As anticipated then, Gaussian processes solve the run-time problem, allowing for support of a hemodynamic model that would potentially take hours to run.

In addition, Gaussian processes (and specifically the library used in the paper under \ref{Capitolo: risultati training}) allow for the study of the \textit{global sensitivity analysis} of parameters, making it possible to understand which among them actually have influence in a given output and which can be discarded from the study. This makes it possible to lighten machine learning by providing the statistical model with fewer parameters, sometimes many fewer, speeding up the process and improving accuracy (similar to what was concluded about $P_d$ in \ref{sensitività}).

These features were implemented in the research enterprise context by the research team coordinated by Dr. Contarino to create the Altegos™ product. In addition, the choice of Gaussian processes is justified by the fact that they have already shown excellent results in domains similar to the one studied in the paper (e.g., in \cite{doi:10.1098/rsta.2019.0334} and in \cite{Yuhn2022.03.10.483573}) and allow for an indication of prediction accuracy in the form of mean and standard deviation. This makes them a preferable choice to other machine learning approaches.


It is therefore evident that this technology and its application are novelties in the world of research and that its study constitutes an important addition to the academic background of a bachelor's degree student. This work, therefore, aims to study this technology applied in a simplified context, namely that of the Windkessel model, so that it can be easily generalized to more real-world and complex contexts such as those addressed in the master's degree. 

\newpage
\section{Organization of the work}
\subsection{Sources}
Each chapter is given an introductory page in which the content is anticipated and the sources used in writing it are cited.
The main source used for the part of Gaussian processes is \cite{rasmussen_gaussian_2006}; for the part related to hemodynamics and the Windkessel model important source of information was Professor Lucas Omar Müller author of \cite{ghitti_toro_müller_2022} and of the jupyter notebook that provided practical results on the Windkessel model (later extensively modified); the python library "GPErks," which can be found on GitHub, was used in the chapter "Methodology and Training Results".

\subsection{Images}
Most of the images were generated by the author of the work; of these the python code is often given in the appendix\footnote{Not of all the images are the codes for generating them given because in some cases it was too massive to include in the paper and sometimes not very useful: the codes for the GPErks library, for example, can be found on the GitHub page and there is no need to include them in the paper}. Of all images not generated by the author, the source can be found in the caption.

\subsection{Code}
The code written for generating the images and the results obtained is written in python. This choice was made because python allows many options in graph creation and because the GPErks library is written in python.

\newpage

\subsection{Structure of the work}
The first two chapters are intended to introduce the broad topic of Gaussian processes. The discussion will not devote itself to framing Gaussian processes in the broad context of stochastic processes but will limit itself to a focused study of its peculiarities useful for the purposes of supervised learning.\\

The chapter "Machine learning" introduces the concepts of Bayesian statistics underlying supervised learning focusing on the case of Gaussian processes. Some optimization methods including the one used to obtain the results shown in the last chapter are also introduced for informative purposes.\\

Next, the Windkessel model is introduced, showing only the differential equation (thus without explaining how to deduce it from the Navier-Stokes equation). Practical results on its use in predicting a patient's blood pressure from his flow are then illustrated. The chapter concludes with a study of the local sensitivity of MAP, DBP, SBP, and PP with respect to $C$ (compliance), $R_1$ (proximal resistance), $R_2$ (peripheral resistance), and $P_d$ (distal pressure), concluding that distal pressure has little influence on the variables, which is why it was excluded from the input parameters in supervised learning. \\

The last chapter shows the results obtained from training Gaussian processes following the GPErks library approach to study the $C$, $R_1$ and $R_2$ dependence of blood pressure.\\

The appendix contains most of the codes used for generating the images and results used in the paper.