\documentclass[notitlepage, twoside, openright,12pt]{book}

%%%%%%%%%%%%%%%%%%%%%%%%%%%%%%%%%%%
%%%%  Pacchetti         
%%%%%%%%%%%%%%%%%%%%%%%%%%%%%%%%%%%
%%%%% Pagine bianche
\usepackage{afterpage}
\newcommand\blankpage{
    \null
    \thispagestyle{empty}%
    %\addtocounter{page}{-1}%
    \newpage}

%%%%% Testo di prova
\usepackage{lipsum}  


%%%%% TOC 
\usepackage[titles]{tocloft}

%%%% CITAZIONI
\usepackage{epigraph}
\setlength{\epigraphwidth}{0.64\textwidth}

\usepackage[absolute,overlay]{textpos}

%%%% VERBATIM INLINE
\usepackage{fancyvrb}



%%%% Sezioni di codice
\usepackage{listings}
\renewcommand\lstlistingname{Codice}
\renewcommand\lstlistlistingname{Lista codici}





%%%% Lingua e tastiera
\usepackage[italian]{babel}
\usepackage[utf8]{inputenc}

\usepackage[a4paper,top=3cm, bottom=3cm, left=3.5cm, right=3.5cm]{geometry}

%%%% Matematica
\usepackage{amsmath}
\usepackage{amssymb}
\usepackage{amsthm}
\usepackage{xfrac}  %per le frazioni oblique
\usepackage{bm}  %bold in math mdoe


%%%% Tabelle e diagrammi
\usepackage{array}
\newcolumntype{L}{>{$}l<{$}}
\newcolumntype{S}[1]{>{\raggedright\arraybackslash}m{#1}}
\newcolumntype{M}[1]{>{\centering\arraybackslash}m{#1}}
\newcolumntype{D}[1]{>{\raggedleft\arraybackslash}m{#1}}
\usepackage[table]{xcolor}
\usepackage{xcolor}
\usepackage{./Settings/nicematrix}
\usepackage[all]{xy}

\usepackage{multirow} %Merge cell

%%%% Bibliografia
\usepackage{csquotes}
\usepackage[style=alphabetic, maxnames=5,minnames=4, backend=biber]{biblatex}
\addbibresource{biblio.bib}


%%%% Immagini
\usepackage{graphicx}
\usepackage{caption}
\usepackage{subcaption}


%%%% Oggetti flottanti
\usepackage{float}


%%%% Personalizzazione capitoli e intestazioni
\usepackage{titlesec}
\usepackage{fancyhdr}
\setlength\headheight{16pt}
\fancyhf{}


%%%% Togliere le intestazioni dalle pagine vuote
\usepackage{emptypage}


%%%% Pdf/a per consegna
\usepackage[a-1b]{pdfx}


%%%% Links
\usepackage[pdfa]{hyperref}


%%%% LANDSCAPE
\usepackage{lscape}



%%%%%%%%%%%%%%%%%%%%%%%%%%%%
%%%% Ambienti matematici %%%%
%%%%%%%%%%%%%%%%%%%%%%%%%%%%

\theoremstyle{plain}
\newtheorem{teo}{Teorema}[section]
\newtheorem*{teo*}{Teorema}
\newtheorem{lemma}[teo]{Lemma}
\newtheorem{cor}[teo]{Corollario}
\newtheorem{prop}[teo]{Proposizione}
\theoremstyle{definition}
\newtheorem{defi}[teo]{Definizione}
\newtheorem{nota}[teo]{Notazione}
\theoremstyle{remark}
\newtheorem{oss}[teo]{Osservazione}
\newtheorem{ese}[teo]{Esempio}


%%%% COMANDI MATEMATICI NUOVI
\newcommand{\indep}{\perp \!\!\! \perp}

%%%%%%%%%%%%%%%%%%%%%%%%%%%%%%%
%% STILE CAPITOLI E SECTION     
%%%%%%%%%%%%%%%%%%%%%%%%%%%%%%%

%%%%%% Comandi personalizzati
\renewcommand{\thechapter}{\Roman{chapter}}

\renewcommand\theequation{\arabic{equation}}



%%%%%%% STILE 0 = (213280)mpd(5)
%\definecolor{gray75}{gray}{0.75}
%\titleformat{\chapter}[block]
%{\Huge\bfseries}
%{\thechapter\hspace{10pt}\textcolor{gray75}{\vline width 4pt}\hspace{10pt}}
%{0pt}
%{\Huge\bfseries}

%\titleformat{\section}
 %  {\normalfont\Large\bfseries}{\thesection}{1em}{}
%%%%%%%%%%%%%%%%%%%%

%%%%%%% STILE (STEFANO COSTA).1
%\usepackage{lmodern}
%\usepackage[explicit]{titlesec}
%\usepackage{microtype}
%\usepackage{tikz}

%\titleformat{\chapter}[hang]%
%{\bfseries}{%
%\begin{minipage}[t]{0.3\linewidth}  
%\vspace{0pt}% do not remove
%\begin{tikzpicture}
%\node[
%outer sep=0pt,
%text width=1.7cm,
%minimum height=2cm,
%fill=black,
%font=\color{white}\fontsize{60}{60}\selectfont,
%align=center
%] (num) {\thechapter};
%\node[
%outer sep=0pt,
%inner sep=0pt,
%anchor=south,
%font=\color{black}\Large\normalfont
%] at ([yshift=5pt]num.north) %{\textls[150]{\textsc{\chaptertitlename}}};
%\end{tikzpicture}  
%\end{minipage}}
%{0pt}%
%{\begin{minipage}[t]{.7\linewidth}%
%    \vspace{6pt} % do not remove
%    \rule{\linewidth}{2pt}\\\vskip -1.75\baselineskip%
%    \rule{\linewidth}{.7pt}\vskip 10pt
%    {\LARGE\raggedright\textsf{#1}}
%\end{minipage}}
%%%%%%%%%%%%%



%%%%%%% STILE (STEFANO COSTA).2
%\usepackage{lmodern}
%\usepackage[explicit]{titlesec}
%\usepackage{microtype}
%\usepackage{tikz}

%\titleformat{\chapter}[hang]%
%{\bfseries}{%
%\begin{minipage}[t]{0.3\linewidth}  
%\vspace{0pt}% do not remove
%\begin{tikzpicture}
%\node[
%outer sep=0pt,
%text width=1cm,
%minimum height=2.5cm,
%fill=black,
%font=\color{white}\fontsize{20}{20}\selectfont,
%align=center
%] (num) {\thechapter};
%\node[
%outer sep=0pt,
%inner sep=0pt,
%anchor=south,
%font=\color{black}\Large\normalfont
%] at ([yshift=5pt]num.north) %{\textls[5]{\textsc{\chaptertitlename}}};
%\end{tikzpicture}  
%\end{minipage}}
%{-20pt}%
%{\begin{minipage}[t]{.7\linewidth}%
%    \vspace{6pt} % do not remove
%    \rule{\linewidth}{2pt}\\\vskip -1.75\baselineskip%
%    \rule{\linewidth}{.7pt}\vskip 10pt
%    {\LARGE\raggedright\textsf{#1}}
%\end{minipage}}
%%%%%%%%%%%%


%%%%%%%% STILE (STEFANO COSTA).3
\usepackage{xhfill}

\titleformat{\chapter}[display]
{\filcenter}{\mbox{}\xrfill[0.4ex]{3pt}[black]\enspace\textsc{\Large\thechapter}\enspace\xrfill[0.4ex]{3pt}[black]\mbox{}}{0.3ex} {{\color{black}\titlerule[1pt]}\vskip3ex\LARGE\textsf\huge\bfseries}[\medskip{\color{black}\titlerule[1pt]}]



%%%%%%% STILE 1
%\titleformat{\chapter}[frame]
%{\normalfont\huge}
%{\chaptertitlename\ \thechapter}
%{20pt}
%{\bfseries\LARGE}

%%%%%%% STILE 2
%\titleformat{\chapter}[display]
%{\normalfont\Large\filcenter\sffamily}
%{\LARGE\MakeUppercase{\chaptertitlename} \thechapter}
%{1pc}
%{\titlerule \vspace{1pc} \LARGE\normalfont\raggedright}[\vspace{1ex}\titlerule]

%%%%%%% STILE 3
%\titleformat{\chapter}[frame]
%{\normalfont}
%{\filleft \enspace Chapter  \thechapter \enspace}
%{8pt}
%{\filcenter\Large\sffamily}
%%%%%%%%%%%%%%%%%%%

%%%%%%% STILE 4
%\titleformat{\chapter}[display]
%  {\bfseries\Large}
%  {\filright\MakeUppercase{\chaptertitlename} \Huge\thechapter}
%  {1ex}
%  {\titlerule\vspace{1ex}\filleft}
%  [\vspace{1ex}\titlerule]










%%%%%%%%%%%%%%%%%%%%%%%%%%%%%%%
%% STILE CODICE    
%%%%%%%%%%%%%%%%%%%%%%%%%%%%%%%
\definecolor{codegreen}{rgb}{0,0.6,0}
\definecolor{codered}{HTML}{cc0029}
\definecolor{background}{RGB}{230, 243, 252}



\lstdefinestyle{mystyle}{
    language=Python,
    backgroundcolor=\color{background}, 
    keywordstyle=\bfseries\color{blue},
    commentstyle=\itshape\color{codegreen},
    numberstyle=\tiny\color{black},
    stringstyle=\color{codered},
    basicstyle=\linespread{0.85}\ttfamily\footnotesize,
    firstnumber=0,  %%numeri partono da 0
    stepnumber=5,   %%numero ogni 5 righe
    frame=single,
    rulecolor=\color{black},  %colore frame
    breakatwhitespace=true,         
    breaklines=true,                 
    captionpos=t,      
    abovecaptionskip=10pt,
    belowcaptionskip=10pt,
    keepspaces=true,                 
    numbers=left,                    
    numbersep=10pt,                  
    showspaces=false,                
    showstringspaces=false,
    showtabs=false,                  
    tabsize=2,
    deletekeywords=[2]{map},
}

\lstset{style=mystyle}




%%%% P di probabilità
\newcommand{\probP}{\text{I\kern-0.18em P}}



%%%%%%%%%%%%%%%%%%%%%%%%%%%%%%%%%%
%%      INIZIO DOCUMENTO        %%
\begin{document}
%%%%%%%%%%%%%%%%%%%%%%%%%%%%%%%%%%





%%%%%%%%%%%%%%%%%%%%%%%%%%%%%%%
%% Frontmatter       
%%%%%%%%%%%%%%%%%%%%%%%%%%%%%%%
\frontmatter



%%%%%FRONTESPIZIO
\thispagestyle{empty}  


%%%% LOGO %%%%%
\begin{figure}[H]
\centering
\includegraphics{images/dmtesi.png}
\end{figure}

%%%% Corso di Laurea %%%%%%
\setlength{\parskip}{-12pt} % Modificare per spaziare diversamente le righe
%\noindent\rule{\textwidth}{1pt}
\begin{center}
\vspace{0.5cm}
\Large Corso di Laurea  in Matematica
\end{center}
%\noindent\rule{\textwidth}{1pt}
%%%%%%%%%%%%%%%

\vspace{5 cm} % Aggiustare a piacere


%%%% Titolo %%%%%%
\begin{center}
{\fontsize{20}{30} \selectfont PROCESSI GAUSSIANI E APPRENDIMENTO SUPERVISIONATO: APPLICAZIONE AL MODELLO WINDKESSEL \par} %}
\end{center}
%%%%%%%%%%%%%%%


\vspace{6 cm} % Aggiustare a piacere


\begin{large}
\begin{tabular}{m{7cm}l}
Candidato: & Relatore:\\
Stefano Costa & Prof. Lucas Omar Müller
\end{tabular}
\end{large}


\vfill % Manda il resto a fondo pagina

%%%% Anno accademico %%%%%
\setlength{\parskip}{-18pt} % Modificare per spaziare diversamente le righe
%\noindent\rule{\textwidth}{1pt}
\begin{center}
{Anno Accademico 2021-22}
\end{center}
%\noindent\rule{\textwidth}{1pt}
\setlength{\parskip}{0pt} % Rimettere il parskip di default % frontespizio senza intestazioni



\chapter*{Ringraziamenti}
\thispagestyle{empty}

Ringrazio il mio relatore, il professor Lucas Omar Müller, per la sua pazienza, la sua guida e il suo sostegno. Ho tratto grande beneficio dalle sue conoscenze e dai suoi consigli, senza i quali questa tesi non sarebbe stata possibile.\\

Un doveroso ringraziamento va anche al dottor Christian Contarino, che non solo è stato fondamentale per la scelta e lo sviluppo dell'argomento, ma mi ha dato un costante sostegno morale.
\clearpage


%%%TOC
\pagestyle{plain} % indice col solo numero di pagina
\setcounter{tocdepth}{1}


\setlength\cftsecnumwidth{30pt}  %distanza della sezione dopo il numero
\setlength\cftchapnumwidth{30pt} %distanza del chapter dopo il numero


\tableofcontents
\addtocontents{toc}{~\hfill\textbf{Pag.}\par} %aggiunge Pag.



\newpage






%% LISTA CODICI

%Distanza dal numero a sinistra
\makeatletter
\def\l@lstlisting#1#2{\@dottedtocline{1}{1.5em}{3.5em}{#1}{#2}}
\makeatother

\setcounter{tocdepth}{2}
\lstlistoflistings
\addcontentsline{toc}{chapter}{Lista codici}


\cftsetindents{figure}{1.5em}{3.5em}  %distanza dal numero a sinistra

\listoffigures
\addcontentsline{toc}{chapter}{Elenco delle figure}


\listoftables
\addcontentsline{toc}{chapter}{Elenco delle tabelle}











%%%%%%%%%%%%%%%%%%%%%%%%%%%%%%%
%% Intestazione per l'introduzione
%%%%%%%%%%%%%%%%%%%%%%%%%%%%%%%

%\pagestyle{fancy}
%\fancyhf{}

%\fancyfoot[C]{\thepage}
%\renewcommand{\headrulewidth}{0pt}
%\renewcommand{\footrulewidth}{0pt}










%% MAINMATTER
\mainmatter


%%%%%%%%%%%%%%%%%%%%%%%%%%%%%%%
%%%%% Intestazione per il corpo principale del documento
%%%%%%%%%%%%%%%%%%%%%%%%%%%%%%%
\pagestyle{fancy}

\fancyhead[LOH]{\nouppercase\rightmark}
\fancyhead[ROH]{\textbf{\thepage}}

\fancyhead[LEH]{\textbf{\thepage}}
\fancyhead[REH]{\nouppercase \leftmark }
\renewcommand{\headrulewidth}{0pt}
\renewcommand{\footrulewidth}{0pt}

%Cambia in alto a destra il capitolo:
%   (nome capitolo)      |  Capi. (#capitolo)
\renewcommand{\chaptermark}[1]{\markboth{#1 \qquad $\vert$ \quad  Cap. \thechapter\quad}{}}

%Cambia in alto a sinistra la sezione:
% sezione: (num sezione) | (nome sezione
\renewcommand{\sectionmark}[1]{\markright{\quad Sez. \thesection \quad $\vert$ \qquad ~#1}}




%%%%%% 
%%    Rimuove il numero di pagina dei capitoli
\makeatletter
\let\oldchapter\chapter  %salvo il comando chapter per richiamarlo dopo in modo che nella bibliografia compaia il numero di pagina
\renewcommand\chapter{\if@openright\cleardoublepage\else\clearpage\fi
                    \thispagestyle{empty}%
                    \global\@topnum\z@
                    \@afterindentfalse
                    \secdef\@chapter\@schapter}
\makeatother
%%%%%%%





%%%%%%%%%%%%%%%%%%%%%%%%%%%%%%
%%%%%%%%%%%% CAPITOLI
%%%%%%%%%%%%%%%%%%%%%%%%%%%%%%

\chapter{Introduzione}
Nel capitolo introduttivo vengono motivati e introdotti gli argomenti trattati nell'elaborato e viene illustrato l'obiettivo principale: l'applicazione al modello Windkessel dell'apprendimento supervisionato nei processi gaussiani. Viene poi mostrata l'organizzazione dell'elaborato in termini di capitoli, fonti, immagini e codice.


\begin{textblock*}{0.64\textwidth}(3.5cm+0.36\textwidth,18.5cm)
\epigraph{\textbf{Breve dialogo con punto di circonferenza}\\
- Per il Centro, mi scusi, qual è la via?\\
- Oh non si affanni, qui è tutto periferia.}{Marco Furgeri}
\end{textblock*}

\newpage




\section{Argomenti trattati e motivazione}
\subsection{Processi gaussiani}
Uno dei temi principali trattati nell'elaborato è l'apprendimento supervisionato, ovvero il problema di apprendere delle relazioni tra input e output a partire da un dataset di esempio per poi fare previsioni su nuovi input che la macchina non ha mai visto. È quindi chiaro che il problema in questione è induttivo: si devono definire dei dati di addestramento (finiti) $D$ e una funzione $f$ che faccia previsioni per tutti i possibili valori di ingresso.\\
Un approccio al problema è quello di definire una classe di funzioni da cui attingere (ad esempio funzioni lineari), ma ciò presenta un problema: la scelta della classe. Questa scelta è infatti molto delicata poiché può portare, ad esempio, ad un modello basato su funzioni che non riescono a modellare accuratamente la funzione target; in tal caso le previsioni saranno imprecise. Inoltre aumentare la vastità della classe di funzioni (ad esempio aumentandone i parametri in un contesto di regressione parametrica) non necessariamente migliora le previsioni, poiché si corre il rischio di \textit{overfitting}, in cui si ottiene un buon adattamento ai dati di addestramento ma un pessimo risultato nelle previsioni su nuovi dati.\\
Un secondo approccio consiste nell'attribuire una probabilità a ogni funzione possibile, dove le probabilità più alte sono attribuite alle funzioni che si considerano più probabili, ad esempio perché sono più lisce (in termini di continuità). Questo approccio non è esente da problemi: esistono infinite funzioni possibili e si è interessati a valutare questo insieme in tempo finito. I processi gaussiani risolvono elegantemente questo problema generalizzando la distribuzione di probabilità gaussiana. Mentre una distribuzione di probabilità descrive variabili casuali che sono scalari o vettori, questo tipo di processo stocastico regola le proprietà delle funzioni. Intuitivamente, si può pensare a una funzione come a un vettore molto lungo (infinito) in cui ogni componente è il valore della funzione $f(x)$ per un certo $x$. Nel corso dell'elaborato viene spiegato come, sebbene sia un'idea semplice, ciò risolva il suddetto problema. Infatti: l'inferenza nei processi gaussiani è in grado di trarre conclusioni a partire dalle proprietà della funzione su un numero finito di punti, ignorandone quindi infiniti. Per farlo, la funzione in questione viene considerata come un vettore con componenti $f(x_i)$ per $i=1,...,N$ che, per le proprietà dei processi gaussiani, è manipolabile con la teoria delle variabili aleatorie gaussiane multivariate, cioè con una teoria relativamente semplice. Ciò è estremamente potente perché è perfettamente adattabile computazionalmente. Seppur non sia un approccio molto conosciuto, in realtà molti modelli comunemente utilizzati nell'apprendimento automatico e nella statistica sono in realtà casi speciali o tipi limitati di processi gaussiani. 

\newpage

\subsection{Modello Windkessel}

I modelli emodinamici basati su rappresentazioni semplificate dei componenti del sistema cardiovascolare possono contribuire fortemente allo studio e alla comprensione della fisiologia e delle patologie circolatorie. 
Questi modelli possono essere derivati dalle equazioni di Navier-Stokes sfruttando caratteristiche specifiche del flusso sanguigno, come la morfologia cilindrica dei vasi, e possono offrire un grande livello di dettaglio e una descrizione potenzialmente accurata delle quantità rilevanti, ma la loro discretizzazione numerica è molto complessa e richiede elevate risorse computazionali.

La rappresentazione e l'analisi del sistema cardiovascolare (in quelli che sono i modelli zero dimensionali) sono iniziate con la modellazione del flusso arterioso utilizzando il modello Windkessel. In particolare è il modello Windkessel a due elementi, proposto per la prima volta da Stephen Hales nel 1733 e successivamente formulato matematicamente da Otto Frank nel 1899, ad essere tra i più semplici e noti modelli (zerodimensionali). Esso è costituito da un condensatore $C$, che descrive le proprietà di elasticità delle grandi arterie, e un resistore $R$ (diviso in due resistori $R1$ e $R2$), che descrive la natura dissipativa dei piccoli vasi periferici, comprese arteriole e capillari. La modellistica si è poi ampliata per coprire la modellazione di altri componenti cardiovascolari, come il cuore, le valvole cardiache e le vene per simulare l'emodinamica globale dell'intero sistema circolatorio.

%È importante sottolineare che la meccanica del sistema cardiovascolare può presentare forti non linearità. Tra queste non linearità, le equazioni costitutive dipendenti dalla pressione e le proprietà dei vasi rappresentano un esempio di grande interesse. Ad esempio nell'elaborato, nella descrizione del modello Windkessel, i valori delle componenti $C$ e $R$ ($=R_1+R_2$) sono considerati costanti mentre, poiché rappresentano parametri fisici reali, sono soggetti a non linearità. Quando il diametro del vaso cambia in base alle variazioni di pressione, la sua capacitanza cambia, così come la sua resistenza al flusso. Questi effetti sono generalmente inclusi in modelli più complessi (uno dimensionali).

%Una comprensione approfondita della propagazione delle onde di pressione e di flusso nel sistema cardiovascolare e dell'impatto delle malattie e delle variazioni anatomiche su questi modelli può fornire informazioni preziose per la diagnosi clinica e il trattamento di patologie.
Tuttavia, la modellazione del flusso sanguigno in reti altamente complesse può comportare simulazioni computazionalmente costose. 
%L'elevato costo computazionale e il tempo di esecuzione aumentano significativamente quando si devono simulare lunghi intervalli temporali, arrivando a richiedere diversi minuti per ogni ciclo cardiaco. 
La situazione peggiora quando, per esempio, si vogliono integrare diversi meccanismi nella microcircolazione cerebrale, ad esempio la perfusione cerebrale o lo scambio di soluti tra il sangue e i vari letti tissutali.

In letteratura si trovano diversi lavori riguardanti modelli per la simulazione del flusso sanguigno arterioso che affrontano le questioni del tempo di esecuzione e dell'ottimizzazione della complessità topologica. Di interesse dell'elaborato, invece, è sfruttare le potenzialità del modello Windkessel (con una facile generalizzazione ad altri modelli emodinamici) senza dover risolvere l'equazione differenziale che lo descrive. Nel modello Windkessel si ha una sola equazione differenziale, dunque il costo computazionale e il tempo richiesto per avere un'approssimazione della soluzione sono molto bassi. Tuttavia, sfruttando l'apprendimento supervisionato con i processi gaussiani si ottengono gli stessi risultati del modello Windkessel (all'interno di una certa regione di incertezza) senza risolvere l'equazione differenziale, diminuendo quindi il tempo di esecuzione. In questo caso semplice non si ha una miglioria sensibile, ma pensando a modelli molto complessi con molte equazioni differenziali, che generalmente richiedono lunghi tempi di esecuzione e addirittura l'hosting su supercomputer, con questo approccio si possono ottenere tempi di attesa accettabili in contesti clinici.

\newpage

\subsection{Motivazione dell'argomento}
Computational Life è un'azienda fondata nel 2018 dal dottor Christian Contarino (dottorato all'università di Trento nel 2018) che si occupa di applicazioni biomediche della matematica. Nell'ultimo periodo si sta occupando di Altegos™, un software di supporto decisionale predittivo paziente-specifico, che sfrutta la tecnologia predittiva dell'apprendimento supervisionato nei processi gaussiani. 

Come anticipato, i modelli emodinamici complessi (ad esempio quelli completi, che modellano tutte le componenti della circolazione) richiedono molto tempo di esecuzione, che può arrivare anche a diverse ore. In un contesto di ricerca ciò non è problematico, ma in un contesto clinico in cui ci si interfaccia con pazienti che necessitano di cure urgenti questa attesa è incompatibile. Come anticipato dunque, i processi gaussiani risolvono il problema del tempo di esecuzione, permettendo di avere il supporto di un modello emodinamico che richiederebbe potenzialmente ore per la sua esecuzione.

Inoltre, i processi gaussiani (e nello specifico la libreria utilizzata nell'elaborato al capitolo \ref{Capitolo: risultati training}) consentono di studiare la \textit{global sensitivity analysis} dei parametri, permettendo di comprendere quali tra essi abbiano effettivamente influenza in un determinato output e quali possano essere scartati dallo studio. Questo consente di alleggerire l'apprendimento automatico fornendo al modello statistico meno parametri, talvolta molti meno, velocizzando il processo e migliorando la precisione (similmente a quanto concluso riguardo a $P_d$ in \ref{sensitività}).

Queste caratteristiche sono state implementate nel contesto di ricerca aziendale dal team di ricerca coordinato dal dottor Contarino per la creazione del prodotto Altegos™. Inoltre, la scelta dei processi gaussiani è giustificata dal fatto che hanno già mostrato ottimi risultati in ambiti simili a quello studiato nell'elaborato (ad esempio in \cite{doi:10.1098/rsta.2019.0334} e in \cite{Yuhn2022.03.10.483573}) e permettono di avere un'indicazione sulla precisione delle previsioni sotto forma di media e deviazione standard. Questo li rende una scelta preferibile agli altri approcci all'apprendimento automatico.


Risulta quindi evidente come questa tecnologia e la sua applicazione siano delle novità nel mondo della ricerca e che il suo studio costituisca un'importante aggiunta al bagaglio di conoscenze accademiche di uno studente di laurea triennale. L'elaborato, quindi, si pone come obiettivo di studiare questa tecnologia applicata in un contesto semplificato, cioè quello del modello Windkessel, affinché possa essere facilmente generalizzata a contesti più reali e complessi come quelli affrontati nella laurea magistrale. 

\newpage
\section{Organizzazione dell'elaborato}
\subsection{Fonti}
Ad ogni capitolo viene dedicata una pagina introduttiva in cui viene anticipato il contenuto e vengono citate le fonti usate per la stesura dello stesso.
La principale fonte usata per la parte dei processi gaussiani è \cite{rasmussen_gaussian_2006}; per la parte legata all'emodinamica e al modello Windkessel fonte importante di informazioni è stato il professore Lucas Omar Müller autore di \cite{ghitti_toro_müller_2022} e della jupyter notebook che ha fornito risultati pratici sul modello Windkessel (poi ampiamente modificata); nel capitolo "Metodologia e risultati training" è stata usata la libreria python "GPErks", che si trova su GitHub.

\subsection{Immagini}
La maggior parte delle immagini è stata generata dall'autore dell'elaborato; di queste viene spesso riportato il codice python nell'appendice\footnote{Non di tutte le immagini vengono riportati i codici per generarle perché in alcuni casi esso era troppo imponente da inserire nell'elaborato e talvolta poco utile: i codici della libreria GPErks, ad esempio, si trovano sulla pagina GitHub e non vi è necessità di riportarli nell'elaborato.}. Di tutte le immagini non generate dall'autore si trova la fonte in didascalia.

\subsection{Codice}
Il codice scritto per la generazione delle immagini e i risultati ottenuti è scritto in python. Questa scelta è stata fatta perché il python permette molte opzioni nella creazione dei grafici e perché la libreria GPErks è scritta in python.

\newpage

\subsection{Struttura dell'elaborato}
I primi due capitoli hanno lo scopo di introdurre il vasto tema dei processi gaussiani. La trattazione non si dedicherà ad inquadrare i processi gaussiani nel vasto contesto dei processi stocastici ma si limiterà ad uno studio mirato delle sue peculiarità utili ai fini dell'apprendimento supervisionato.\\

Nel capitolo "Machine learning" vengono introdotti i concetti di statistica bayesiana che stanno alla base dell'apprendimento supervisionato focalizzandosi sul caso dei processi gaussiani. Vengono anche introdotti, a titolo informativo, alcuni metodi di ottimizzazione tra cui quello usato per ottenere i risultati illustrati nell'ultimo capitolo.\\

Successivamente viene introdotto il modello Windkessel, mostrando unicamente l'equazione differenziale (senza quindi spiegare come dedurla dall'equazione Navier-Stokes). Vengono poi illustrati risultati pratici sul suo uso nella predizione della pressione arteriosa di un paziente a partire dal suo flusso. Il capitolo si conclude con lo studio della sensitività locale di MAP, DBP, SBP e PP rispetto a $C$ (compliance), $R_1$ (resistenza prossimale), $R_2$ (resistenza periferica), $P_d$ (pressione distale), concludendo che la pressione distale influenza poco le variabili, motivo per cui è stata esclusa dai parametri di input nell'apprendimento supervisionato. \\

Nell'ultimo capitolo vengono illustrati i risultati ottenuti dal training di processi gaussiani seguendo l'approccio della libreria GPErks per studiare la dipendenza da $C$, $R_1$  e $R_2$ della pressione arteriosa.\\

In appendice vi sono la maggior parte dei codici usati per la generazione delle immagini e dei risultati usati nell'elaborato.
\chapter{Distribuzione gaussiana}
In questo capitolo vengono fornite le proprietà di base delle distribuzioni gaussiane, con particolare attenzione nei riguardi della \textbf{distribuzione gaussiana multivariata}. Come suggerisce il nome, i \textbf{processi gaussiani} si basano infatti su queste distribuzioni multivariate, sfruttando le proprietà che verranno mostrate nel corso del capitolo.\\
Le fonti usate per la stesura del capitolo sono \cite{gut_intermediate_2009}, \cite{wilkinson_introduction_2020}, \cite{murphy_probabilistic_2022}.


\begin{textblock*}{0.64\textwidth}(3.5cm+0.36\textwidth,18.5cm)
\epigraph{At a purely formal level, one could call probability theory the study of measure spaces with total measure one, but that would be like calling number theory the study of strings of digits which terminate.}{Terence Tao}
\end{textblock*}

\newpage

%%%%%%%%%%%%%%%%%%%%%%%%%%%%%%%%%%%
%%%%%% GAUSSIANA UNIVARIATA
%%%%%%%%%%%%%%%%%%%%%%%%%%%%%%%%%%%


\section{Distribuzione gaussiana univariata}
\begin{defi}[Distribuzione gaussiana univariata]
La \textbf{distribuzione gaussiana univariata} è una distribuzione di probabilità continua. \\
Data $Y\sim \mathcal{N}(\mu, \sigma^2)$, la sua funzione di densità di probabilità si esprime come:
\[f_Y(y) = \frac{1}{\sqrt{2\pi \sigma^2}} \text{exp}\bigg\{-\frac{1}{2}\frac{(y-\mu)^2}{\sigma^2}\bigg\}.\]
\end{defi}


%%%%%%%%%%%%%%%%%%%%%%%%%%%%%%%%%%%
%%%%%% IMMAGINI GAUSSIANA
%%%%%%%%%%%%%%%%%%%%%%%%%%%%%%%%%%%
\begin{figure}[h]
\centering
\begin{subfigure}{.5\textwidth}
  \centering
  \includegraphics[width=\linewidth]{images/Gaussiane/PDFNormalDistribution.png}
  \caption{Densità di probabilità}
  \label{fig:sub1}
\end{subfigure}%
\begin{subfigure}{.5\textwidth}
  \centering
  \includegraphics[width=\linewidth]{images/Gaussiane/CDFNormalDistribution.png}
  \caption{Funzione di ripartizione}
  \label{fig:sub2}
\end{subfigure}
\caption{Funzione di ripartizione e densità di probabilità di una distribuzione normale standard \cite{wikiNormalDistribution}.}
\label{fig:gaussian}
\end{figure}



\begin{oss} \label{normal decomposition}
Data $Z\sim \mathcal{N}(0,1)$ segue:  $Y=\mu + \sigma Z\sim \mathcal{N}(\mu,\sigma^2)$.
\end{oss}

\begin{oss}
Informalmente parlando, la distribuzione gaussiana è "conveniente" matematicamente per via delle sue proprietà:
\begin{itemize}
    \item la distribuzione normale ha due parametri di facile interpretazione: la media e la varianza;
    \item la distribuzione normale è chiusa per operazioni lineari;
    \item la distribuzione normale è chiusa per marginalizzazione e condizionamento (si veda la proposizione \ref{marginale-condizionata});
    \item a parità di media e varianza, la distribuzione normale ha massima entropia;
    \item dal \textbf{teorema del limite centrale}, la distribuzione normale è il limite di una somma di variabili aleatorie; occorre cioè naturalmente;
    \item la distribuzione normale ha una forma matematica semplice che ne facilita l'implementazione.
\end{itemize}
\end{oss}


\newpage


%%%%%%%%%%%%%%%%%%%%%%%%%%%%%%%%%%%
%%%%%% GAUSSIANA MULTIVARIATA
%%%%%%%%%%%%%%%%%%%%%%%%%%%%%%%%%%%
\section{Distribuzione gaussiana multivariata}


La distribuzione gaussiana multivariata è una generalizzazione della distribuzione normale (univariata).



\begin{defi}[Distribuzione gaussiana multivariata] Un vettore $n$-dimensionale $\mathbf{X}$ di variabili aleatorie è detto normale (\textbf{multivariato normale}) se e solo se $\forall \mathbf{a}\in \mathbb{R}^n$ la variabile aleatoria $\mathbf{a}^\text{T}\mathbf{X}$ è una distribuzione normale.  
La densità\footnote{Nel caso generalizzato non è possibile definire la densità della distribuzione gaussiana multivariata. Come verrà mostrato in seguito, nei processi gaussiani la densità non è importante quanto la matrice di covarianza e il vettore media.} di una \textbf{distribuzione gaussiana multivariata} si esprime come
\[f_{\mathbf{X}}(\mathbf{x})=\frac{1}{(2\pi)^{\sfrac{n}{2}}  \text{det}(\bm{\Sigma})^{\sfrac{1}{2}}} \text{ exp}\left[-\frac{1}{2}(\bm{x}-\bm{\mu})^\text{T}\bm{\Sigma}^{-1}(\bm{x}-\bm{\mu})\right]
\]\end{defi}
dove $\bm{\mu}=\mathbb{E}[\bm{X}]\in \mathbb{R}^n$ è il \textit{vettore media}, $\bm{\Sigma}=\text{Cov}[\bm{X}]$ è una matrice $n\times n$ chiamata \textit{matrice di covarianza}, definita come segue:
\[\begin{split}
\text{Cov}[\bm{X}] &= \mathbb{E}\left[(\bm{X}-\mathbb{E}[\bm{X}])(\bm{X}-\mathbb{E}[\bm{X}])^\text{T} \right]\\
 & = \begin{pmatrix}
    \mathbb{V}[X_1] & \text{Cov}[X_1,X_2] & \dots & \text{Cov}[X_1,X_n]\\
    \text{Cov}[X_2,X_1] & \mathbb{V}[X_2] & \dots & \text{Cov}[X_2,X_n]\\
    \vdots & \vdots & \ddots & \vdots\\
    \text{Cov}[X_n,X_1] & \text{Cov}[X_n,X_2] & \dots & \mathbb{V}[X_n]
    \end{pmatrix}
\end{split}
\]
dove:
\[ \text{Cov}[X_i,X_j]=\mathbb{E}[(X_i-\mathbb{E}[X_i])(X_j-\mathbb{E}[X_j])] = \mathbb{E}[X_iX_j]-\mathbb{E}[X_i]\mathbb{E}[X_j]  \]
\[ \mathbb{V}[X_i]=\text{Cov}[X_i, X_i]. \]


\begin{oss}\label{ossGaussianaMultivariata}
La matrice di covarianza è simmetrica e semidefinita positiva, cioè: $\forall \mathbf{a}\in \mathbb{R}^n$ si ha $\mathbf{a}^\text{T}\mathbf{\Sigma} \mathbf{a}\geq 0$.
\end{oss}


%%%%%%%%%%%%%%%%%%%%%%%%%%%%%%%%%%%
%%%%%% COROLLARIO DEFINIZIONE
%%%%%%%%%%%%%%%%%%%%%%%%%%%%%%%%%%%
\begin{cor}
Dato $\mathbf{X}$ un vettore multivariato normale, dalla definizione di distribuzione gaussiana multivariata seguono immediatamente:
\begin{enumerate}
    \item ogni componente di $\mathbf{X}$ è una variabile aleatoria gaussiana;
    \item $\sum_{i=1}^{n} a_iX_i$ è una variabile aleatoria gaussiana $\forall a_i\in \mathbb{R}$;
    \item se le componenti di $\mathbf{X}$ sono variabili aleatorie gaussiane indipendenti, allora $\mathbf{X}$ è un vettore multivariato normale.
\end{enumerate}
\end{cor}

\begin{oss}
Il terzo punto del corollario precedente non vale se le componenti non sono indipendenti. Per esempio: $X\sim \mathcal{N}(0,1)$, $Z\indep X$,  $\mathbb{P}(Z=1)=\mathbb{P}(Z=-1)=\sfrac{1}{2}$. Si vede facilmente che $Y=ZX$ non è indipendente da $X$ e $\begin{pmatrix}
X\\
Y
\end{pmatrix}$ non è multivariato normale .
\end{oss}

\newpage


%%%%%%%%%%%%%%%%%%%%%%%%%%%%%%%%%%%
%%%%%% PROPOSIZIONE
%%%%%%%%%%%%%%%%%%%%%%%%%%%%%%%%%%%
Di fondamentale importanza è la prossima proposizione. La dimostrazione viene omessa in quanto consta di lunghi calcoli che non rientrano nei fini dell'elaborato. Per la dimostrazione si rimanda a \cite{murphy_probabilistic_2022}.
\begin{prop}[Distribuzione marginale e condizionata] \label{marginale-condizionata}
Sia $\bm{Y}=\begin{pmatrix}\bm{y}_1 \\ \bm{y}_2\end{pmatrix}$ vettore multivariato gaussiano dove:
\[
\bm{\mu}=\begin{pmatrix}\bm{\mu}_1\\ \bm{\mu}_2\end{pmatrix}\quad
 \bm{\Sigma}=\begin{pmatrix}\bm{\Sigma}_{11}&\bm{\Sigma}_{12}\\ \bm{\Sigma}_{21}&\bm{\Sigma}_{22}\end{pmatrix}\quad \bm{\Lambda}=\bm{\Sigma}^{-1}=\begin{pmatrix}\bm{\Lambda}_{11}&\bm{\Lambda}_{12}\\\bm{\Lambda}_{21}&\bm{\Lambda}_{22}\end{pmatrix}
\]
Allora le distribuzioni marginali sono:
\[
\bm{y}_1\sim \mathcal{N}(\bm{\mu}_1, \bm{\Sigma}_{11}) \qquad \bm{y}_2\sim \mathcal{N}(\bm{\mu}_2, \bm{\Sigma}_{22})
\]
Mentre le distribuzioni condizionate sono:
\[
\bm{y}_1 | \bm{y}_2 \sim \mathcal{N}(\bm{\mu}_{1|2}, \bm{\Sigma}_{1|2}) \qquad
\def\arraystretch{1.4}
\begin{array}{l}
    \bm{\mu}_{1|2}        =\bm{\mu}_1+\bm{\Sigma}_{12}\bm{\Sigma}_{22}^{-1}(\bm{y}_2-\bm{\mu}_2)\\
    \bm{\Sigma}_{1|2}=\bm{\Sigma}_{11}-\bm{\Sigma}_{12}\bm{\Sigma}_{22}^{-1}\bm{\Sigma}_{21}=\bm{\Lambda}_{11}^{-1}
\end{array}
\]

\[
\bm{y}_2 | \bm{y}_1 \sim \mathcal{N}(\bm{\mu}_{2|1}, \bm{\Sigma}_{2|1})
\qquad
\def\arraystretch{1.4}
\begin{array}{l}
    \bm{\mu}_{2|1}        =\bm{\mu}_2+\bm{\Sigma}_{21}\bm{\Sigma}_{11}^{-1}(\bm{y}_1-\bm{\mu}_1)\\
    \bm{\Sigma}_{2|1}=\bm{\Sigma}_{22}-\bm{\Sigma}_{21}\bm{\Sigma}_{11}^{-1}\bm{\Sigma}_{12}=\bm{\Lambda}_{22}^{-1}
\end{array}
\]
\end{prop}

\newpage

%%%%%%%%%%%%%%%%%%%%%%%%%%%%%%%%%%%
%%%%%% CORRELAZIONE VETTORI
%%%%%%%%%%%%%%%%%%%%%%%%%%%%%%%%%%%
\section{Correlazione per vettori multivariati} \label{sezioneCorrelazione}


\begin{defi}[Indice di correlazione di Pearson]
L'\textbf{indice di correlazione di Pearson} tra due variabili casuali è un indice che esprime una relazione di linearità tra esse. Esso si esprime come:
\[ \rho_{XY}=\frac{\text{Cov}[X,Y]}{\sqrt{\mathbb{V}[X]\mathbb{V}[Y]}}.
\]
\end{defi}
\begin{oss}[Significato di $\rho_{XY}$]
Per la disuguaglianza di Cauchy-Schwarz vale: $-1\leq \rho_{XY}\leq 1$. Si hanno tre casi principali: $\rho_{XY}=1$ indica una perfetta relazione lineare positiva;  $\rho_{XY}=-1$ indica una perfetta relazione lineare negativa; $\rho_{XY}=0$ indica assenza di correlazione lineare. 
\end{oss}

Il seguito del capitolo sarà focalizzato sull'analisi di vettori gaussiani bivariati e la correlazione delle loro componenti per poi generalizzare l'approccio visivo a più dimensioni.

%%%%%%%%%%%%%%%%%%%%%%%%%%%%%%%%%%%
%%%%%% CORRELAZIONE GAUSSIANI
%%%%%%%%%%%%%%%%%%%%%%%%%%%%%%%%%%%
Si consideri un vettore gaussiano a due dimensioni: \[\mathbf{Y}=\begin{pmatrix}Y_1\\Y_2\end{pmatrix} \qquad\text{con} \qquad \bm{\mu} = \begin{pmatrix}0\\0\end{pmatrix} \quad \mathbf{\Sigma}=\begin{pmatrix}1&0\\0&1\end{pmatrix}.
\]
Generando punti dalla distribuzione di $\mathbf{Y}$ (quindi ogni punto consiste in un vettore bidimensionale) si ottiene quanto è mostrato in figura \ref{correlazione1}.

%%%%%%%%%IMMAGINE
\begin{figure}[h]
    \centering
    \includegraphics[width=0.65\textwidth]{images/Gaussiane/VettoreBivariatoIndipendenza.png}
    \caption{Nube di punti generata da una distribuzione gaussiana bivariata con componenti incorrelate \cite{wilkinson_introduction_2020}.}
    \label{correlazione1}
\end{figure}


\newpage
Si noti che la nube di punti è centrata in zero come conseguenza della scelta di $\bm{\mu}$. È facile calcolare $\rho_{Y_1,Y_2}=0$ che garantisce l'incorrelazione delle componenti del vettore gaussiano. Questo risultato si intuisce dalla forma della nube: dato un punto nella nube, la conoscenza della sua coordinata $Y_1$ non dà alcuna informazione sulla sua coordinata $Y_2$.\\
Si consideri ora $\bm{\mu} = \begin{pmatrix}0\\0\end{pmatrix}$ e $\mathbf{\Sigma}=\begin{pmatrix}1&0.9\\0.9&1\end{pmatrix}$. Generando nuovamente una nube di punti si ottiene quanto mostrato in figura \ref{correlazione2}.\\

%%%%%%%%%IMMAGINE
\begin{figure}[h]
    \centering
    \includegraphics[width=0.7\textwidth]{images/Gaussiane/VettoreBivariatoIndipendenza2.png}
    \caption{Nube di punti generata da una distribuzione gaussiana bivariata con componenti correlate \cite{wilkinson_introduction_2020}.}
    \label{correlazione2}
\end{figure}

Dalla forma della nube è ora possibile notare la correlazione tra le componenti del vettore gaussiano. Dato un punto della nube, infatti, la prima delle due componenti dà un'idea approssimativa del valore della seconda componente (e viceversa): la nube, con una certa approssimazione, si addensa intorno ad una retta passante per l'origine.\\
La forma ellittica della nube è conseguenza del valore dell'indice di correlazione: $\rho_{Y_1,Y_2}=0.9$, cioè c'è una grande correlazione lineare tra le due componenti.\\

\newpage
Evidentemente la forma ellittica si accentua all'aumentare del valore dell'indice di correlazione, come è possibile notare dalla figura \ref{correlazione3}, nella quale $\mathbf{\Sigma}=\begin{pmatrix}1&0.99\\0.99&1\end{pmatrix}$ e dunque $\rho_{Y_1,Y_2}=0.99$.

%%%%%%%%%%%IMMAGINE
\begin{figure}[h]
    \centering
    \includegraphics[width=0.7\textwidth]{images/Gaussiane/VettoreBivariatoIndipendenza3.png}
    \caption{Nube di punti generata da una distribuzione gaussiana bivariata con componenti fortemente correlate \cite{wilkinson_introduction_2020}.}
    \label{correlazione3}
\end{figure}

\newpage

Per poter generalizzare a più di due dimensioni il processo visivo di interpretazione della correlazione tra le componenti di un vettore gaussiano multivariato è necessario adottare una diversa strategia: per ogni campione (cioè per ogni vettore) generato dalla distribuzione gaussiana multivariata si traccia in un grafico il valore delle componenti collegando i rispettivi valori da un segmento.\\
In figura \ref{correlazione4} viene riportato l'esempio in due dimensioni con $\bm{\mu} = \begin{pmatrix}0\\0\end{pmatrix}$ e $\mathbf{\Sigma}=\begin{pmatrix}1&0.54\\0.54&0.3\end{pmatrix}$. Nel grafico a sinistra viene riportato l'approccio usato fino ad ora; nel grafico a destra per ogni punto del grafico a sinistra viene tracciato il valore della prima componente in corrispondenza dell'indice $1$ sulle ascisse e il valore della seconda componente in corrispondenza dell'indice $2$ sulle ascisse.


%%%%%%%%%%%IMMAGINE
\begin{figure}[h]
    \centering
    \includegraphics[width=\textwidth]{images/Gaussiane/CorrelazioneMultidimensionale.png}
    \caption{Due diversi approcci visivi alla correlazione di componenti di vettori gaussiani multivariati: $n=2$ \cite{wilkinson_introduction_2020}.}
    \label{correlazione4}
\end{figure}


\newpage 

Questo approccio permette di generalizzare a dimensioni $n>2$. Sia ora:
\[\bm{\mu}=\bm{0} \qquad\qquad \bm{\Sigma}=\begin{pmatrix}1&0.99&0.98&0.97&0.96\\0.99&1&0.99&0.98&0.97\\0.98&0.99&1&0.99&0.98\\0.97&0.98&0.99&1&0.99\\0.96&0.97&0.98&0.99&1 \end{pmatrix}
\]
dove le componenti sono fortemente correlate tra loro. Si ottiene quanto in figura \ref{correlazione5}.

%%%%%%%%%%%IMMAGINE
\begin{figure}[h]
    \centering
    \includegraphics[width=0.7\textwidth]{images/Gaussiane/CorrelazioneMultidimensionale2.png}
    \caption{Visualizzazione tramite segmenti della correlazione di componenti di vettori gaussiani multivariati: $n=5$ \cite{wilkinson_introduction_2020}.}
    \label{correlazione5}
\end{figure}

Essendo in dimensione $n=5$, ogni campione generato dal vettore gaussiano ha cinque componenti, dunque quando un campione viene tracciato nel grafico ha cinque punti in corrispondenza degli indici 1, 2, 3, 4, 5.
\newpage

In questo caso, la forte correlazione tra le componenti del vettore gaussiano si riflette nel segmento che unisce i singoli punti di ogni campione, come mostrato in figura \ref{SegmentCorrelation}.

%%%%%%%%%%%IMMAGINE
\begin{figure}[h]
    \centering
    \includegraphics[width=1\textwidth]{images/Gaussiane/CorrelazioneUnidimensionale.png}
    \caption{Debole e forte correlazione delle componenti di vettori gaussiani multivariati visualizzata tramite segmenti \cite{damianou_gaussian_2016}.}
    \label{SegmentCorrelation}
\end{figure}

Nel caso $n=50$ (con vettore norma nullo ma omettendo la matrice di covarianza, che continua ad avere uno sulla diagonale e valori prossimi a uno nelle altre entrate), si ottiene quanto in figura \ref{correlazione6}.


%%%%%%%%%%%IMMAGINE
\begin{figure}[h]
    \centering
    \includegraphics[width=0.7\textwidth]{images/Gaussiane/CorrelazioneMultidimensionale3.png}
    \caption{Visualizzazione tramite segmenti della correlazione di componenti di vettori gaussiani multivariati: $n=50$ \cite{wilkinson_introduction_2020}.}
    \label{correlazione6}
\end{figure}

Si noti quindi che all'aumentare di $n$ ciò che otteniamo inizia ad assomigliare a una funzione per ogni campione generato.\\
A partire da questa approccio è possibile interpretare i processi gaussiani come funzioni oppure come distribuzioni gaussiane multivariate infinito-dimensionali ($n=\infty$) con un indice continuo (introducendo una \textit{mean function} e una \textit{covariance function}). Questa interpretazione sarà chiarita nel prossimo capitolo.
\chapter{Processi gaussiani}\label{gaussianProcessChapter}
In questo capitono vengono introdotti i \textbf{processi gaussiani}. Inevitabilmente il capitolo non sarà esaustivo dell'argomento ma saranno trattate le sue caratteristiche principali e interessanti ai fini dell'elaborato. Per questo motivo la trattazione non si dedicherà ad inquadrare i processi gaussiani nel vasto contesto dei processi stocastici e si limiterà ad uno studio mirato delle sue peculiarità.\\
I codici python usati nel capitolo richiedono di eseguire i codici \ref{import} e \ref{import2} di import delle librerie. Inoltre la media cubica viene per tutti i kernel definita come nel codice \ref{cubedMean}. 
Il codice per le immagini dei kernel si ispira a quello scritto da Peter Roelants nel suo blog alla pagina "\href{https://peterroelants.github.io/posts/gaussian-process-tutorial/#Gaussian-processes-(1/3)---From-scratch}{Gaussian processes - From scratch}". Sono state fatte importanti modifiche al codice, in particolare vengono usati i kernel implementati dalla libreria scikit-learn.
Il codice per la predizione di dati prende invece spunto dal sito  \href{https://docs.w3cub.com/about/}{W3cubDocs} alla pagina "\href{https://docs.w3cub.com/scikit_learn/auto_examples/gaussian_process/plot_gpr_noisy_targets}{Gaussian Processes regression: basic introductory example}". Anche in questo caso sono state fatte importanti modifiche nel codice.
Il codice sulla concentrazione di $CO_2$ è una modifica del codice "\href{https://scikit-learn.org/stable/auto_examples/gaussian_process/plot_gpr_co2.html}{Gaussian process regression (GPR) on Mauna Loa CO2 data}" presente nella documentazione di scikit-learn.\\
Le fonti usate per la stesura del capitolo sono: \cite{rasmussen_gaussian_2006}, \cite{murphy_probabilistic_2022}, \cite{rasmussen_gaussian_2004}, \cite{duvenaud_automatic_2014}, \cite{gortler_visual_2019}, \cite{murphy_machine_2012}.


\begin{textblock*}{0.64\textwidth}(3.5cm+0.36\textwidth,18.5cm)
\epigraph{In applying mathematics to subjects such as physics
or statistics we make tentative assumptions about the
real world which we know are false but which we believe
may be useful nonetheless. The physicist knows that
particles have mass and yet certain results, approximating
what really happens, may be derived from the assuhption that they do not. Equally, the statistician knows, for
example, that in nature there never was a normal distribution, there never was a straight line, yet with normal and
linear assumptions, known to be false, he can often derive
results which match, to a useful approximation, those
found in the real world.}{George E. P. Box}
\end{textblock*}

\newpage


%%%%%%%%%%%%%%%%%%%%%%%%%%%%%%
%%%%% DEFINIZIONE
%%%%%%%%%%%%%%%%%%%%%%%%%%%%%%
\section{Definizione e motivazione}


\begin{defi}[Processo gaussiano]
Un \textbf{processo gaussiano} è un insieme di variabili casuali tali che ogni suo sottoinsieme finito abbia distribuzione gaussiana multivariata.
\end{defi}

Dalla definizione risulta evidente come la teoria delle distribuzioni gaussiane multivariate abbia notevole importanza nello studio dei processi gaussiani.\\
Similmente alla distribuzione gaussiana, completamente determinata dal suo vettore media e dalla sua matrice di covarianza,  i processi gaussiani sono completamente determinati da una \textit{mean function} (che ne determina la media), denotata $m(x)$, e da una \textit{covariance function} (che ne determina la covarianza), denotata $k(x,x')$. Verrà approfondito il ruolo delle due funzioni nella prossima sezione.\\
Nonostante questa similarità però, le distribuzioni gaussiane e i processi gaussiani differiscono per una importante caratteristica: le prime lavorano con vettori, i secondi con funzioni.

\vspace{0.5cm}

\begin{oss}[Motivazione: regressione nonlineare]
La principale applicazione dei processi gaussiani, nonché il tema principale di questo capitolo, è quello della \textbf{regressione nonlineare}.\\
Mentre il modello di regressione lineare cerca, a partire da dati osservati, una relazione lineare tra una variabile dipendente $Y$ e una variabile indipendente $x$, tenendo conto di un errore statistico $\epsilon$, in forma: $Y_i=\beta_0+\beta_1 x_i+\epsilon_i$, dove le incognite sono $\beta_0$ e $\beta_1$; la regressione nonlineare è una forma di regressione in cui la funzione che esprime la relazione tra variabili indipendenti e dipendenti è una combinazione non lineare dei parametri del modello e dipende da una o più variabili indipendenti, dunque in forma: $Y=f(X,\theta)+\epsilon$, dove le incognite sono $\theta$ e la funzione $f$.\\
Una importante differenza tra i due tipi di regressione è che per quella nonlineare, a differenza di quella lineare, non esiste un metodo generale per la determinazione dei valori dei parametri. 

%%%%%%%%%%%%%%%%%%%%%%%%%%%%%%%%%%%
%%%%%% IMMAGINE REGRESSIONE NONLINEARE
%%%%%%%%%%%%%%%%%%%%%%%%%%%%%%%%%%%
\begin{figure}[h]
\centering
\begin{subfigure}{.5\textwidth}
  \centering
  \includegraphics[width=\linewidth]{images/Gaussian process/motivazione2.png}
\end{subfigure}%
\begin{subfigure}{.5\textwidth}
  \centering
  \includegraphics[width=\linewidth]{images/Gaussian process/motivazione3.png}
\end{subfigure}
\caption{Regressione nonlineare \cite{turner_gaussian_2016}.}
\end{figure}


\newpage

I \textbf{processi gaussiani} forniscono un buon metodo per la determinazione dei parametri nella regressione nonlineare non solo perché sfruttano la teoria della distribuzione gaussiana multivariata, complessivamente semplice, ma perché oltre ad una funzione che  approssima i dati in modo nonlineare con massima probabilità (concetto chiarito successivamente) forniscono un'indicazione della confidenza della regressione in forma di un'area all'interno della quale funzioni interpolanti meno probabili (ma comunque possibili) risiedono, come illustrato in figura \ref{nonlinearRegressionGaussianProcess}.


%%%%%%%%%%%%%%%%%%%%%%%%%%%%%%%%%%%
%%%%%% IMMAGINE REGRESSIONE NONLINEARE GAUSSIANA
%%%%%%%%%%%%%%%%%%%%%%%%%%%%%%%%%%%
\begin{figure}[h]
    \centering
    \includegraphics[width=0.85\textwidth]{images/Gaussian process/motivazione4.png}
    \caption{Regressione nonlineare con processi gaussiani \cite{turner_gaussian_2016}.}
    \label{nonlinearRegressionGaussianProcess}
\end{figure}

\end{oss}

\newpage

\section{Introduzione ai processi gaussiani e alla notazione}
Seppur non rientri negli scopi dell'elaborato approfondire questo aspetto, viene riportata la definizione di processo stocastico per facilitare l'introduzione della notazione usata.



%%%%%%%%%%%%%%%%%%%%%%%%%%%%
%%%%%% PROCESSO STOCASTICO
%%%%%%%%%%%%%%%%%%%%%%%%%%%%
\begin{defi}[Processo stocastico]
Un \textbf{processo stocastico} su uno spazio di probabilità $(\Omega, \mathcal{F}, \mathbb{P})$ è una famiglia $\{X_t\}_{t\in T\subset \mathbb{R}}$ di variabili casuali indicizzata da un parametro $t$.
\end{defi}



È comune chiamare $t$ l'indice per sottolineare il ruolo del tempo nei processi stocastici: un processo stocastico, generalmente, descrive matematicamente l'evoluzione temporale di un sistema caratterizzato dall'essere soggetto al caso, un sistema, cioè, del quale lo stato al tempo $t$ non può essere determinato con certezza ma solo da una variabile casuale.\\
In un sistema stocastico è dunque possibile solo calcolare la \textit{probabilità} che il sistema si trovi in uno degli stati possibili ad un tempo $t>t_0$ se è noto lo stato al tempo $t_0$; in un sistema deterministico invece l'evoluzione viene descritta da regole (tipicamente in forma di equazione differenziale) che consentono di determinare con precisione lo stato del sistema ad ogni tempo $t>t_0$ noto lo stato al tempo $t_0$.




%%%%%%%%%%%%%%%%%%%%%%%%%
%%%%%%% NOTAZIONE
%%%%%%%%%%%%%%%%%%%%%%%%%
\begin{nota}[Funzioni e indicizzazione nei processi gaussiani]
Scrivendo $f\sim \mathcal{GP}(m,k)$ si  intende che "\textit{la funzione $f$ è distribuita come un processo gaussiano con mean function $m(\cdot)$ e covariance function $k(\cdot,\cdot)$}".\\
La funzione così ottenuta si può descrivere come:
\[f: \chi \rightarrow \mathbb{R},\]
dove $\chi$ è un qualsiasi dominio. Per ogni elemento $x$ nel dominio $\chi$ esiste una variabile casuale $f(x)$ a cui è associato.
\end{nota}



\begin{oss}[Versatilità dei processi gaussiani e generalizzazione dei vettori gaussiani] \label{oss1}
Si noti che il dominio $\chi$ non ha restrizioni, caratteristica che rende i processi gaussiani molto versatili. Si noti inoltre che con un dominio finito un processo gaussiano è un vettore gaussiano multivariato. In questo senso, come detto in precedenza, i processi gaussiani sono una generalizzazione dei vettori gaussiani multivariati. 
\end{oss}


\newpage
%%%%%%%%%%%%%%%%%%%%%
%%%%%% MEAN FUNCTION
%%%%%%%%%%%%%%%%%%%%%
\begin{defi}[Mean function]
Dato un processo gaussiano $f\sim \mathcal{GP}(m,k)$, la \textbf{mean function} è una funzione
\[
m:\chi \rightarrow \mathbb{R}
\]
dove $m(x)=\mathbb{E}[f(x)]$.
\end{defi}

\begin{oss}
Per la \textit{mean function} non vi è alcuna richiesta in termini di proprietà della funzione. Tipiche scelte sono $m(x)=0$ oppure $m(x)=c$ con $c$ costante.
\end{oss}

Diversamente dalla \textit{mean function}, la scelta della \textit{covariance function} è ristretta a una determinata classe di funzioni, perciò prima della definizione vengono introdotti alcuni concetti preliminari.




%%%%%%%%%%%%%%%%%%%%%%%%
%%%% MERCER KERNEL
%%%%%%%%%%%%%%%%%%%%%%%%
\begin{defi}[Mercer kernel/positive definite kernel]
Si definisce \textbf{Mercer kernel} (o \textbf{positive definite kernel}) qualunque funzione simmetrica 
\[
\mathcal{K}: \chi\times\chi \rightarrow \mathbb{R}^+
\]
 tale che: $\sum_{i=1}^N\sum_{j=1}^N\mathcal{K}(x_i,x_j)c_ic_j \geq 0$ per qualunque insieme di elementi distinti $\{x_i\}_{i=1}^N\subset \chi$, $\{c_i\}_{i=1}^N\subset  \mathbb{R}$.
\end{defi}

Una definizione alternativa si basa sul concetto di \textit{matrice di Gram}:

\begin{defi}[Matrice di Gram]
Data una funzione $\mathcal{K}: \chi\times \chi\rightarrow \mathbb{R}^+$, sia $\{x_i\}_{i=1}^N\subset \chi$ un insieme qualsiasi di elementi distinti, si definisce la \textbf{matrice di Gram} di $\mathcal{K}$ la seguente:
\[
\bm{K}=\begin{pmatrix}
    \mathcal{K}(x_1,x_1) & \mathcal{K}(x_1,x_2) & \dots & \mathcal{K}(x_1,x_N)\\
    \mathcal{K}(x_2,x_1) & \mathcal{K}(x_2,x_2) & \dots & \mathcal{K}(x_2,x_N)\\
    \vdots & \vdots & \ddots & \vdots\\
    \mathcal{K}(x_N,x_1) & \mathcal{K}(x_N,x_2) & \dots & \mathcal{K}(x_N,x_N)
    \end{pmatrix}
\]
\end{defi}

\begin{defi}[Mercer kernel/positive definite kernel]
Data una funzione $
\mathcal{K}: \chi\times\chi \rightarrow \mathbb{R}^+
$, $\mathcal{K}$ si dice \textbf{Mercer kernel} se e solo se la sua matrice di Gram è semidefinita positiva.
\end{defi}

È ora possibile definire la covariance function.

\newpage


%%%%%%%%%%%%%%%%%%%%%%%%
%%%% COVARIANCE FUNCTION
%%%%%%%%%%%%%%%%%%%%%%%%
\begin{defi}[Covariance/kernel function]
Dato un processo gaussiano $f\sim \mathcal{GP}(m,k)$, la \textbf{covariance function} è una funzione
\[
k:\chi\times \chi \rightarrow \mathbb{R}
\]
tale che $k(\cdot,\cdot)$ è un \textit{Mercer kernel}. Si ha 
\[k(x,x')=\mathbb{E}[(f(x)-m(x))(f(x')-m(x'))]=\text{Cov}[f(x),f(x')].
\]
\end{defi}

\begin{oss}[Similarità tra processi gaussiani e distribuzione gaussiana multivariata]
Si è definita la \textit{mean function} senza fare restrizioni sulle proprietà della funzione mentre per la \textit{covariance function} è stato richiesto che la sua matrice di Gram sia semidefinita positiva.\\
Questo non deve sorprendere: come precedentemente detto, i processi gaussiani generalizzano la distribuzione gaussiana multivariata e come spiegato nel precedente capitolo (e enfatizzato nell'osservazione \ref{ossGaussianaMultivariata}) quest'ultima è definita da due parametri: un vettore media, senza alcuna restrizione, e una matrice di covarianza, che deve essere simmetrica e  semidefinita positiva. Vi è dunque, anche in questo senso, similarità tra processi gaussiani e distribuzioni gaussiane multivariate.
\end{oss}

\vspace{0.5cm}

%%%%%%%%%%%%%%%%%%%%%%%%%%%%
%%%%%%% ESEMPIO GP
%%%%%%%%%%%%%%%%%%%%%%%%%%%
\begin{ese}[Esempio di processo gaussiano] \label{esempioProcessoGaussiano}
Si consideri $f\sim \mathcal{GP}(m,k)$, dove:
\[
m(x)=\frac{x^2}{4}\qquad k(x,x')=\text{exp}\left( -\frac{1}{2}(x-x')^2\right).
\]
Per comprendere questo esempio di processo gaussiano si consideri il grafico di alcuni campioni della funzione $f$. Per non lavorare nel caso infinito, si consideri un dominio finito\footnote{Si tenga conto che dal punto di vista teorico i processi gaussiani che vengono considerati in questo elaborato si costruiscono su un dominio infinito di numeri reali; tuttavia quando si generano i grafici si è costretti a considerare un insieme finito di punti che poi vengono uniti per generare la curva. Con questa (obbligata) filosofia vengono costruiti i grafici della prossima sezione.\label{footnote 1}}: $\chi = \left\{x_i \right\}_{i=1}^{n}$. Essendo $\chi$ finito, quello che si ottiene valutando $m(\cdot)$ e $k(\cdot,\cdot)$ sul dominio sono un vettore $\bm{\mu}$ e una matrice $\bm{\Sigma}$ dove:
\[
\begin{split}
\mu_i&=m(x_i)=\frac{x_i^2}{4} \qquad\qquad\qquad\qquad\qquad\qquad i=1,\dots, n \\
\Sigma_{ij}&=k(x_i,x_j)=\text{exp}\left( -\frac{1}{2}(x_i-x_j)^2\right)\qquad i,j=1,\dots, n
\end{split}
\]
Si ottiene quindi, come anticipato nell'osservazione \ref{oss1}, che per ogni $x_i$ nel dominio $\chi$ la variabile casuale $f(x_i)$ è un variabile casuale gaussiana di media $\mu_i$ e varianza $\Sigma_{ii}$; chiamando $\bm{f}=(f(x_1), \dots, f(x_n))^{\text{T}}$ si ha cioè: 
\[
\bm{f}\sim \mathcal{N}(\bm{\mu}, \bm{\Sigma}).
\]

\newpage

 Avendo ottenuto un vettore gaussiano multivariato $n$-dimensionale risulta naturale utilizzare lo stesso metodo introdotto nella sezione \ref{sezioneCorrelazione} per ricavare il grafico della "funzione" (è in realtà un vettore) $\bm{f}$. Nella figura \ref{esempioProcessoGaussianoImmagine} vengono riportati i "grafici" di quattro campioni diversi di vettore gaussiano di dimensione $n=60$. Si noti che la forma dei quattro grafici è ben diversa da quella della figura \ref{correlazione6} come conseguenza del fatto che le distribuzioni hanno matrice di covarianza diversa. Questo sottolinea quanto sia importante la covariance function nei processi gaussiani, essendo responsabile della matrice di covarianza. Questo argomento verrà analizzato in dettaglio successivamente.\newline
 Con questo esempio si è chiarito cosa si intende quando si è detto che i processi gaussiani generalizzano la distribuzione gaussiana multivariata.

%%%%%%%%%%%%%%%%%%%%%%%%%
%%%%%%%%% IMMAGINE
%%%%%%%%%%%%%%%%%%%%%%%%
\begin{figure}[h]
    \centering
    \includegraphics[width=0.85\textwidth]{images/Gaussian process/esempioProcessoGaussiano.pdf}
    \caption{Quattro vettori generati da un processo gaussiano definito come  nell'esempio \ref{esempioProcessoGaussiano}. Codice \ref{Example}.}
    \label{esempioProcessoGaussianoImmagine}
\end{figure}

Si noti che il codice \ref{Example} non genererà lo stesso grafico della figura \ref{esempioProcessoGaussianoImmagine} in quanto c'è una componente randomica! Lo stesso vale per gli altri codici citati in questo capitolo.
\end{ese}


\newpage


%%%%%%%%%%%%%%%%%%%%%%%%%%
%%%%% SULLE COVARIANCE FUNCTION
%%%%%%%%%%%%%%%%%%%%%%%%%%


\section{Sulle covariance function}
Per un processo gaussiano è di fondamentale importanza la covariance function: è infatti la funzione $k(\cdot,\cdot)$ che determina come il processo gaussiano interpreta i dati: a partire dalla definizione della covariance function è possibile ricavare diversi modelli, ad esempio la regressione lineare \footnote{Per approfondire si veda \cite{rasmussen_gaussian_2006}  e  \cite{williams_prediction_1998}} o le splines \footnote{Per approfondire si veda \cite{kimeldorf_correspondence_1970}}.\\
\begin{oss}[Importanza della covarianza]
Nel precedente capitolo, alla sezione \ref{sezioneCorrelazione}, è stata mostrata un'interpretazione visiva dell'indice di correlazione di Pearson ed è stato  chiarito come la covarianza influisca sulla correlazione di due variabili casuali.\\
Risulta quindi evidente l'importanza della covariance function nella correlazione tra le variabili casuali $f(x)$ e $f(x')$ per ogni $x,x'\in \chi$. Per questo motivo è significativo, nella scelta della covariance function, tenere conto di come questa dipenda dalla coppia $(x,x')$.
\end{oss}

Seguono i tre principali tipi di covariance function in base alla loro dipendenza da $(x,x')$ dove $x,x'\in \mathbb{R}^D$.




%%%%%%%%%%%%%%%%%%%%%%%%%%
%%%%%%% PROPRIETA KERNEL
%%%%%%%%%%%%%%%%%%%%%%%%%%
\begin{defi}[Covariance function stazionaria]
Una covariance function \textbf{stazionaria} è una covariance function che dipende da $x-x'$.\\
Una covariance function di questo tipo è invariante per traslazioni.
\end{defi}

\begin{defi}[Covariance function isotropica]
Una covariance function \textbf{isotropica} è una covariance function che dipende da $||x-x'||$.\\
Una covariance function di questo tipo è invariante per movimenti rigidi.
\end{defi}

\begin{defi}[Dot product covariance function]
Una \textbf{dot product} covariance function è una covariance function che dipende da $x$ e $x'$ solo tramite $x\cdot x'$.\\
Una covariance function di questo tipo è invariante per rotazioni centrate nell'origine ma non per traslazioni.
\end{defi}

Di seguito vengono mostrate le principali covariance function. Come già detto in precedenza, questa non può che essere un'introduzione alle possibili scelte: importanti aspetti delle covariance function riguardano la loro generalizzazione data dall'uso della distanza di Mahalanobis, una corretta scelta del kernel per l'ottimizzazione numerica, relazione tra scelta del kernel e deep learning per i processi gaussiani \footnote{Per approfondire si veda \cite{murphy_probabilistic_2022}}, adattamento dei kernel a modelli a più dimensioni \footnote{Per approfondire si veda \cite{duvenaud_automatic_2014}}... \\
In merito alle covariance function, è negli interessi di questo elaborato comprendere come queste influiscano nel processo gaussiano. Insieme ai prossimi esempi verranno quindi riportati dei grafici a supporto di questo interesse.

\newpage

\subsection{Linear kernel}
%%%%%%%%%%%%%%%%%%%%%%%%%
%%%% LINEAR
%%%%%%%%%%%%%%%%%%%%%%%
\begin{defi}[Linear kernel]
Il \textbf{linear kernel} ha forma \[
k(x,x')=\sigma_b^2+\sigma_v^2 (x-c)(x'-c).
\]
\end{defi}

Viene riportato il grafico della funzione $k(x,x')$. Si ottiene una retta con gli usuali parametri.
%%%%%%%%%%%%%%%%%%%%%%%%%
%%%%%%%%% IMMAGINE
%%%%%%%%%%%%%%%%%%%%%%%%
\begin{figure}[h]
    \centering
    \includegraphics[width=0.6\textwidth]{images/Gaussian process/Linear Kernel.pdf}
    \caption{Grafico di $k(x,x')$ linear kernel, $\sigma_b^2=1$, $\sigma_v^2=1$, $c=-1$ e $x'=1$. Codice \ref{linear kernel code}}
    \label{linear kernel}
\end{figure}

\newpage

Nella figura \ref{10 sample linear kernel zero mean} vengono mostrati grafici di funzioni con distribuzione $f\sim \mathcal{GP}(m,k)$ dove $m(x)=0$ e $k(x,x')$ è il linear kernel, $\sigma_b^2=0$, $\sigma_v^2=1$, $c=0$.



%%%%%%%%%%%%%%%%%%%%%%%%%%%%
%%%%%%%% IMMAGINE
%%%%%%%%%%%%%%%%%%%%%%%%%%%
\begin{figure}[h]
    \centering
    \includegraphics[width=0.85\textwidth]{images/Gaussian process/Linear sample.pdf}
    \caption{Grafico di funzioni con distribuzione  $f\sim \mathcal{GP}(\bm{0},k)$ dove $k(x,x')$ è il linear kernel e $\sigma_b^2=0$, $\sigma_v^2=1$, $c=0$. Codice \ref{linear sample}.}
    \label{10 sample linear kernel zero mean}
\end{figure}

Si noti che il linear kernel genera delle rette, da cui il nome.\\
Imporre la mean function uguale a zero aumenta la tendenza delle linee a passare per l'origine. Imponendo $m(x)=\alpha\in\mathbb{R}$ si aumenta la tendenza delle rette a passare per il punto $(0,\alpha)$. 
%Esploreremo di seguito l'influenza degli altri parametri del kernel sui grafici delle funzioni, ma è già in parte evidente l'importanza della struttura del kernel nel grafico delle funzioni con distribuzione $\mathcal{GP}(m,k)$.



\newpage






\vspace{1cm}
Per comprendere l'influenza del parametro $c$, vengono mostrati grafici di funzioni con distribuzione $f\sim \mathcal{GP}(m,k)$ dove $m(x)=0$ e $k(x,x')$ è il linear kernel e viene variato il valore di $c$.

%%%%%%%%%%%%%%%%%%%%%%%%%%%%%%%%%%%
%%%%%% IMMAGINI: PARAMETRO c
%%%%%%%%%%%%%%%%%%%%%%%%%%%%%%%%%%%
\begin{figure}[h]
\centering
\begin{subfigure}{.5\textwidth}
  \centering
  \includegraphics[width=\linewidth]{images/Gaussian process/Linear - c=-3.pdf}
  \caption{$c=-3$}
\end{subfigure}%
\begin{subfigure}{.5\textwidth}
  \centering
  \includegraphics[width=\linewidth]{images/Gaussian process/Linear - c=0.pdf}
  \caption{$c=0$}
\end{subfigure}
\caption{Grafico di funzioni con distribuzione $f\sim \mathcal{GP}(\bm{0},k)$ dove $k(x,x')$ è il linear kernel e $\sigma_b^2=0$, $\sigma_v^2=1$, il parametro $c$ viene variato. Codice \ref{Linear - c}.}
\label{10 sample linear modified c}
\end{figure}


Il parametro $c$ dunque impone un punto di passaggio per tutte le rette. Dunque $c$ svolge lo stesso ruolo della mean function $m(\cdot)$.

Per comprendere l'influenza del parametro $\sigma_b^2$, vengono mostrati grafici di funzioni con distribuzione $f\sim \mathcal{GP}(m,k)$ dove $m(x)=0$ e $k(x,x')$ è il linear kernel e viene variato il valore di $\sigma_b^2$.

%%%%%%%%%%%%%%%%%%%%%%%%%%%%%%%%%%%
%%%%%% IMMAGINI: PARAMETRO sigma_b
%%%%%%%%%%%%%%%%%%%%%%%%%%%%%%%%%%%
\begin{figure}[h]
\centering
\begin{subfigure}{.5\textwidth}
  \centering
  \includegraphics[width=\linewidth]{images/Gaussian process/Linear - sigmab=0.pdf}
  \caption{$\sigma_b^2=0$}
\end{subfigure}%
\begin{subfigure}{.5\textwidth}
  \centering
  \includegraphics[width=\linewidth]{images/Gaussian process/Linear - sigmab=05.pdf}
  \caption{$\sigma_b^2=0.5$}
\end{subfigure}
\caption{Grafico di funzioni con distribuzione $f\sim \mathcal{GP}(\bm{0},k)$ dove $k(x,x')$ è il linear kernel e $\sigma_v^2=1$, $c=0$, il parametro $\sigma_b^2$ viene variato. Codice \ref{Linear - sigmab}.}
\label{10 sample linear modified sigmab}
\end{figure}
Dunque il parametro $\sigma_b$ influenza la precisione con cui le funzioni tendono a passare per il punto $(0,c)$.

\newpage

Per comprendere l'influenza del parametro $\sigma_v^2$, vengono mostrati grafici di funzioni con distribuzione $f\sim \mathcal{GP}(m,k)$ dove $m(x)=0$ e $k(x,x')$ è il linear kernel e viene variato il valore di $\sigma_v^2$.


%%%%%%%%%%%%%%%%%%%%%%%%%%%%%%%%%%%
%%%%%% IMMAGINI: PARAMETRO sigma_v
%%%%%%%%%%%%%%%%%%%%%%%%%%%%%%%%%%%
\begin{figure}[h]
\centering
\begin{subfigure}{.5\textwidth}
  \centering
  \includegraphics[width=\linewidth]{images/Gaussian process/Linear - sigmav=1.pdf}
  \caption{$\sigma_v^2=1$}
\end{subfigure}%
\begin{subfigure}{.5\textwidth}
  \centering
  \includegraphics[width=\linewidth]{images/Gaussian process/Linear - sigmav=10.pdf}
  \caption{$\sigma_v^2=10$}
\end{subfigure}
\caption{Grafico di funzioni con distribuzione $f\sim \mathcal{GP}(\bm{0},k)$ dove $k(x,x')$ è il linear kernel e $\sigma_b^2=0$, $c=0$, il parametro $\sigma_v^2$ viene variato. Codice \ref{Linear - sigmav}.}
\label{10 sample linear modified sigmav}
\end{figure}

Dunque il parametro $\sigma_v^2$ influenza la pendenza delle rette, che è proporzionale al suo valore.

Per comprendere l'influenza della mean function, vengono di seguito mostrati grafici di funzioni con distribuzione $f\sim \mathcal{GP}(m,k)$ dove $m(x)=x^3$ e $k(\cdot,\cdot)$ è il linear kernel.


%%%%%%%%%%%%%%%%%%%%%%%%%%%%
%%%%%%%% IMMAGINE
%%%%%%%%%%%%%%%%%%%%%%%%%%%
\begin{figure}[h]
    \centering
    \includegraphics[width=0.85\textwidth]{images/Gaussian process/Linear - cubedmean.pdf}
    \caption{Grafico di funzioni con distribuzione $f\sim \mathcal{GP}(m,k)$ dove $m(x)=x^3$ e $k(x,x')$ è il linear kernel, $\sigma_b^2=1$, $\sigma_v^2=10$, $c=0$. Codice \ref{linear cubedmean}.}
    \label{10 sample linear kernel cubed mean}
\end{figure}


\newpage

Dal grafico è evidente che i grafici assomigliano alla funzione $x^3$. Sarà più chiaro nei prossimi esempi di kernel, ma dalla definizione di processo gaussiano (pensando alla distribuzione gaussiana multivariata, che generalizza) ogni punto è interpretabile come campione di una distribuzione gaussiana. Ricordando l'osservazione \ref{normal decomposition}, sappiamo che ogni distribuzione normale (univariata) è scomponibile in $Y=\mu+\sigma Z$ dove $Z\sim \mathcal{N}(0,1)$; dunque ogni punto $x_i$ del grafico di una funzione con distribuzione il processo gaussiano come in figura \ref{10 sample linear kernel cubed mean} ha scomposizione $x_i^3+k(x_i,x_i)Z$. È dunque chiaro dalla scomposizione di ogni punto che i grafici aggiungeranno alla funzione $x^3$ un addendo dovuto alla covariance function.





\subsection{Squared-exponential kernel}
%%%%%%%%%%%%%%%%
%%% SQUARED-EXPONENTIAL
%%%%%%%%%%%%%


%% A volte è usata ||x-x'||
\begin{defi}[Squared-exponential kernel]
Il \textbf{squared-exponential kernel} ha forma:
\[
k(x,x')=\sigma^2 \text{exp}\left( -\frac{(x-x')^2}{2l^2} \right).
\]
È dunque una covariance function isotropica.
\end{defi}

Viene riportato il grafico della funzione $k(x,x')$. Si noti che il parametro $\sigma^2$ influisce sul picco della funzione, mentre il parametro $l$ influisce indirettamente  modificandone la velocità con cui si annulla.



%%%%%%%%%%%%%%%%%%%%%%%%%
%%%%%%%%% IMMAGINE
%%%%%%%%%%%%%%%%%%%%%%%%
\begin{figure}[h]
    \centering
    \includegraphics[width=0.6\textwidth]{images/Gaussian process/Squared-exponential kernel.pdf}
    \caption{Grafico di $k(x,x')$ squared-exponential kernel, $\sigma^2=1$ e $l^2=1$. Codice \ref{squared-exponential}.}
    \label{squared-exponential kernel}
\end{figure}

\newpage

Una funzione distribuita come il processo gaussiano con questo tipo di kernel è $C^\infty$. Esistono diverse variazioni di questo kernel codificanti ipotesi leggermente diverse
sulla continuità (anche locale) della funzione ma non sono negli interessi dell'elaborato. \footnote{Per approfondire si veda \cite{duvenaud_automatic_2014}}
\vspace{0.5cm}\\

Vengono di seguito mostrati grafici di funzioni con distribuzione $f\sim \mathcal{GP}(m,k)$ dove $m(x)=0$ e $k(x,x')$ è il squared-exponential kernel.


%%%%%%%%%%%%%%%%%%%%%%%%%
%%%%%%%%% IMMAGINE
%%%%%%%%%%%%%%%%%%%%%%%%
\begin{figure}[h]
    \centering
    \includegraphics[width=0.85\textwidth]{images/Gaussian process/RBFSample.pdf}
    \caption{Grafico di funzioni con distribuzione $f\sim \mathcal{GP}(\bm{0},k)$ dove $k(x,x')$ è lo squared-exponential kernel e $l^2=1$, $\sigma^2=1$. Codice \ref{RBF sample}.}
    \label{10 sample exponential kerne zero mean}
\end{figure}

Dalla figura \ref{10 sample exponential kerne zero mean} è possibile notare la differenza che il kernel ha causato nella forma del grafico delle funzioni: paragonandolo alla figura \ref{10 sample linear kernel zero mean} risulta evidente l'importanza della scelta del kernel in funzione del contesto del suo utilizzo.\\
Come nel caso del linear kernel, imporre la mean function ad un'altra costante comporterà la traslazione delle funzioni sull'asse delle $y$.


\newpage 



Per comprendere l'influenza del parametro $\sigma^2$ vengono mostrati grafici di funzioni con distribuzione $f\sim \mathcal{GP}(m,k)$ dove $m(x)=0$ e $k(x,x')$ è il squared-exponential kernel, con $l^2=1$ e due valori diversi di $\sigma^2$.
%%%%%%%%%%%%%%%%%%%%%%%%%%%%%%%%%%%
%%%%%% IMMAGINI: PARAMETRO sigma
%%%%%%%%%%%%%%%%%%%%%%%%%%%%%%%%%%%
\begin{figure}[h]
\centering
\begin{subfigure}{.5\textwidth}
  \centering
  \includegraphics[width=\linewidth]{images/Gaussian process/RBF - sigma=01.pdf}
  \caption{$\sigma^2=0.1$}
\end{subfigure}%
\begin{subfigure}{.5\textwidth}
  \centering
  \includegraphics[width=\linewidth]{images/Gaussian process/RBF - sigma=10.pdf}
  \caption{$\sigma^2=10$}
\end{subfigure}
\caption{Grafico di funzioni con distribuzione $f\sim \mathcal{GP}(\bm{0},k)$ dove $k(x,x')$ è lo squared-exponential kernel e $l^2=1$, il parametro $\sigma^2$ viene variato. Codice \ref{RBF - sigma}.}
\label{10 sample exponential modified sigma}
\end{figure}

Nei due casi dunque cambia quanto le funzioni si distanziano dalla retta $x=0$, cioè proporzionalmente al valore di $\sigma^2$. In realtà $\sigma^2$ influisce sulla tendenza delle funzioni a distanziarsi dalla media $m(x)$.

Per comprendere l'influenza del parametro $l^2$, vengono di seguito mostrati grafici di funzioni con distribuzione $f\sim \mathcal{GP}(m,k)$ dove $m(x)=0$ e $k(x,x')$ è il squared-exponential kernel e il parametro $l^2$ viene variato.


%%%%%%%%%%%%%%%%%%%%%%%%%%%%%%%%%%%
%%%%%% IMMAGINI: PARAMETRO l
%%%%%%%%%%%%%%%%%%%%%%%%%%%%%%%%%%%
\begin{figure}[h]
\centering
\begin{subfigure}{.5\textwidth}
  \centering
  \includegraphics[width=\linewidth]{images/Gaussian process/RBF - l=03.pdf}
  \caption{$l^2=0.3$}
\end{subfigure}%
\begin{subfigure}{.5\textwidth}
  \centering
  \includegraphics[width=\linewidth]{images/Gaussian process/RBF - l=10.pdf}
  \caption{$l^2=10$}
\end{subfigure}
\caption{Grafico di funzioni con distribuzione  $f\sim \mathcal{GP}(\bm{0},k)$ dove $k(x,x')$ è lo squared-exponential kernel e $\sigma^2=1$, il parametro $l^2$ viene variato. Codice \ref{codice9}.}
\label{10 sample exponential modified l}
\end{figure}

Il parametro $l^2$ dunque modifica la frequenza di oscillazione delle funzioni.

\newpage

Per comprendere l'influenza della mean function, vengono di seguito mostrati grafici di funzioni con distribuzione $f\sim \mathcal{GP}(m,k)$ dove $m(x)=x^3$ e $k(x,x')$ lo squared-exponential kernel.
%%%%%%%%%%%%%%%%%%%%%%%%%
%%%%%%%%% IMMAGINE
%%%%%%%%%%%%%%%%%%%%%%%%
\begin{figure}[h]
    \centering
    \includegraphics[width=0.85\textwidth]{images/Gaussian process/RBF - cubedmean.pdf}
    \caption{Grafico di funzioni con distribuzione  $f\sim \mathcal{GP}(m,k)$ dove $m(x)=x^3$ e $k(x,x')$ lo squared-exponential kernel, $\sigma^2=7$ e $l=0.3$. Codice \ref{codice10}.}
    \label{10 sample exponential kernel cubed mean}
\end{figure}

Sono stati scelti dei parametri in modo da risaltare i grafici delle funzioni. Con $\sigma^2=7$ (quindi un grande distanziamento dalla mean function) e $l^2=0.3$ (quindi una grande frequenza di oscillazione) le funzioni tendono a emulare la media $m(x)=x^3$ mantenendo le proprietà indotte dal kernel.

\newpage








\newpage

\subsection{Periodic kernel}

%%%%%%%%%%%%%%%%%%%%%%%%%%%%
%%%%%%%%% PERIODIC
%%%%%%%%%%%%%%%%%%%%%%%%%%%%
\begin{defi}[Periodic kernel]
Il \textbf{periodic kernel} ha forma:
\[
k(x,x')=\sigma^2 \text{exp}\left( -\frac{2}{l^2} \text{sin}^2\left( \pi \frac{|x-x'|}{p}\right)\right).
\]
Anche questo kernel è dunque isotropico.
\end{defi}

Viene riportato il grafico della funzione $k(x,x')$. Si noti che il parametro $\sigma^2$ influisce sul picco della funzione come nel \textit{squared-esponential kernel}, similmente il parametro $l^2$ influisce sulla funzione come nel precedente kernel, il parametro $p$ influenza la periodicità del kernel. 


%%%%%%%%%%%%%%%%%%%%%%%%%
%%%%%%%%% IMMAGINE
%%%%%%%%%%%%%%%%%%%%%%%%
\begin{figure}[h]
    \centering
    \includegraphics[width=0.6\textwidth]{images/Gaussian process/Periodic kernel.pdf}
    \caption{Grafico di $k(x,x')$ periodic kernel, $\sigma^2=1$, $l^2=1$, $p=2$. Codice \ref{periodic Kernel}.}
    \label{periodic kernel}
\end{figure}




\newpage
Vengono di seguito mostrati grafici di funzioni con distribuzione $f\sim \mathcal{GP}(m,k)$ dove $m(x)=0$ e $k(x,x')$ è il periodic kernel.

%%%%%%%%%%%%%%%%%%%%%%%%%
%%%%%%%%% IMMAGINE
%%%%%%%%%%%%%%%%%%%%%%%%
\begin{figure}[h]
    \centering
    \includegraphics[width=0.85\textwidth]{images/Gaussian process/Periodic sample.pdf}
    \caption{Grafico di funzioni con distribuzione  $f\sim \mathcal{GP}(\bm{0},k)$ dove $k(x,x')$ è il periodic kernel e $\sigma^2=1$, $l^2=2$, $p=1$. Codice \ref{periodic sample}.}
    \label{3 sample periodic kerne zero mean}
\end{figure}

Come suggerisce il nome del kernel, le funzioni hanno andamento periodico.
Per comprendere l'influenza del parametro $\sigma^2$, vengono di seguito mostrati grafici di funzioni con distribuzione $f\sim \mathcal{GP}(m,k)$ dove $m(x)=0$ e $k(x,x')$ è il periodic kernel e il parametro $\sigma^2$ viene variato.

%%%%%%%%%%%%%%%%%%%%%%%%%%%%%%%%%%%
%%%%%% IMMAGINI: PARAMETRO sigma
%%%%%%%%%%%%%%%%%%%%%%%%%%%%%%%%%%%
\begin{figure}[h]
\centering
\begin{subfigure}{.5\textwidth}
  \centering
  \includegraphics[width=\linewidth]{images/Gaussian process/Periodic - sigma=01.pdf}
  \caption{$\sigma^2=0.1$}
\end{subfigure}%
\begin{subfigure}{.5\textwidth}
  \centering
  \includegraphics[width=\linewidth]{images/Gaussian process/Periodic - sigma=10.pdf}
  \caption{$\sigma^2=10$}
\end{subfigure}
\caption{Grafico di funzioni con distribuzione  $f\sim \mathcal{GP}(\bm{0},k)$ dove $k(x,x')$ è il periodic kernel e $l^2=1, p=1$, il parametro $\sigma^2$ viene variato. Codice \ref{Periodic sigma}.}
\label{10 sample periodic modified sigma}
\end{figure}

Il parametro $\sigma^2$ è dunque responsabile dell'allontamento delle funzioni dalla media, esattamente come per lo squared-exponential kernel (la figura \ref{10 sample exponential modified sigma} riporta infatti risultati simili).


\newpage

Per comprendere l'influenza del parametro $p$, vengono di seguito mostrati grafici di funzioni con distribuzione $f\sim \mathcal{GP}(m,k)$ dove $m(x)=0$ e $k(x,x')$ è il periodic kernel e il parametro $p$ viene variato.

%%%%%%%%%%%%%%%%%%%%%%%%%%%%%%%%%%%
%%%%%% IMMAGINI: PARAMETRO p
%%%%%%%%%%%%%%%%%%%%%%%%%%%%%%%%%%%
\begin{figure}[h]
\centering
\begin{subfigure}{.5\textwidth}
  \centering
  \includegraphics[width=\linewidth]{images/Gaussian process/Periodic - p=08.pdf}
  \caption{$p=0.8$}
\end{subfigure}%
\begin{subfigure}{.5\textwidth}
  \centering
  \includegraphics[width=\linewidth]{images/Gaussian process/Periodic - p=10.pdf}
  \caption{$p=10$}
\end{subfigure}
\caption{Grafico di funzioni con distribuzione  $f\sim \mathcal{GP}(\bm{0},k)$ dove $k(x,x')$ è il periodic kernel e $\sigma^2=1$, $l^2=1$ e il parametro $p$ viene variato. Codice \ref{periodic p}.}
\label{10 sample periodic modified p}
\end{figure}

Il parametro $p$ influenza dunque il periodo delle funzioni.

Per comprendere l'influenza del parametro $l^2$, vengono di seguito mostrati grafici di funzioni con distribuzione $f\sim \mathcal{GP}(m,k)$ dove $m(x)=0$ e $k(x,x')$ è il periodic kernel e il parametro $l^2$ viene variato.

%%%%%%%%%%%%%%%%%%%%%%%%%%%%%%%%%%%
%%%%%% IMMAGINI: PARAMETRO l
%%%%%%%%%%%%%%%%%%%%%%%%%%%%%%%%%%%
\begin{figure}[h]
\centering
\begin{subfigure}{.5\textwidth}
  \centering
  \includegraphics[width=\linewidth]{images/Gaussian process/Periodic - l=0.8.pdf}
  \caption{$l^2=0.8$}
\end{subfigure}%
\begin{subfigure}{.5\textwidth}
  \centering
  \includegraphics[width=\linewidth]{images/Gaussian process/Periodic - l=4.pdf}
  \caption{$l^2=4$}
\end{subfigure}
\caption{Grafico di funzioni con distribuzione  $f\sim \mathcal{GP}(\bm{0},k)$ dove $k(x,x')$ è il periodic kernel e $\sigma^2=1$, $p=1$ e il parametro $l^2$ viene variato. Codice \ref{periodic l}.}
\label{10 sample periodic modified l}
\end{figure}

Il parametro $l^2$ influenza dunque la "morbidezza" della frequenza delle funzioni.


\newpage 
Per comprendere l'influenza della mean function, vengono di seguito mostrati grafici di funzioni con distribuzione $f\sim \mathcal{GP}(m,k)$ dove $m(x)=x^3$ e $k(x,x')$ il periodic kernel.
%%%%%%%%%%%%%%%%%%%%%%%%%
%%%%%%%%% IMMAGINE
%%%%%%%%%%%%%%%%%%%%%%%%
\begin{figure}[h]
    \centering
    \includegraphics[width=0.85\textwidth]{images/Gaussian process/Periodic - cubedmean.pdf}
    \caption{Grafico di funzioni con distribuzione  $f\sim \mathcal{GP}(m,k)$ dove $m(x)=x^3$ e $k(x,x')$ il periodic kernel, $\sigma^2=8$, $l^2=1$ e $p=0.5$. Codice \ref{priodic cubedmean}.}
    \label{3 sample periodic kernel cubed mean}
\end{figure}

Sono stati scelti dei parametri in modo da risaltare i grafici delle funzioni. Con $\sigma^2=8$ (quindi un grande distanziamento dalla mean function) e $p=0.5$ (quindi una grande frequenza di oscillazione) e $l^2=1$ (quindi molto "spigolose") le funzioni tendono a emulare la media $m(x)=x^3$ mantenendo le proprietà indotte dal kernel.

\newpage



\subsection{Covariance function in più dimensioni}\label{multidimensionalKernel}
Si possono estendere le covariance function in più dimensioni in diversi modi. Poiché l'estensione a più dimensioni è pressocchè identica per ogni covariance function, viene mostrato il caso del squared-exponential kernel, in quanto verrà utilizzato nella parte di elaborato dedicata al training.\\
Per estendere il squared-exponential kernel è necessario estendere in più dimensioni la sottrazione $x-x'$. Il modo più semplice è considerare la norma $||x-x'||$, un modo più flessibile è considerare $(x-x')^\text{T}M(x-x')$ dove $M$ è una matrice.
In questo modo è possibile generalizzare il caso:
\[
k(x,x')=\sigma^2 \text{exp}\left( -\frac{||x-x'||^2}{2l^2} \right),
\]
che si può ottenere come:
\[
k(x,x')=\sigma^2 \text{exp}\left( -\frac{(x-x')^\text{T}M(x-x')}{2} \right)
\]
imponendo $M=l^{-2}I$.\\
La struttura del kernel con matrice permette di dotare ogni dimensione di una differente scala di lunghezza $l_i$, imponendo $M=\text{diag}(\bf{l})^{-2}$, dove $\mathbf{l}=(l_1,...,l_n)^\text{T}$. Questa caratteristica risulta fondamentale nel training (trattato successivamente nell'elaborato) perché se uno di questi $l_i$ diventa grande, la dimensione corrispondente (che si vedrà corrispondere ad un parametro da ottimizzare) "perde di rilevanza". L'immagine \ref{multidimensional} fornisce un esempio di un comportamento simile.

%%%%%%%%%%%%%%%%%%%%%%%%%
%%%%%%%%% IMMAGINE
%%%%%%%%%%%%%%%%%%%%%%%%
\begin{figure}[h]
    \centering
    \includegraphics[width=0.6\textwidth]{images/Gaussian process/Multidimensional RBF.pdf}
    \caption{Grafico di funzione con distribuzione $f\sim \mathcal{GP}(0,k)$ con $k(x,x')$ lo squared-exponential kernel in due dimensioni in cui $M=\text{diag}(1,3)^{-2}$. La funzione tende a cambiare più velocemente lungo la direzione $x_1$ che lungo la direzione $x_2$. \cite{murphy_machine_2012}}
    \label{multidimensional}
\end{figure}




\newpage

\subsection{Combinare le covariance function}
Nelle precedenti sezioni è stata spiegata l'influenza della covariance function nella forma del grafico delle funzioni. Tuttavia non è raro dover usare funzioni con forme diverse da quelle imposte dalle covariance function precedentemente introdotte. In tal caso è possibile costruire un nuovo kernel secondo la proposizione \ref{combining kernel}.


\begin{prop} \label{combining kernel}
Dati due kernel $K_1(x,x')$ e $K_2(x,x')$, per le proprietà di un kernel i seguenti sono ancora kernel:
\[
\begin{array}{l}
    K(x,x') = c\cdot K_1(x,x') \quad \forall c>0 \text{ costante}\\
    K(x,x') = f(x)K_1(x,x')f(x') \quad \forall f \text{ funzione}\\
    K(x,x') = q(K_1(x,x')) \quad \forall q \text{ funz. polin. a coeff. non negativi}\\
    K(x,x') = \text{exp}(K_1(x,x'))\\
    K(x,x') = K_1(x,x')+K_2(x,x')\\
    K(x,x') = K_1(x,x')\times K_2(x,x')\\
\end{array}
\]
\end{prop}
Non è negli interessi dell'elaborato studiare a fondo le combinazioni possibili dei kernel introdotti; vengono però portati due semplici casi di combinazioni di kernel a titolo puramente illustrativo.

\newpage




\subsubsection{Squared exponential + periodic kernel}


%%%%%%%%%%%%%%%%%%%%%%%%%
%%%%%%%%% IMMAGINE
%%%%%%%%%%%%%%%%%%%%%%%%
\begin{figure}[h]
    \centering
    \includegraphics[width=0.55\textwidth]{images/Gaussian process/RBF + periodic kernel.pdf}
    \caption{Grafico di $k(x,x')$ squared-exponential kernel sommato a periodic kernel. $\sigma^2=1$, $l=2$, $p=1$. Codice \ref{RBF + periodic kernel}.}
    \label{SE + periodic kernel}
\end{figure}



%%%%%%%%%%%%%%%%%%%%%%%%%
%%%%%%%%% IMMAGINE
%%%%%%%%%%%%%%%%%%%%%%%%
\begin{figure}[h]
    \centering    \includegraphics[width=0.69\textwidth]{images/Gaussian process/RBF + periodic sample.pdf}
    \caption{Grafico di funzioni con distribuzione  $f\sim \mathcal{GP}(\bm{0},k)$ dove $k(x,x')$ è lo squared-exponential sommato al periodic kernel e $l^2=1.1$, $\sigma^2=1$, $p=1.1$. Codice \ref{RBF + periodic sample}.}
    \label{SE + periodic sample}
\end{figure}

Combinare i due kernel genera dunque funzioni periodiche con delle perturbazioni sulla coppia degli assi.

\newpage

\subsubsection{Linear $\times$ linear kernel}

%%%%%%%%%%%%%%%%%%%%%%%%%
%%%%%%%%% IMMAGINE
%%%%%%%%%%%%%%%%%%%%%%%%
\begin{figure}[h]
    \centering
    \includegraphics[width=0.55\textwidth]{images/Gaussian process/Linear x Linear kernel.pdf}
    \caption{Grafico di $k(x,x')$ linear kernel moltiplicato a linear kernel. $\sigma_b^2=0$, $\sigma_v^2=1$, $c=0$, $x'=1$. Codice \ref{linear x linear}.}
    \label{linear * linear kernel}
\end{figure}


%%%%%%%%%%%%%%%%%%%%%%%%%
%%%%%%%%% IMMAGINE
%%%%%%%%%%%%%%%%%%%%%%%%
\begin{figure}[h]
    \centering
    \includegraphics[width=0.68\textwidth]{images/Gaussian process/linear x linear sample.pdf}
    \caption{Grafico di funzioni con distribuzione  $f\sim \mathcal{GP}(\bm{0},k)$ dove $k(x,x')$ è il linear kernel moltiplicato al linear kernel e $\sigma_b^2=0$, $\sigma_v^2=1$, $c=0$. Codice \ref{linear x linear sample}.}
    \label{linear * linear sample}
\end{figure}


Combinare i due kernel genera dunque funzioni con comportamento parabolico. In un certo senso era possibile prevedere questo risultato ricordando che il linear kernel genera delle rette.

\newpage



\subsubsection{Esempio reale di covariance function composta}\label{section: mauna loa}
Viene ripreso da \cite{rasmussen_gaussian_2006} un esempio di regressione in cui è necessario comporre più kernel function.  I dati consistono in concentrazioni medie mensili di $CO_2$ atmosferica (in parti per milione di volume: $ppmv$) derivate da campioni d'aria raccolti presso l'osservatorio di Mauna Loa, nelle Hawaii, tra il 1958 e il 2003 (con alcuni valori mancanti). L'obiettivo è modellare la concentrazione di $CO_2$ in funzione del tempo $x$, cioè predire quella che è conosciuta come \textit{curva di Keeling}. Dalla figura \ref{CO2} sono evidenti alcune caratteristiche: una tendenza all'aumento a lungo termine, una pronunciata variazione stagionale e alcune irregolarità minori. 


%%%%%%%%%%%%%%%%%%%%%%%%%
%%%%%%%%% IMMAGINE
%%%%%%%%%%%%%%%%%%%%%%%%
\begin{figure}[ht]
    \centering
    \includegraphics[width=1\textwidth]{images/Gaussian process/Co2_example.PNG}
    \caption{545 osservazioni delle medie mensili della concentrazione atmosferica di $CO_2$ tra il 1958 e il 2003, inoltre viene mostrata la regione di confidenza del $95\%$ per un modello di regressione di processo gaussiano a 20 anni nel futuro. \cite{rasmussen_gaussian_2006}}
    \label{CO2}
\end{figure}


Per modellare l'andamento regolare e crescente a lungo termine viene utilizzato uno squared-exponential kernel:
\[
k_1(x,x')=\theta_1^2\text{exp}\left(-\frac{(x-x')^2}{2\theta_2^2}\right).
\]


Viene usato il periodic kernel con un periodo di un anno per modellare la variazione stagionale. Poiché l'andamento stagionale non è esattamente periodico, viene considerato il prodotto con un squared-exponential kernel per consentire un decadimento non esattamente periodico: 
\[
k_2(x, x') = \theta^2_3 \text{exp}\left( - \frac{(x - x')^2}{2\theta^2_4} - \frac{2 \text{sin}^2(\pi(x - x'))}{\theta^2_5} \right).
\]
Per modellare le (piccole) irregolarità a medio termine viene utilizzato un \textit{rational quadratic} kernel (non è stato introdotto nell'elaborato): 
\[
k_3(x, x') = \theta^2_6 \left( 1 + \frac{(x - x')^2}{2\theta_8\theta^2_7} \right)^{-\theta_8}.
\] 
Infine, si modella il rumore come somma di un contributo di uno squared-exponential e di una componente indipendente:
\[
k_4(x_p, x_q) = \theta^2_9 \text{exp}\left( \frac{- (x_p - x_q)^2}{2\theta^2_{10}} \right) + \theta^2_{11} \delta_{pq}.
\]

Il kernel complessivo risulta:
\[
k(x,x')=k_1(x,x')+k_2(x,x')+k_3(x,x')+k_4(x,x'),
\]
dove si hanno $\bm{\theta}=(\theta_1,...,\theta_{11})$ iperparametri\footnote{Si veda \ref{gerarchica}.}. Dopo una fase di training del modello vengono dati dei valori ai $\theta_i$\footnote{Viene spiegato come questo venga fatto nel capitolo \ref{machineLearning}.}. La figura \ref{CO2} mostra come il modello predice l'andamento della concentrazione atmosferica di $CO_2$ nei venti anni successivi all'ultima misurazione, mostrando anche la regione di confidenza del novantacinque percento. Si nota che più si va avanti nel tempo e più la regione di confidenza si allarga.\\
Intuitivamente il modello sembra seguire bene l'andamento del grafico, seppur più avanti negli anni la regione di incertezza diventa più ampia, informando poco sull'effettiva concentrazione di $CO_2$. Nella figura \ref{CO2_comparison} viene comparata la predizione del modello con i dati reali presi dal sito del \href{https://gml.noaa.gov/ccgg/trends/}{Global Monitoring Laboratory}.


%%%%%%%%%%%%%%%%%%%%%%%%%
%%%%%%%%% IMMAGINE
%%%%%%%%%%%%%%%%%%%%%%%%
\begin{figure}[h]
    \centering
    \includegraphics[width=1\textwidth]{images/Gaussian process/MaunaLoaPrediction.pdf}
    \caption{Comparazione della predizione della concentrazione di $CO_2$ con i dati reali fino a maggio 2022.}
    \label{CO2_comparison}
\end{figure}

\newpage

Si nota dunque che la previsione è stata conservativa nel lungo termine: il grande aumento nello concentrazione di $CO_2$ è dovuto, secondo alcune fonti, a come la flora ha risposto ai cambiamenti climatici (siccità e precipitazioni principalmente), ma anche alle grandi emissioni dovute all'uso di carburanti fossili.

In un certo senso, dunque, è ragionevole che la predizione non sia precisa negli ultimi anni poiché avrebbe dovuto prevedere eventi (dai cambiamenti climatici ai diversi tassi di consumo di carburante) che non erano presenti negli anni in cui il processo gaussiano "ha imparato".

In figura \ref{CO2_comparison_zoomed} viene ingrandita l'immagine precedente dal 1995 al 2022, dunque enfatizzando il periodo di tempo che il processo gaussiano ha predetto.

Si nota che dopo il 2003 la predizione è piuttosto imprecisa, non riuscendo a stare dietro ai veloci cambiamenti umani (in termini di inquinamento).  

%%%%%%%%%%%%%%%%%%%%%%%%%
%%%%%%%%% IMMAGINE
%%%%%%%%%%%%%%%%%%%%%%%%
\begin{figure}[h]
    \centering
    \includegraphics[width=1\textwidth]{images/Gaussian process/MaunaLoaPredictionZoom.pdf}
    \caption{Comparazione della predizione della concentrazione di $CO_2$ con i dati reali fino dal 1995 a maggio 2022.}
    \label{CO2_comparison_zoomed}
\end{figure}




\newpage



\section{Predizioni con osservazioni senza rumore}\label{regressioneGP}
In questa sezione viene spiegato come sfruttare le informazioni fornite dai \textit{training data} per generare una funzione che incorpori la conoscenza a priori nel caso di osservazioni senza rumore. \\


\begin{defi}[Noise-free training set]
Il \textbf{noise-free training set} è l'insieme di osservazioni \textit{noise-free} definito come: $\mathcal{D}=\{(x_n,y_n): n=1,...,N\}$ dove $y_n=f(x_n)$ è il valore osservato della funzione $f$ nel punto $x_n$. 
\end{defi}


Lo scopo del (noise-free) training set è quello di definire un insieme di punti $\mathcal{P}=\{x_n: n=1,...,N\}$ dei quali si conosce il valore della funzione per richiedere al processo gaussiano di generare funzioni che interpolino questo insieme di punti (dunque che abbiano valore $y_n$ su ogni $x_n\in \mathcal{P}$ senza incertezza).

\begin{defi}[Test set]
il \textbf{test set} è l'insieme di valori $X_*=\{x_n:n=1,...,N^*\}$ dei quali si vuole avere la predizione, cioè l'output della funzione.
\end{defi}

Pragmaticamente, quello che si sta facendo definendo il \textit{training set} $\mathcal{D}$ è aggiungere i punti di $\mathcal{P}$ all'insieme di punti su cui valutare la covariance function per costruire la matrice di covarianza\footnote{Come accennato in \ref{footnote 1}, per ottenere un grafico è necessario partire da un dominio finito. Per questo motivo si parla di \textit{matrice di covarianza}.}. Tramite il processo di condizionamento si ottiene la distribuzione \textit{a posteriori}, la distribuzione cioè delle funzioni che interpolano i punti del training set, come illustrato nella figura \ref{intuitiveExplanationOfConditioning}.


%%%%%%%%%%%%%%%%%%%%%%%%%
%%%%%%%%% IMMAGINE
%%%%%%%%%%%%%%%%%%%%%%%%
\begin{figure}[h]
    \centering
    \includegraphics[width=1\textwidth]{images/Gaussian process/GPposterior.PNG}
    \caption{Spiegazione grafica di come viene incorporata la conoscenza a priori \cite{gortler_visual_2019}.}
    \label{intuitiveExplanationOfConditioning}
\end{figure}

\newpage
Si consideri ora una funzione distribuita come $f\sim \mathcal{GP}(m,k)$. 
Si consideri ora $X^*=\{x_i^*:i=1,\dots,N^*\}$ un \textit{test set} di cui si vuole predire gli output $\bm{f^*}=\left[f(x^*_1), \dots, f(x^*_{N^*}) \right]$; siano inoltre  $\bm{f}=\left[f(x_1), \dots, f(x_N) \right]$ gli output del training set dove $x_i\in \mathcal{P}$. Ricordando che gli output della funzione $f$ seguono una distribuzione gaussiana, è possibile ricavare la distribuzione congiunta:
\[
\begin{pmatrix}
\bm{f}\\
\bm{f^*}
\end{pmatrix}
=
\mathcal{N}\left(
\begin{pmatrix}
\bm{\mu}\\
\bm{\mu_*}
\end{pmatrix},
\begin{pmatrix}
\bm{K}_{X,X} & \bm{K}_{X,X^*}\\
\bm{K}_{X^*,X} & \bm{K}_{X^*,X^*}
\end{pmatrix}
\right)
\]

dove $\bm{\mu}=\begin{pmatrix}m(x_1) \\ \vdots \\ m(x_N)\end{pmatrix}$, $\bm{\mu_*}=\begin{pmatrix}m(x^*_1) \\ \vdots \\ m(x^*_{N^*})\end{pmatrix}$, $\bm{K}_{X,X}=k(X,X)$ è una matrice $N\times N$, $\bm{K}_{X,X^*}=k(X,X^*)$ è una matrice $N\times N^*$, $\bm{K}_{X^*,X}=k(X^*,X)$ è una matrice $N^*\times N$, $\bm{K}_{X^*,X^*}=k(X^*,X^*)$ è una matrice $N^*\times N^*$.\\

Dalla proposizione \ref{marginale-condizionata} ricaviamo la distribuzione condizionata di $\bm{f^*} | X^*, \mathcal{D}$:

\[
\bm{f^*} | X^*, \mathcal{D} \sim \mathcal{N}(\bm{\mu^*}, \bm{\Sigma^*})
\]

dove: 
\[
\begin{split}
\bm{\mu^*}=m(X^*)+\bm{K}_{X,X^*}^\text{T}\bm{K}^{-1}_{X,X}(\bm{f}-m(X))\\
\bm{\Sigma^*}=\bm{K}_{X^*,X^*}-\bm{K}_{X,X^*}^\text{T}\bm{K}^{-1}_{X,X}\bm{K}_{X,X^*}
\end{split}
\]


\newpage 
In figura \ref{Interpolation} viene mostrato un esempio grafico di interpolazione di sei punti.


%%%%%%%%%%%%%%%%%%%%%%%%%
%%%%%%%%% IMMAGINE
%%%%%%%%%%%%%%%%%%%%%%%%
\begin{figure}[h]
    \centering
    \includegraphics[width=1\textwidth]{images/Gaussian process/Noise-free - mean&f(x).pdf}
    \caption{Grafico di funzioni con distribuzione $f\sim \mathcal{GP}(\bm{0},k)$ dove $k(\cdot,\cdot)$ è il squared-exponential kernel; il processo gaussiano è stato condizionato per interpolare sei punti. Viene mostrata in rosso la funzione da cui sono stati scelti i punti da interpolare, in blu la media del processo gaussiano condizionato, come linee tratteggiate alcuni sample del processo gaussiano. Codice \ref{interpolation code}.}
    \label{Interpolation}
\end{figure}

La media del processo gaussiano condizionato per interpolare i sei punti è la più affidabile nel predire la funzione $x\cdot sin(x)$ da cui sono stati generati i punti da interpolare. Le funzioni con distribuzione il processo gaussiano condizionato hanno tutte la proprietà di interpolazione dei suddetti punti, tuttavia nel resto del piano si distanziano più facilmente dalla media, anche se tendono a restare nei suoi dintorni. Nella figura \ref{Interpolation confidence region} vengono sfruttate le informazioni della matrice di covarianza a posteriori per disegnare la regione di confidenza del 95\%, dentro la quale la maggior parte dei sample risiede.


\newpage
%%%%%%%%%%%%%%%%%%%%%%%%%
%%%%%%%%% IMMAGINE
%%%%%%%%%%%%%%%%%%%%%%%%
\begin{figure}[h]
    \centering
    \includegraphics[width=1\textwidth]{images/Gaussian process/Noise-free - mean&samples.pdf}
    \caption{Grafico di funzioni con distribuzione $f\sim \mathcal{GP}(\bm{0},k)$ dove $k(\cdot,\cdot)$ è il squared-exponential kernel; il processo gaussiano è stato condizionato per interpolare sei punti. Viene mostrata in blu la regione di confidenza al 95\%, in blu la media del processo gaussiano condizionato e come linee tratteggiate alcuni sample. Codice \ref{interpolation confidence region code}.}
    \label{Interpolation confidence region}
\end{figure}

Come anticipato, all'interno dell'area data da $\bm{\mu^*}(x_i)\pm 1.96\bm{\Sigma^*}(x_i,x_i)$ (regione di confidenza al 95\%) risiedono la maggior parte dei sample della distribuzione a posteriori. Più ci si distanzia dai punti interpolati e più facilmente le funzioni si discostano dalla media, tendendo ad uscire dall'area di incertezza, come si osserva ad esempio nella zona nei pressi di $x=10$ della figura \ref{Interpolation confidence region}.



\section{Predizioni con osservazioni rumorose}\label{noisyPrediction}
In questa sezione viene spiegato come sfruttare le informazioni fornite dai \textit{training data} per generare una funzione che incorpori la conoscenza a priori nel caso di osservazioni con rumore. \\

\begin{defi}[Noisy training set]
Il \textbf{noisy training set} è l'insieme di osservazioni \textit{noisy} (cioè con una componente di errore) definito come: $\mathcal{D}=\{(x_n,y_n): n=1,...,N\}$ dove $y_n=f(x_n)+\epsilon$ è il valore osservato della funzione $f$ nel punto $x_n$ con una componente di rumore $\epsilon\sim \mathcal{N}(0,\sigma_n^2)$ indipendente e identicamente distribuito. 
\end{defi}
In questo caso si ha:
\[
\text{Cov}(y_p,y_q)=k(x_p,x_q)+\sigma_n^2 \delta_{pq},
\]
o equivalentemente:
\[
\text{Cov}(\mathbf{y})=k(X,X)+\sigma_n^2I.
\]

\newpage

Viene usata una notazione analoga alla sezione \ref{regressioneGP}: $X^*=\{x_i^*:i=1,\dots,N^*\}$ è il \textit{test set} di cui si vogliono predire gli output $\bm{f^*}=\left[f(x^*_1), \dots, f(x^*_{N^*}) \right]$;  $\bm{y}=\left[f(x_1)+\epsilon, \dots, f(x_N)+\epsilon \right]$ gli output del training set. \\
Si procede come nel caso della predizione con osservazioni senza rumore:
\[
\begin{pmatrix}
\bm{y}\\
\bm{f^*}
\end{pmatrix}
=
\mathcal{N}\left(
\begin{pmatrix}
\bm{\mu}\\
\bm{\mu_*}
\end{pmatrix},
\begin{pmatrix}
\bm{K}_{X,X}+\sigma_n^2I & \bm{K}_{X,X^*}\\
\bm{K}_{X^*,X} & \bm{K}_{X^*,X^*}
\end{pmatrix}
\right)
\]
Si ricava:
\[
\bm{f^*} | X^*, \mathcal{D} \sim \mathcal{N}(\bm{\mu^*}, \bm{\Sigma^*})
\]

dove: 
\[
\begin{split}
\bm{\mu^*}=m(X^*)+\bm{K}_{X,X^*}^\text{T}(\bm{K}_{X,X}+\sigma_n^2I)^{-1}(\bm{y}-m(X))\\
\bm{\Sigma^*}=\bm{K}_{X^*,X^*}-\bm{K}_{X,X^*}^\text{T}(\bm{K}_{X,X}+\sigma_n^2I)^{-1}\bm{K}_{X,X^*}
\end{split}
\]


In figura \ref{Noisy} viene mostrato un esempio grafico di come un processo gaussiano interpreti le informazioni date da osservazioni con rumore per predire una funzione.
Come ci si può aspettare, essendo le osservazioni in questo caso con rumore, il processo gaussiano ha bisogno di più osservazioni per poter predire l'andamento della funzione. Non verificandosi interpolazione, il processo gaussiano predice la funzione con minore precisione.
%%%%%%%%%%%%%%%%%%%%%%%%%
%%%%%%%%% IMMAGINE
%%%%%%%%%%%%%%%%%%%%%%%%
\begin{figure}[h]
    \centering
    \includegraphics[width=1\textwidth]{images/Gaussian process/Noise - mean&f(x).pdf}
    \caption{Grafico di funzioni con distribuzione $f\sim \mathcal{GP}(\bm{0},k)$ dove $k(\cdot,\cdot)$ è il squared-exponential kernel; il processo gaussiano è stato condizionato per predire una funzione a partire dalle sue osservazioni rumorose. Viene mostrata in rosso la funzione da predire, in rosso i punti osservati della funzione con le barre rappresentanti il rumore, in blu la media del processo gaussiano condizionato, come linee tratteggiate alcuni sample del processo gaussiano. Codice \ref{Noise code}.}
    \label{Noisy}
\end{figure}

\newpage

Anche in questo caso la media del processo gaussiano condizionato rappresenta la funzione che meglio predice $x\cdot sin(x)$. Poiché le osservazioni presentano un errore, non si ha interpolazione dei punti seppur, tendenzialmente, la media e i sample del processo gaussiano tocchino almeno le barre di errore di ogni punto.
Sono stati presi in considerazione più punti del caso noise-free poiché con meno punti non si sarebbe ottenuto un risultato così "preciso" (comunque con un comportamento diverso da quello con osservazioni senza rumore).

Se nel caso senza rumore i sample tendenvano a rimanere nei pressi della media nelle vicinanze di ogni punto interpolato, qui la vicinanza alla media dipende sia dai punti che dal loro errore: nei pressi di punti con minore rumore (cioè con barre più corte) le funzioni tendono ad avvicinarsi alla media. Viceversa, nei pressi di funzioni con un grande errore (barre più lunghe) le funzioni tendono ad allontanarsi con più facilità dalla media.

Anche in questo caso è possibile disegnare la regione di confidenza del 95\%. La regione di confidenza ha ampiezza proporzionale alla lunghezza delle barre delle osservazioni; di conseguenza la regione di confidenza ha comportamento decisamente diverso dal caso noise-free.

%%%%%%%%%%%%%%%%%%%%%%%%%
%%%%%%%%% IMMAGINE
%%%%%%%%%%%%%%%%%%%%%%%%
\begin{figure}[h]
    \centering
    \includegraphics[width=1\textwidth]{images/Gaussian process/Noise - mean&samples.pdf}
    \caption{Grafico di funzioni con distribuzione $f\sim \mathcal{GP}(\bm{0},k)$ dove $k(\cdot,\cdot)$ è il squared-exponential kernel; il processo gaussiano è stato condizionato per predire una funzione a partire dalle sue osservazioni rumorose. Viene mostrata in blu la regione di confidenza al 95\%, in rosso con le barre di errore le osservazioni, in blu la media del processo gaussiano condizionato e come linee tratteggiate alcuni sample. Codice \ref{Noise confidence region code}.}
    \label{Noisy confidence region}
\end{figure}
\chapter{Machine learning}\label{machineLearning}

Il capitolo spiega in breve la teoria dell'\textbf{apprendimento supervisionato} focalizzandosi nel contesto di regressione con processi gaussiani. Viene introdotto teoricamente il modello statistico che verrà usato, nel capitolo \ref{Capitolo: risultati training}, per produrre risultati legati al modello Windkessel visto nel capitolo \ref{windkessel}.\\
Le fonti usate per la stesura del capitolo sono: \cite{murphy_probabilistic_2022}, \cite{wiki:datasets}, \cite{wiki:overfitting}, \cite{pytorch:R2score}, \cite{wang_intuitive_2022}, \cite{noauthor_tutorial_nodate}, \cite{murphy_machine_2012}, \cite{gelman_bayesian_1995}, \cite{wiki:gradientDescend}, \cite{ruder_2022}, \cite{kingma_adam_2017}, \cite{JMLR:v12:duchi11a}, \cite{bottou2012stochastic}.


\begin{textblock*}{0.64\textwidth}(3.5cm+0.36\textwidth,18.5cm)
\epigraph{Il sole è sorto e tramontato per miliardi di anni. Il sole è tramontato anche stanotte. Con un'elevata probabilità, il sole domani sorgerà. Ma questo numero è molto più grande per colui che, vedendo nella totalità dei fenomeni il principio che regola i giorni e le stagioni, si rende conto che nulla al momento attuale può arrestarne il corso.}{Pierre Simon Laplace}
\end{textblock*}

\newpage

\section{Introduzione al machine learning}
Il \textbf{machine learning} (in italiano \textit{apprendimento automatico}) è una branca dell'\textbf{intelligenza artificiale}, la quale è una disciplina che studia i fondamenti teorici e le tecniche che consentono la progettazione di sistemi capaci di fornire all'elaboratore prestazioni che sembrerebbero essere di pertinenza esclusiva dell’intelligenza umana.\\
Di questa vasta branca, l'elaborato si interessa ai soli problemi di regressione, contrapposti ai problemi di classificazione.

In termini più pratici, il principale scopo del machine learning è lo studio e la costruzione di algoritmi che possano imparare a fare previsioni su dei dati. Tali algoritmi funzionano basandosi su decisioni guidate dal dataset  attraverso la costruzione di un modello matematico.\\
Nell'elaborato viene usato uno specifico tipo di apprendimento che è quello \textit{supervisionato}.

\begin{defi}[Apprendimento supervisionato]
L'\textbf{apprendimento supervisionato} è una tecnica di apprendimento automatico che mira a istruire un sistema informatico in modo da consentirgli di elaborare previsioni sulla base di una serie di esempi costituiti da coppie di input e output.
\end{defi}

\section{Dataset}\label{dataset}

To optimise the construction of predictive algorithms, the input data is divided into several datasets with different roles. Typically, a specific partition of the data into three datasets is used: \textbf{training}, \textbf{validation} and \textbf{test set}.


\begin{defi}[Training set]
    The \textbf{training set} is a set of examples used during the learning process to determine (or learn) optimal combinations of parameters.
\end{defi}

In practice, it is on the data of the training set that the chosen optimisation method is performed, thus updating the values of the parameters or hyper-parameters.

\begin{defi}[Validation set]
    The \textbf{validation set} is an independent data set used to evaluate the model trained on the training set.
\end{defi}
The evaluation (or \textit{validation}) of the model on the validation set leads to deciding which are the best values for the parameters (or hyper-parameters) based on the performance on the validation set. In the case of the elaboration set, an early-stopper is used, which chooses as the best model the one with the lowest error just before overfitting occurs.

\begin{defi}[Test set]
    A \textbf{test set} is an independent set of data from the training set used only to evaluate the performance of the model.
\end{defi}

\newpage

That is, the test set is used to evaluate on a third independent data set the chosen model based on the performance on the validation set. Performance characteristics such as accuracy, sensitivity, specificity... are thus obtained. This dataset is important because it allows the model's performance to be evaluated on a third dataset independent of the previous ones, avoiding the risk of overfitting: if a model trained on the test set also fits the test set, minimal overfitting has occurred.

The problem of \textbf{overfitting} has been mentioned several times, so the definition is given.

\begin{defi}[Overfitting]
    Overfitting is the generation of a model (or analysis) that corresponds too closely to a particular data set and may therefore fail to predict future observations reliably.
\end{defi}

Overfitting may occur, for example, by including too many adjustable parameters or by using an approach that is too complicated, as shown in figure \ref{overfittigComplex}. Clearly, when comparing different types of models, the complexity must take into account the influence of each parameter on the output \footnote{ Note that the opposite problem can be encountered by using an approach that is too simple: underfitting. For example, trying to approximate a sample with a parabolic trend with a linear regression}.

\begin{figure}[htbp]
    \centering
    \includegraphics[width=0.6\textwidth]{images/Machine learning/Overfitted_Data.png}
    \caption{Example of overfitting. Data (approximately linear) are approximated by a linear function and a polynomial function. Although the polynomial function provides an almost perfect fit, the linear function can be expected to generalise the data better. \cite{wiki:overfitting}}
    \label{overfittigComplex}
\end{figure}

\newpage

Overfitting is particularly likely in cases where learning has been performed for too long or where there is little data for learning, causing the model to fit very specific random features of the training data that have no causal relationship with the output. In this case of overfitting, the performance on the training set continues to increase while the performance on the validation set deteriorates, as shown in figure \ref{overfittingError}.



\begin{figure}[htbp]
    \centering
    \includegraphics[width=0.6\textwidth]{images/Machine learning/Overfitting error.png}
    \caption{Overfitting in supervised learning. The training error (error on training set) is shown in blue, the validation error (error on validation set) in red, both as a function of the number of training cycles. \cite{wiki:overfitting}}
    \label{overfittingError}
\end{figure}


\subsection{Loss function}
So far, the problem that interests the paper has been outlined: looking for a function $f$ to predict the output $Y$ from the values of the input $X$. This requires a loss function, $L(Y, f (X))$ to penalise prediction errors.

\begin{defi}[Loss function]
    A \textbf{loss function} is a function in the form $\mathcal{L}(y_{\text{true}},y_{ \text{guess}})$ that defines the loss incurred (or the error committed) by predicting the value $y_{\text{guess}}$ when the true value is $y_{\text{true}}$, thus providing an evaluation of the model's predictive capabilities.
\end{defi}

This will be the function to be minimised by an optimisation algorithm for fitting the hyperparameters of the Gaussian process. Obviously, the choice of loss function depends on the context.

%\section{Mean squared error}
%Per gli scopi dell'elaborato, si farà uso del \textit{mean squared error}.

%\begin{defi}[Mean squared error]
%Il \textbf{mean squared error} (o \textit{MSE}) è una \textit{loss function} definita come:
%\[
%\frac{1}{n}\sum^{n}_{i=1}(Y_i-f(X_i))^2,
%\]
%dove su $n$ dati raccolti, $Y_i$ sono gli output reali, $f(X_i)$ sono gli output predetti a partire da $X_i$.
%Poiché deriva dal quadrato della distanza euclidea, è sempre un valore positivo che diminuisce man mano che l'errore si avvicina a zero.
%\end{defi}
%Verrà usato non come loss function, ma per monitorare l'andamento del training.


%\section{Coefficiente di determinazione}
%Il coefficiente di determinazione fornisce una misura della bontà di adattamento di un modello, cioè quanto le previsioni di regressione approssimano i punti dei dati reali. Varia tra 0 e 1\footnote{Si possono apportare modifiche al coefficiente per fargli assumere anche altri valori, ma non vengono applicate nell'elaborato}, un $R^2$ di 1 indica che le previsioni di regressione si adattano perfettamente ai dati.

%Poichè il coefficiente di determinazione varia in un intervallo unitario, può essere più (intuitivamente) informativo di altre misure di errore (come il mean absolute error o il mean squared error) poiché tendono ad assumere valori su intervalli arbitrari oppure ad esprimere una percentuale.

%\begin{defi}[Coefficiente di determinazione]
%Date $y_i$ le osservazioni, $\overline{y}$ la media delle osservazioni, $\hat{y}_i$ i dati predetti dal modello:
%\[
%R^2 = 1-\frac{\sum_{i=1}^{n}(y_i-\hat{y}_i)}{\sum_{%i=1}^{n}(y_i-\overline{y})}
%\]
%\end{defi}
%Si noti che questo coefficiente non indica se:
%\begin{itemize}
%    \item una variabile sia statisticamente %significativa;
%    \item è stata usata la regressione corretta;
%    \item è stato scelto l'insieme più appropriato %di variabili indipendenti;
%    \item ci sono abbastanza dati per trarre una %conclusione solida.
%\end{itemize}

%Si noti inoltre che $R^2$ è una versione riscalata del mean squared error più facile da interpretare in quanto non dipende dalla scala dei dati.

\newpage
\section{Modelli parametrici e non parametrici}
Vi è una importante distinzione da fare sui modelli statistici: modelli parametrici e modelli non parametrici. Il tipo di modello, infatti, modifica notevolmente l'approccio teorico da usare e quindi i risultati applicativi. 


\subsection{Modelli parametrici}
I \textbf{modelli parametrici} presuppongono che la distribuzione dei dati possa essere modellata in termini di un insieme finito di parametri.\\
Un classico esempio è quello della regressione in cui, date delle osservazioni, si cerca di stimare un'eventuale relazione funzionale esistente tra la variabile dipendente e le variabili indipendenti. Se si assume un modello di regressione lineare, in cui la relazione è espressa dalla funzione $f(x) = \theta_1 + \theta_2x$ dove $x$ sono i dati di input, è necessario trovare i valori dei parametri $\theta_1$ e $\theta_2$ per definire $f$. In molti casi, l'assunzione del modello lineare non è sufficiente, ed è necessario un modello polinomiale (con più parametri): $f(x) = \theta_1+\theta_2x+\theta_3x^2$.\\
Quindi dato un dataset $D$ con $n$ punti osservati, dopo il processo di \textit{training} si assume che tutte le informazioni sulla relazione funzionale siano state catturate dai parametri $\theta_i$.  Quando si effettuano regressioni utilizzando modelli parametrici, la complessità e la flessibilità dei modelli è limitata dal numero di parametri. 

\subsection{Modelli non parametrici}
Contrariamente a quanto si possa pensare, un \textbf{modello non parametrico} non è un modello privo di parametri. \\
Intuitivamente, se il numero di parametri di un modello cresce con la dimensione del dataset di dati osservati, si tratta di un modello non parametrico. In via teorica, questo permette al modello di avere infiniti parametri e quindi di non dipendere da una struttura della funzione $f$ rigida, aumentandone la flessibilità.\\
Di interesse per l'elaborato è la regressione tramite processi gaussiani, che segue un approccio bayesiano non parametrico (visto nella sezione \ref{regressioneGP}). È in grado di \textit{apprendere} un'ampia varietà di  relazioni tra input e output utilizzando un numero teoricamente infinito di parametri e lasciando che i dati determinino il livello di complessità attraverso i mezzi di inferenza bayesiana.

Si veda \ref{gerarchica} per un chiarimento sulla differenza tra parametri e iperparametri nella regressione tramite processi gaussiani.

\newpage

\subsection{Struttura gerarchica}\label{gerarchica}
È comune usare una struttura gerarchica dei parametri e degli iperparametri nei modelli di regressione.

I parametri risiedono al livello più basso. Per esempio, nel caso della regressione lineare, i parametri sono i $\theta_i$, o nel caso di un modello di rete neurale i pesi associati ai neuroni.\\
Al secondo livello ci sono gli \textit{iperparametri} che controllano la distribuzione dei parametri al livello inferiore e dunque il cui valore è usato per controllare il processo di apprendimento.\\

Nel caso dei processi gaussiani, poiché sono un modello non parametrico, non è ovvio quali siano i parametri del modello. Si possono considerare come parametri i valori della funzione (priva di rumore) valutata sui dati osservati. Quindi sia $X=\{(x_i,y_i): i\in I\}$ l'insieme dei dati osservati, gli elementi $y_i$ possono essere considerati i parametri del modello. Evidentemente, più è grande $X$, più parametri ci sono. In termini pratici, si è visto in \ref{regressioneGP} come i dati osservati corrispondano alle informazioni sulla forma della funzione, quindi come la precisione dipenda dal numero di dati osservati.

È anche possibile dare una diversa lettura dei parametri del modello usando un'interpretazione teorica diversa (\textit{weight-space view}), che non è stata introdotta nell'elaborato.\footnote{Per approfondire: \cite{rasmussen_gaussian_2006}}\\

Gli iperparametri, invece, sono i parametri della mean function e della kernel function. Pragmaticamente, per fare regressione è necessario lavorare sulla stima degli iperparametri.


\newpage


\section{Ottimizzazione degli iperparametri}

\subsection{Inferenza bayesiana}
L'\textbf{inferenza bayesiana} è un metodo di inferenza statistica in cui il teorema di Bayes viene utilizzato per aggiornare la probabilità di un'ipotesi man mano che si rendono disponibili ulteriori  informazioni. 

Il processo bayesiano di analisi dei dati può essere idealizzato suddividendolo nelle seguenti tre fasi:
\begin{enumerate}
    \item Creazione di un modello di probabilità completo\\
    Il modello consiste in una distribuzione di probabilità congiunta per tutte le quantità relative al problema e deve essere coerente con le conoscenze del problema scientifico di base.
    \item Condizionamento dei dati osservati\\ Viene calcolata e interpretata l'appropriata distribuzione a posteriori, ovvero la distribuzione di probabilità delle quantità non osservate condizionata rispetto i dati osservati.
    \item Valutazione dell'adattamento del modello\\
    Viene valutato quanto il modello si adatta ai dati e se le conclusioni sostanziali sono ragionevoli. In questa fase è possibile modificare o espandere il modello e ripetere le tre fasi.
\end{enumerate}
L'approccio bayesiano può essere interpretato come una formalizzazione del metodo scientifico, in cui si cerca di aggiornare la conoscenza scientifica attraverso esperimenti e osservazioni.\\
Inoltre, questo approccio statistico gode di una flessibilità tale da essere utilizzabile in problemi complessi, anche con molti parametri.\\

Nell'ambito dei processi gaussiani viene studiato l'uso dell'inferenza bayesiana nell'ottimizzazione degli iperparametri.

\newpage

\subsection{Teorema di Bayes}
\begin{teo}[Teorema di Bayes]
Siano $A, B$ eventi, sia $\probP(B)\neq 0$, allora:
\[
\probP(A|B) = \frac{\probP(B|A)\probP(A)}{\probP(B)}.
\]
\end{teo}
\begin{proof}
Dalla definizione di probabilità condizionata si ha: $\probP(A\cap B) = \probP(A|B)\probP(B)$, da cui segue $\probP(A|B)=\frac{\probP(A\cap B)}{\probP(B)}$ se $\probP(B)\neq 0$; analogamente: $\probP(B|A)=\frac{\probP(A\cap B)}{\probP(A)}$ se $\probP(A)\neq 0$. Sostituendo $\probP(A\cap B)$ nell'espressione di $\probP(A|B)$ si conclude.
\end{proof}

\vspace{0.5cm}

Nel contesto bayesiano lo scopo dell'analisi di un modello è quello di trarre conclusioni statistiche su un parametro $\theta$ (o analogamente su un vettore di parametri\footnote{Per il momento viene preso in considerazione il caso dei parametri, ma quanto detto si estende immediatamente al caso degli iperparametri, come visto in \ref{evidenzaBayesiana}}) o su dati non osservati; le suddette conclusioni vengono formulate in termini di probabilità condizionate al valore osservato di $y$. 
Per poter fare affermazioni di probabilità su $\theta$ osservati i dati $y$ è necessario iniziare con un modello che fornisca una distribuzione di probabilità congiunta per $\theta$ e $y$, fattorizzabile in due componenti: $\probP(\theta)$, cioè la distribuzione a priori che descrive la conoscenza sulla distribuzione di $\theta$ prima di osservare i dati, e $\probP(y|\theta)$, cioè la distribuzione dei dati (o campionaria):
\[
\probP(\theta, y)=\probP(\theta)\probP(y|\theta).
\]

Applicando il teorema di Bayes si ottiene la \textit{distribuzione a posteriori}:
\[
\probP(\theta | y)=\frac{\probP(\theta)\probP(y|\theta)}{\probP(y)},
\]
nella quale il denominatore può essere riscritto, per il teorema della probabilità assoluta (in inglese \textit{law of total probability}), come:
\[
\probP(y)=\begin{cases}\sum_\theta \probP(\theta)\probP(y| \theta)\\\int\probP(\theta)\probP(y| \theta)d\theta \end{cases} \text{ in base a }\theta.
\]
Poiché $\probP(y)$ non dipende da $\theta$ e fissando $y$ è possibile riscrivere la distribuzione a posteriori come:

\[
\probP(\theta | y)\propto \probP(\theta)\probP(y|\theta),
\]
dove, avendo fissato $y$, $\probP(y|\theta)$ è funzione di $\theta$.

\newpage

\subsection{Evidenza bayesiana}\label{evidenzaBayesiana}
Nel contesto dei modelli statistici, la reinterpretazione del teorema di Bayes (e di quanto detto nella sezione precedente) è:
\[
\probP(\theta | X, \alpha) = \frac{\probP(X|\theta, \alpha)\probP(\theta | \alpha)}{\probP(X|\alpha)}\propto \probP(\theta | \alpha) \probP(X|\theta, \alpha),
\]
dove: $\theta$ è il vettore di parametri, $\alpha$ il vettore degli iperparametri, $X$ il campione di dati osservati. In questo caso la distribuzione a priori è composta da $\probP(\theta|\alpha)$\footnote{Si noti che è condizionata rispetto al valore degli iperparametri. Questo è dovuto alla definizione degli iperparametri, che controllano la distribuzione dei parametri, come detto nella sezione \ref{gerarchica}.}, la distribuzione dei parametri prima di osservare i dati $X$; $\probP(X|\theta)$, la distribuzione campionaria che è la distribuzione dei dati osservati condizionata al valore dei parametri; e $\probP(X|\alpha)$, l'\textbf{evidenza} (anche chiamata \textit{verosimiglianza marginale}, in inglese \textit{marginal likelihood}) che è la distribuzione dei dati condizionata rispetto gli iperparametri. Quest'ultima è possibile riscriverla come
\[
\probP(X|\alpha) = \int \probP(X|\theta)\probP(\theta | \alpha)d\theta,
\]
cioè come la distribuzione dei dati marginalizzata sui parametri.

Oltre ad essere il termine di normalizzazione nel teorema di Bayes, un possibile altro modo di interpretare la marginal likelihood è quello di considerarla la probabilità di aver generato il dataset (osservato) $X$ dati gli iperparametri $\alpha$.

Utilizzando ancora il teorema di Bayes si ottiene:
\[
\probP(\alpha |X) = \frac{\probP(X|\alpha)\probP(\alpha)}{\probP(X)}\propto \probP(X|\alpha) \probP(\alpha),
\]
cioè la distribuzione a posteriori (che rappresenta la distribuzione degli iperparametri osservati i dati $X$) è proporzionale alla marginal likelihood. Poiché nel caso preso in considerazione dall'elaborato, cioè i processi gaussiani, uno degli obiettivi è trovare la distribuzione degli iperparametri (cioè la forma della covariance function, in altri termini $\probP(\alpha|X)$), l'approccio adottato è quello di massimizzare la marginal likelihood, sfruttando la dipendenza evidenziata nella formula precedente. Questo approccio è chiamato \textbf{evidence approach}.

\newpage

\subsection{Alternative all'evidenza bayesiana}
La scelta dell'approccio per stimare il valore di parametri o iperparametri dipende molto dalla situazione. Ad esempio:
\begin{itemize}
    \item Massimizzazione della \textit{likelihood}:
    si massimizza $L(\alpha, \theta | X)$ (la funzione di verosimiglianza), dunque si trovano valori di $\alpha$ e  $\theta$ che la massimizzano;
    \item Massimizzazione parziale della \textit{likelihood}:
    dato il caso in cui $L(\alpha, \theta |X)=L_1(\alpha | X)L_2(\theta|X)$, si massimizza solo il fattore di interesse, nel caso in esame si massimizza $L_1(\alpha | X)$;
    \item Massimizzazione della \textit{marginal likelihood}:
    si marginalizza su $\theta$, quindi si massimizza $\probP(X|\alpha)$ che dipende solo da $\alpha$.
\end{itemize}
L'elaborato applica l'ultimo caso poiché i calcoli necessari sono tutti analiticamente risolubili. In altre situazioni questo approccio richiede un'approssimazione numerica che può portare a problemi di stabilità, motivo per cui non è sempre un approccio usato.




\subsection{Nei processi gaussiani}\label{neiProcessiGaussiani}
L'approccio che i processi gaussiani seguono è quello di definire una \textit{distribuzione sulle funzioni} (in inglese \textit{distribution over functions}), aggiornarla a partire dai dati osservati e usarla per fare previsioni su nuovi input. \\
Si consideri quanto fatto nella sezione \ref{noisyPrediction}: si è definita una distribuzione a priori sulle funzioni che poi è stata convertita in una distribuzione a posteriori aggiornando le informazioni grazie ai dati osservati. A differenza del caso \textit{noise-free}, però, non si ottiene una funzione che interpoli perfettamente i dati. In questo caso è dunque necessario scegliere opportuni iperparametri, in modo da ottimizzare la forma della funzione: questa stima viene fatta tramite evidence approach sfruttando il fatto che le espressioni degli integrali sono tutte analiticamente risolubili.\\


In questo caso la marginal likelihood è:
\[
\probP(\bf{y}|\bf{\alpha}) = \int \probP(\bf{y}| \mathbf{f},\bf{\alpha})\probP(\mathbf{f}|\bf{\alpha}) d\mathbf{f},
\]

dove $\bf{\alpha}$ sono gli iperparametri, $\mathbf{f}$ sono dati osservati senza errore (sono i parametri), $\bf{y}$ sono i dati osservati.\\
Si noti che $\probP(\mathbf{f}|\bf{\alpha})$ è la distribuzione a priori della funzione priva di errori ed è una distribuzione gaussiana multivariata in quanto $\mathbf{f}|\alpha\sim \mathcal{N}(m, K)$ con $K$ la matrice di Gram della covariance function e $m$ la valutazione della mean function sul vettore di input. Inoltre, dalla definizione di $\mathbf{y}$ si ha $\mathbf{y}|\mathbf{f}\sim \mathcal{N}(\mathbf{f}, \sigma_n^2I)$, dove $\epsilon\sim \mathcal{N}(0,\sigma_n^2)$.

\newpage

Con i calcoli riportati su \cite{rasmussen_gaussian_2006} si conclude:
\[
L=\text{log}(\probP(\mathbf{y}|\alpha))=-\frac{1}{2}\text{log}(|K+\sigma_n^2I|)-\frac{1}{2}(\mathbf{y}-m)^{\text{T}}(K+\sigma_n^2I)^{-1}(\mathbf{y}-m)-\frac{n}{2}\text{log}(2\pi)
\]

Lo stesso risultato si può ottenere evitando i conti analitici notando: $\mathbf{y}\sim \mathcal{N}(m, K+\sigma_n^2I)$. A questo punto è semplice calcolare le derivate parziali rispetto agli iperparametri della covariance function e della mean function. Questo è di fondamentale importanza per i metodi numerici di ottimizzazione.

Si noti che la massimizzazione della marginal likelihood è preferibile rispetto alla massimizzazione della likelihood (o altri metodi di likelihood) poiché con altre forme di likelihood è possibile ottenere un overfitting (si veda \ref{dataset} per la definizione di overfitting). Al contrario, la marginal likelihood non esegue un \textit{fitting} dei valori delle funzioni, ma li integra (li marginalizza), cioè tecnicamente non può fare overfitting perché non avviene alcun \textit{fit}. \\

Per ulteriori considerazioni, dettagli sui conti e approfondimenti su come ottimizzare i calcoli numerici si rimanda a \cite{rasmussen_gaussian_2006}.




\section{Metodo di ottimizzazione}
L'\textbf{ottimizzazione} è la branca della matematica che studia teoria e metodi per la ricerca dei punti di massimo e minimo di una funzione. Questa sezione si ripropone di ripercorrere i principali passi teorici che hanno portato alla costruzione del metodo di ottimizzazione usato successivamente nella massimizzazione della marginal likelihood: Adam.

\subsection{Forma della funzione da ottimizzare}\label{costfunction}
Per comprendere la differenza tra i metodi di ottimizzazione che vengono riportati è importante comprendere la forma della funzione da ottimizzare.
Nell'ambito dell'apprendimento supervisionato si ha un dataset di osservazioni da cui si vuole imparare. Per farlo, in breve, si definisce una funzione che rappresenta l'errore che il nostro modello compie sul dataset di valori osservati\footnote{Si veda la sezione \ref{dataset}, in cui viene spiegato come dividere un dataset in sottoinsiemi con diversi ruoli.}: l'obiettivo è minimizzare questa funzione e per farlo è necessario, usando un opportuno algoritmo, cambiare i valori dei parametri (nel caso in esame, un modello non parametrico, vengono modificati gli iperparametri).

\newpage

Siccome il dataset di dati osservati conterrà potenzialmente molti dati, è necessario tenere conto dell'errore commesso valutandolo in ogni dato osservato. Generalmente, infatti, la funzione di errore ha forma:
\[
Q(w)=\sum_{i=1}^{n}Q_i(w),
\]
dove tendenzialmente ogni addendo dipende da un singolo dato del dataset di input. Un esempio è:
\[
f(m,b)=\sum_{i=1}^{N}(y_i-(mx_i+b))^2,
\]
cioè il metodo dei minimi quadrati nella regressione lineare.\\
Si noti che nel caso in esame la funzione da ottimizzare, la marginal likelihood, è una somma in cui l'addendo $i$-esimo non dipende solo dall'$i$-esimo dato osservato.


\subsection{Learning rate}
I metodi iterativi di ottimizzazione hanno una forma simile tra loro: dato $\theta_0$ punto iniziale, ad ogni iterazione si aggiorna il punto come:
\[
\theta_{t+1}=\theta_t+\eta_td_t,
\]
dove $d_t$ rappresenta una direzione ed è definita in base al metodo, $\eta_t$ è la lunghezza del passo (in inglese \textit{step size}) o \textit{learning rate} e indica di quanto ci si sposta nella direzione $d_t$. Il learning rate può essere costante oppure può essere aggiornato ogni iterazione.


\newpage

\subsection{Metodo del gradiente \cite{ruder_2022}}
Il \textbf{metodo del gradiente} sfrutta il fatto che il gradiente è la direzione di massima crescita della funzione. Ha forma:
\begin{align*}
    &\text{Dato un valore iniziale }\omega_0, \eta_0\\
    &\text{Da ripetere fino a che si trova un'approssimazione del minimo:}\\
    &\quad \omega_{k+1}=\omega_k - \eta_k\cdot \nabla Q(\omega_k).
\end{align*}
Con "\textit{Da ripetere fino a che si trova un'approssimazione del minimo}" si intende che si ripete il metodo fino a che non viene soddisfatta una condizione di uscita che incorpora la precisione che si chiede all'approssimazione.\\
Poiché, come visto in \ref{costfunction}, è necessario valutare il gradiente su ogni elemento del dataset, può essere un metodo oneroso soprattutto in termini di memoria.\\
Il learning rate $\eta_k$ può essere costante per ogni $k$ oppure può essere modificato ad ogni iterazione. In entrambi i casi svolge un ruolo importante poiché può portare, se troppo piccolo, ad una convergenza troppo lenta oppure, se troppo grande, ad una divergenza. Nel caso di un learning rate aggiornato ad ogni iterazione, generalmente lo si diminuisce mano a mano che ci si avvicina al minimo; ad esempio si veda la figura \ref{gradientDescentImage}.

\begin{figure}[h]
    \centering
    \includegraphics[width=0.6\textwidth]{images/Training (teoria)/Gradient descend.png}
    \caption{Illustrazione del metodo del gradiente su degli insiemi di livello, ad ogni iterazione viene aggiornato il learning rate.\cite{wiki:gradientDescend}}
    \label{gradientDescentImage}
\end{figure}



\subsection{Discesa stocastica del gradiente \cite{bottou2012stochastic}}
Il metodo di \textbf{discesa stocastica del gradiente}, diversamente dal metodo del gradiente, esegue un aggiornamento per ogni $Q_i$\footnote{Nel caso in cui la $Q$ si scriva come somma di termini che dipendono ognuno solo da un elemento del dataset, comporta che il metodo esegue un aggiornamento per ogni elemento del dataset.}. Ha forma:
\begin{align*}
    &\text{Dato un valore iniziale }\omega_0, \eta_0\\
    &\text{Da ripetere fino a che si trova un'approssimazione del minimo:}\\
    &\quad\text{Mescola il dataset}\\
    &\quad\text{Per }i=1,...,n:\\
    &\quad\quad\omega_{k+1}=\omega_k -\eta_k\cdot \nabla Q_i(\omega_k)
\end{align*}
Dovendo calcolare il gradiente per un solo addendo di $Q$ alla volta, è un metodo molto più veloce, seppur questi frequenti aggiornamenti causino un'alta fluttuazione della funzione obiettivo, come si può notare dalla figura \ref{MomentumMethod}.\\
Anche se la fluttuazione può sembrare uno svantaggio del metodo, è proprio questa che gli permette di evitare minimi locali e tendere a migliori approssimazioni. Il metodo del gradiente visto precedentemente invece converge al minimo del bacino in cui è stato scelto il dato iniziale.\\
D'altra parte, la fluttuazione complica la convergenza verso il minimo esatto a causa dei frequenti cambiamenti che la funzione subisce. Tuttavia è stato dimostrato che quando si diminuisce lentamente il learning rate il metodo mostra lo stesso comportamento di convergenza del metodo del gradiente. 


\newpage
\subsection{Metodo dei momenti \cite{ruder_2022}}
Il metodo stocastico del gradiente ha difficoltà a percorrere i \textit{burroni}, cioè le aree in cui la superficie curva più ripidamente in una dimensione rispetto a un'altra. In questi scenari, il metodo oscilla lungo le pendici del burrone, mentre progredisce con esitazione lungo il fondo verso il minimo, come mostrato nella figura \ref{MomentumMethod}.\\
Il \textbf{metodo dei momenti} è un metodo che aiuta ad accelerare il metodo stocastico del gradiente nella direzione desiderata e a smorzare le oscillazioni, come si vede nella figura \ref{MomentumMethod}. Ha forma:

\begin{align*}
    &\text{Dato un valore iniziale }\omega_0, \eta_0, v_0=0\\
    &\text{Da ripetere fino a che si trova un'approssimazione del minimo:}\\
    &\quad\text{Mescola il dataset}\\
    &\quad\text{Per }i=1,...,n:\\
    & \quad\quad v_t=\gamma v_{t-1}+\eta_k\cdot \nabla Q_i(\omega_k)\\
    &\quad\quad\omega_{k+1}=\omega_k - v_t
\end{align*}
Il termine di $\gamma$ è chiamato \textit{momento} (o \textit{quantità di moto}) ed è solitamente impostato a 0.9.\\
In sostanza quello che avviene è analogo a quando si spinge una palla giù per una collina: la palla accumula quantità di moto
mentre rotola in discesa  diventando sempre più veloce (finché non raggiunge la sua velocità terminale, se c'è
la resistenza dell'aria, cioè $\gamma<1$). La stessa cosa accade agli aggiornamenti dei parametri: nel vettore il termine di "quantità di moto" favorisce le dimensioni in cui il gradiente punta nella stessa direzione e sfavorisce le altre. Di conseguenza, si ottiene una convergenza più rapida e una riduzione dell'oscillazione, riducendo i problemi del metodo stocastico del gradiente, come si mostra in figura \ref{MomentumMethod}.

\begin{figure}[h]
\centering
\begin{subfigure}{.5\textwidth}
  \centering
  \includegraphics[width=\linewidth]{images/Training (teoria)/SGD Momentum a.PNG}
  \caption{Senza momento.}
\end{subfigure}%
\begin{subfigure}{.5\textwidth}
  \centering
  \includegraphics[width=\linewidth]{images/Training (teoria)/SGD Momentum b.PNG}
  \caption{Con momento.}
\end{subfigure}
\caption{Metodo dei momenti applicato al metodo stocastico del gradiente. \cite{ruder_2022}}
\label{MomentumMethod}
\end{figure}

\newpage
\subsection{AdaGrad \cite{JMLR:v12:duchi11a}}
Il metodo \textbf{AdaGrad} (abbreviazione di \textit{Adaptive Gradient algorithm})  è un algoritmo di ottimizzazione basato sul metodo stocastico del gradiente che usa un learning rate indipendente per ogni $\omega_i$. Per i termini $\omega_i$ che hanno gradiente più elevato o aggiornamenti frequenti, il metodo impone un learning rate più basso, in modo da non superare il minimo. Viceversa quelli con gradiente basso o aggiornamenti poco frequenti avranno un learning rate più alto, in modo da essere addestrati rapidamente.\\



Adagrad migliora la robustezza del metodo stocastico del gradiente e converge più rapidamente soprattutto quando la distribuzione dei dati è sparsa; ha dimostrato ottimi risultati nell'addestramento di reti neurali su larga scala (ad esempio in Google).

\begin{align*}
    &\text{Dato un valore iniziale }\omega_0, \eta\\
    &\text{Da ripetere fino a che si trova un'approssimazione del minimo:}\\
    &\quad\text{Mescola il dataset}\\
    &\quad\text{Per }i=1,...,n:\\
    &\quad\quad \text{Per ogni componente $j$ di $\omega$:}\\
    &\quad\quad\quad g_{k,j}=(\nabla Q_i(\omega_k))_j\\
    &\quad\quad\quad\omega_{k+1,j}=\omega_{k,j} - \frac{\eta}{\sqrt{G_{k,j}}+\epsilon}g_{k,j}
\end{align*}
Dove $G_{k,j}=\sum_{t=1}^{k}g_{t,j}^2$ e $\epsilon$ un termine piccolo per evitare divisioni per zero (generalmente $10^{-8}$).\\
Poiché adatta il learning rate, uno dei principali vantaggi del metodo è che elimina la necessità di regolare manualmente $\eta_k$: generalmente si imposta $\eta=0,01$.\\
Il punto debole di Adagrad è l'accumulo dei gradienti al quadrato nel denominatore: poiché ogni termine aggiunto è positivo, la somma accumulata continua a crescere durante l'addestramento. Questo fa sì che il learning rate si riduca e tenda a diventare infinitesimamente piccolo, a quel punto l'algoritmo non è più in grado di acquisire ulteriori conoscenze.


\newpage
\subsection{Adam \cite{kingma_adam_2017}}\label{adam}
Il metodo \textbf{Adam} (abbreviazione di \textit{Adaptive Moment Estimation}) è, come il precedente, un metodo a learning rate adattivi per ogni parametro. Deriva da altri metodi (AdaDelta, RMSprop) che cercano di ridurre il tasso di apprendimento monotonicamente decrescente del metodo AdaGrad. Infatti: invece di accumulare tutti i gradienti passati al quadrato, la somma dei gradienti è ricorsivamente definita come una media decrescente di tutti i gradienti quadratici passati, che dipende quindi dalla media precedente e dal gradiente corrente, ed è chiamata $v_k$. Inoltre, il metodo di Adam prevede la stessa media decrescente dei gradienti passati, chiamata $m_k$.


\begin{align*}
    &\text{Dato un valore iniziale }\omega_0, \eta, m_0=0, v_0=0\\
    &\text{Da ripetere fino a che si trova un'approssimazione del minimo:}\\
    &\quad\text{Mescola randomicamente il dataset}\\
    &\quad\text{Per }i=1,...,n:\\
    &\quad\quad \text{Per ogni componente $j$ di $\omega$:}\\
    &\quad\quad\quad g_{k,j}=(\nabla Q_i(\omega_k))_j\\
    &\quad\quad\quad m_k=\beta_1m_{k-1}+(1-\beta_1)g_{k,j}\\
    &\quad\quad\quad v_k=\beta_2v_{k-1}+(1-\beta_2)g^2_{k,j}\\
    &\quad\quad\quad \hat{m}_k = \frac{m_k}{1-\beta_1^k}\\
    &\quad\quad\quad \hat{v}_k = \frac{v_k}{1-\beta_2^k}\\
    &\quad\quad\quad\omega_{k+1,j}=\omega_{k,j} - \frac{\eta}{\sqrt{\hat{v_k}}+\epsilon}\hat{m_k}
\end{align*}
I termini $m_k$ e $v_k$, le medie decrescenti dei gradienti e dei gradienti quadrati, sono stime del primo momento e del secondo momento dei gradienti, come definito nel metodo dei momenti. Poiché  $m_k$ e $v_k$ sono distorti verso valori prossimi allo zero, si usano stimatori corretti $\hat{m}_k$ e $\hat{v}_k$. Generalmente si impongono valori $\beta_1=0.9$ e $\beta_2=0.999$.

\newpage
In figura \ref{comparison} vengono comparati alcuni metodi nella minimizzazione della cost function in una neural network. Si nota che il metodo di Adam è molto più efficiente degli altri metodi introdotti nell'elaborato.
\begin{figure}[h]
    \centering
    \includegraphics[width=0.75\textwidth]{images/Machine learning/Comparison.PNG}
    \caption{Comparazione di metodi di minimizzazione di una cost function in una rete neurale. \cite{kingma_adam_2017}}
    \label{comparison}
\end{figure}
\chapter{Windkessel model}\label{windkessel}
This chapter introduces some concepts of physiology necessary for understanding the \textbf{Windkessel model}, later introduced. The chapter is not intended to be complete with the theory behind the model, but it is intended to introduce the idea on which it is based: the \textbf{Windkessel effect}.\\
The chapter makes use of a jupyter notebook used in the course \textit{Introduction to Python for Scientific Computing} taught by Professor Lucas Omar Müller in the academic year 2021/2022.\\
The sources used in writing the chapter are: \cite{AaronsonPhilipI.PhilipIrving2020Tcsa}, \cite{wiki:Vascularresistance}, \cite{wiki:Compliance}, \cite{wiki:WindkesselEffect},
\cite{wiki:cicloCardiaco},
\cite{wiki:DiagrammaWiggers},
\cite{westerhof_arterial_2008}, \cite{ghitti_toro_müller_2022}. \\



\begin{textblock*}{0.64\textwidth}(3.5cm+0.36\textwidth,18.5cm)
\epigraph{Medicine is a science of uncertainty and an art of probability.}{William Osler}
\end{textblock*}




\newpage

\section{Introduction}
From \cite{westerhof_arterial_2008} we consider data from a patient in whom aortic flow $Q_{text{in}}$ and systemic pressure $P$ were measured.

Viewing the graph of the two values yields what is shown in figure \ref{figDatiReali}.

\begin{figure}[h]
    \centering
    \includegraphics[width=0.7\textwidth]{images/Windkessel/DatiReali.pdf}
    \caption{Systemic pressure and aortic flow measured in a patient. Code \ref{datiReali}.}
    \label{figDatiReali}
\end{figure}

A simple model of the cardiovascular system is now considered:
\[
P=Q_{\text{in}}R,
\]
Where $R$ is the total peripheral resistance. This model allows calculation of arterial pressure from the blood flow entering the system and the total peripheral resistance.



From the patient's actual data, it is possible to derive the value of total peripheral resistance $R$, or resistance of the cardiovascular system. Given $T$ the duration of the cardiac cycle, we have:
\[
R = \frac{\frac{1}{T}\int_TP(t)dt}{\frac{1}{T}\int_TQ_{\text{in}}(t)dt}.
\]

\newpage

Having derived $R=1.037 mmHg/mL\cdot s$ (as per the code \ref{resistenzatotale}) knowing $Q_{\text{in}}$, it is now possible to use the model to derive the trend of $P$, as shown in Figure \ref{figModelloSemplic}.


\begin{figure}[h]
    \centering
    \includegraphics[width=0.7\textwidth]{images/Windkessel/modelloSemplice.pdf}
    \caption{Real pressure graph and simple model output. Code \ref{modelloSemplice}.}
    \label{figModelloSemplic}
\end{figure}

As is clear from the figure, in the systole phase the model predicts a very high peak pressure (on the order of five times higher than the actual figure). As will be shown in the rest of the chapter, this is because the model neglects a fundamental aspect of arterial circulation: the Windkessel effect, that is, the storage of energy in the arteries during the systole phase.\\
A model that takes this effect into account and thus provides a much more accurate approximation will be proposed at the end of the chapter.


\newpage




\section{Concepts of physiology}

\subsection{Cardiac cycle}
The cardiac cycle, of which a table summary is given \ref{cicloCardiaco}, will not be analyzed in detail, but the period of systole and diastole are briefly described, useful for understanding the next topics. The image \ref{wiki: cuore} may come in handy, so as to have in mind the anatomy of the human heart necessary for understanding the subsequent concepts.

\begin{figure}[h]
    \centering
    \includegraphics[width=0.7\textwidth]{images/Windkessel/Cuore.png}
    \caption{Anatomy of the human heart \cite{wiki:cicloCardiaco}.}
    \label{wiki: cuore}
\end{figure}

Cardiac diastole is the period in the cardiac cycle when, after contraction, the heart relaxes and expands as it fills with returning blood from the circulatory system. Both atrioventricular valves (tricuspid and mitral) open to facilitate the "unpressurized" flow of blood directly through the atria into both ventricles, where it is collected for the next contraction. \\

Atrial systole is the contraction of cardiac muscle cells in both atria as a result of electrical stimulation and conduction of electrical currents through the atrial chambers. \\
Defined as part of the contraction and ejection sequence, atrial systole actually plays the role of completing diastole, that is, finalizing the filling of both ventricles with blood while they are relaxed and expanded for that purpose. 

\newpage

Atrial systole overlaps with the end of diastole and applies contraction pressure to shift blood volumes to both ventricles; this atrial "kick" concludes diastole immediately before the heart begins contracting again and expelling blood from the ventricles (ventricular systole) to the aorta and arteries.\\

Ventricular systole is the contraction, following electrical stimulation, of the ventricular syncytium of heart muscle cells in the right and left ventricles. \\
Right ventricular contractions provide pulmonary circulation by pulsing oxygen-depleted blood through the pulmonary valve and then through the pulmonary arteries to the lungs. Simultaneously, left ventricular systole contractions provide systemic circulation of oxygen-depleted blood to all body systems by pumping blood through the aortic valve, aorta and all arteries.\\
Two simple diagrams depicting the displacement of oxygenated (red) and non-oxygenated (blue) blood in the systole and diastole phases are shown in Figure \ref{sistoleDiastole}.

\begin{figure}[h]
\centering
\begin{subfigure}{0.5\textwidth}
  \centering
  \includegraphics[width=0.5\linewidth]{images/Windkessel/sistole.png}
  \caption{Systole}
\end{subfigure}%
\begin{subfigure}{0.5\textwidth}
  \centering
  \includegraphics[width=0.5\linewidth]{images/Windkessel/diastole.png}
  \caption{Diastole}
\end{subfigure}
\caption{Diagrams summarizing the systole and diastole of a human heart \cite{wiki:cicloCardiaco}.}
\label{sistoleDiastole}
\end{figure}

\newpage

%%%%%%%%%%%%%%%%%%% TABELLA
\begin{landscape}
%%%%%%%%%%%%%%%%%%%%%%%%%%%%%%%%%%%%%%%%%%%%%%%
%%%%%%%%%%%%%%%%% TABELLA
%%%%%%%%%%%%%%%%%%%%%%%%%%%%%%%%%%%%%%%%%%%%%
% Please add the following required packages to your document preamble:
% \usepackage{graphicx}
\begin{table}
\centering
\resizebox{1.6\textwidth}{!}{%
\begin{tabular}{|l|c|c|l|}
\hline
\multicolumn{1}{|c|}{\textbf{Fase}} &
  \textbf{\begin{tabular}[c]{@{}c@{}}Valvole atrioventricolari\\ (tricuspide e mitrale)\end{tabular}} &
  \textbf{\begin{tabular}[c]{@{}c@{}}Valvole semilunari \\ (polmonare e aortica)\end{tabular}} &
  \multicolumn{1}{c|}{\textbf{Stato dei ventricoli e degli atri; flusso sanguigno}} \\ \hline
1 - Rilassamento isovolumetrico &
  chiuse &
  chiuse &
  \begin{tabular}[c]{@{}l@{}}Le valvole semilunari si chiudono alla fine della fase di eiezione; \\ il flusso di sangue si ferma.\end{tabular} \\ \hline
\begin{tabular}[c]{@{}l@{}}2a - Afflusso (riempimento \\ ventricolare)\end{tabular} &
  aperte &
  chiuse &
  \begin{tabular}[c]{@{}l@{}}I ventricoli e gli atri si rilassano e si espandono insieme; \\ il sangue scorre nel cuore durante la diastole ventricolare e atriale.\end{tabular} \\ \hline
\begin{tabular}[c]{@{}l@{}}2b - Influsso (Riempimento \\ ventricolare con sistole atriale)\end{tabular} &
  aperte &
  chiuse &
  \begin{tabular}[c]{@{}l@{}}Ventricoli rilassati ed espansi; la contrazione atriale (sistole) spinge \\ il sangue sotto pressione nei ventricoli durante la diastole ventricolare.\end{tabular} \\ \hline
3 - Contrazione isovolumetrica &
  chiuse &
  chiuse &
  \begin{tabular}[c]{@{}l@{}}Le valvole AV si chiudono alla fine della diastole ventricolare; \\ il flusso di sangue si ferma; i ventricoli cominciano a contrarsi.\end{tabular} \\ \hline
\begin{tabular}[c]{@{}l@{}}4 - Espulsione ventricolare\end{tabular} &
  chiuse &
  aperte &
  \begin{tabular}[c]{@{}l@{}}I ventricoli si contraggono (sistole ventricolare); il sangue scorre \\ dal cuore ai polmoni e al resto del corpo durante l'espulsione ventricolare.\end{tabular} \\ \hline
\end{tabular}%
}
\caption{Tabella riassuntiva ciclo cardiaco \cite{wiki:cicloCardiaco}.}
\label{cicloCardiaco}
\end{table}
%%%%%%%%%%%%%%%%%%%%%%%%%%%%%%%%%%%%%%%%%%%%%
%%%%%%%%%%%%%%% FINE TABELLA
%%%%%%%%%%%%%%%%%%%%%%%%%%%%%%%%%%%%%%%%%%
\end{landscape}

\newpage

Useful in understanding the cardiac cycle is the Wiggers diagram: this is a standard diagram that is used in teaching cardiac physiology in which the horizontal axis is used to plot time while the vertical axis contains all of the following on a single grid: blood pressure, aortic pressure, ventricular pressure, atrial pressure, ventricular volume, electrocardiogram, and arterial flow.\\
The Wiggers diagram clearly illustrates the coordinated variation of these values as the heart beats, helping to understand the entire cardiac cycle. It also helps to become familiar with the shape of the parameter curves that will be used in the paper. An example of a diagram is shown in figure \ref{WiggersDiagramma}.


\begin{figure}[h]
    \centering
    \includegraphics[width=1\textwidth]{images/Windkessel/Wiggers_Diagram_IT.svg.png}
    \caption{Example of Wiggers diagram \cite{wiki:DiagrammaWiggers}.}
    \label{WiggersDiagramma}
\end{figure}



\newpage

\subsection{Useful terminology}\label{terminologia}

%%%%%%%%%%%%  VOLUME SISTOLIC
\subsubsection{Systolic volume}
The \textbf{systolic volume} (English \textit{stroke volume}, or SV) is the amount of blood pumped from a ventricle at each ventricular systole. The systolic volume is usually equal in the two ventricles, about $70ml$ in a healthy $70kg$ man.


%%%%%%%%%%%%%%%% ELASTIC ARTERY
\subsubsection{Elastic artery}
An \textbf{elastic artery} is an artery formed by many filaments of collagen and elastin, which give it the ability to stretch in response to any pulsation. 
\\
It is by virtue of this elasticity that the Windkessel effect occurs, which helps maintain a constant pressure in the arteries despite the pulsating nature of blood flow. 
\\
Elastic arteries include the largest arteries in the body, those closest to the heart.
\\
The pulmonary arteries, the aorta and its branches together constitute the body's elastic artery system. Examples are: aorta, brachiocephalic, common carotid, subclavian, common iliac.


%%%%%%%%%%%%%%% RESISTENZA VASCOLARE
\subsubsection{Vascular resistance}
The \textbf{vascular resistance} is the resistance that must be overcome to push blood through the circulatory system and create flow. 
\\
The resistance offered by the systemic circulation is known as \textbf{systemic vascular resistance} or \textbf{total peripheral resistance} \footnote{There is the resistance offered by the pulmonary circulation known as \textbf{pulmonary vascular resistance} or PVR. However, it will not be studied in this paper}.
\\La vasocostrizione (cioè la diminuzione del diametro dei vasi sanguigni) aumenta la SVR, mentre la vasodilatazione (aumento del diametro) diminuisce la SVR.
\\
The formula is valid: $R=\frac{\Delta P}{Q}$, where $R$ is the resistance, $\Delta P$ the pressure change during the circulatory loop\footnote{That is, from just after exit from the left ventricle/right ventricle to entry into the right atrium/left atrium}, $Q$ is the flow.\\
Note that this is the hydraulic version of Ohm's law, V=IR, in which pressure differential is analogous to electrical voltage drop, flow is analogous to electric current, and vascular resistance is analogous to electrical resistance.

\newpage
%%%%%%%%%%%%% COMPLIANCE
\subsubsection{Capacity or compliance}\label{capacitanza}
The \textbf{capacity} (or compliance) is the ability of a hollow organ to stretch and increase in volume with increasing pressure. It is the fluid-dynamic equivalent of electrical capacity.\\
In the case of blood vessels, this physically means that vessels with higher compliance deform more easily than vessels with lower compliance under the same conditions of pressure and volume. \\
Venous compliance is about 30 times greater than arterial compliance, largely because of their thinner walls.\\
The $C$ compliance of a blood vessel is directly proportional to the elasticity of its walls and is a measure of the ratios of pressure changes to volume changes. It is defined as: $C=\frac{\Delta V}{\Delta P}$, where $\Delta V$ is the volume change, $\Delta P$ is the pressure change, i.e., the difference between intravascular and external pressure.\\
One feature that makes its estimation a subject of study is the difficulty of measurement: arterial compliance can be measured by several techniques, but most of them are invasive and not clinically appropriate. 

%%%%%%%%%%%%%% EFFETTO WINDKESSEL
\subsection{Windkessel effect}
The \textbf{windkessel effect} is a term used in medicine to explain the waveform of arterial pressure in terms of the interaction between the systolic volume and compliance of the aorta and large elastic arteries and the resistance of smaller arteries and arterioles.\\
Windkessel, from German, means \textit{air chamber} and was used in the 18th century by firefighters to ensure a continuous supply of water in fighting fires; in the cardiovascular case it is an elastic reservoir, but the operation is the same. Figure \ref{windkesselEffect} illustrates this analogy.

\begin{figure}[h]
    \centering
    \includegraphics[width=0.7\textwidth]{images/Windkessel/WindkesselEffect.png}
    \caption{Illustration of the analogy on the windkessel effect \cite{wiki:WindkesselEffect}.}
    \label{windkesselEffect}
\end{figure}

\newpage
As introduced earlier, elastic arteries relax when blood pressure rises during systole and retract when blood pressure falls during diastole, as shown in figure \ref{windkesselEffect(libro)}. \\
Because the rate of blood entering this type of artery exceeds the rate of blood leaving, there is a net deposit of blood during systole, which is discharged during diastole. The compliance of the aorta and the great elastic arteries is thus analogous to that of a capacitor; in other words, these arteries collectively act as a hydraulic accumulator.\\
The Windkessel effect helps dampen blood pressure fluctuation during the cardiac cycle and helps maintain organ perfusion during diastole, when cardiac ejection ceases. 


\begin{figure}[h]
    \centering
    \includegraphics[width=0.7\textwidth]{images/Windkessel/WindkesselEffect(libro).PNG}
    \caption{Illustration of the windkessel effect  \cite{AaronsonPhilipI.PhilipIrving2020Tcsa}.}
    \label{windkesselEffect(libro)}
\end{figure}

\newpage

\section{Two-element Windkessel model}
Over the centuries, the arterial system has been modeled in many ways and with different strategies. An example was shown in the introduction to this chapter.
Stephen Hales\footnote{Stephen Hales (1677-1761) was an English clergyman who made important contributions to a number of scientific fields including botany, pneumatic chemistry, and physiology.} He was the first to measure blood pressure and
noted that pressure in the arterial system is not constant,
but varies over the course of the heartbeat, suggesting that
pressure variations could be related to the elasticity of the large arteries. \\
It was Otto Frank\footnote{Otto Frank (1865 - 1944) was a German physician and physiologist who contributed to cardiac physiology and cardiology} who quantitatively formulated and popularized the so-called \textbf{two-element Windkessel model}
consisting of a resistance element and a compliance element.


Poiseuille's law states that resistance is inversely proportional to the radius of the blood vessel to the fourth power. The
resistance to flow in the arterial system is therefore mainly
in the resistance vessels: the smaller arteries and the
arterioles. When all the individual resistances in the microcirculation are correctly summed, the resistance of the entire
systemic vascular bed called 
peripheral (total) resistance. The peripheral resistance, R, can
be calculated as:

\begin{equation}\label{R}
R = \frac{ P_{\text{ao; mean}}-P_{\text{ven; mean}}}{CO}\approx \frac{ P_{\text{ao; mean}}}{CO},
\end{equation}

where $P_{\text{ao; mean}}$ is the mean aortic pressure, while $P_{\text{ven; mean}}$ is the mean venous pressure, $CO$ is the cardiac output.\\
The compliance component is mainly
determined by the elasticity of the large
arteries. It can be obtained by summing the compliance
of all vessels and is therefore called total arterial compliance.\\
For the calculation of $C$ (as shown in \ref{capacitanza}), it is very difficult to perform an experiment in which a volume is injected into the arterial system without any loss in the periphery. For this reason, several methods have been developed to derive total arterial compliance without resorting to experiments.\\
The two-element Windkessel predicts that in diastole,
when the aortic valve is closed, the pressure decays
exponentially with a characteristic decay time, with which peripheral resistance can be calculated with aortic pressure in diastole and an estimate of total arterial compliance.\\
The average flow, i.e., cardiac output $CO$, is then derived from (\ref{R}). \\
The Windkessel is a so-called \textit{lumped} model. In other words, this model describes the entire arterial system
in terms of a pressure-flow relationship at its inlet,
exploiting two parameters that have physiological significance.\\
Interestingly, research in the past has mainly focused on peripheral resistance, while the contribution of total arterial compliance has often been neglected. 


The two-element Windkessel model shows how the load on the heart consists of peripheral and total arterial resistance and that both play an important role.\\

With the development of the electromagnetic flowmeter and thus the measurement of aortic flow, it became clear that in systole the relationship between pressure and flow was poorly predicted by the two-element Windkessel model. Measurements of aortic flow and developments in technological possibilities led to improvements in the two-element model: the three-element model also takes into account aortic valve resistance, while the four-element model also takes into account blood flow inertia.\\

\vspace{1cm}
To show the accuracy that the two-element Windkessel model can guarantee, the procedure in Python to obtain the pressure estimate by comparing it with the same real data used in the introductory section is shown to follow.\\
For the Python code shown, you need to import the right libraries shown in the code \ref{configurazione1}; you also need the code \ref{flusso} that defines the flow function\footnote{The code uses the real data taken from \cite{westerhof_arterial_2008}, so you need the code $\ref{datiReali}$.}.


\newpage

\subsection{Definition of the model}
As clarified in the definitions in \ref{terminologia}, the Windkessel model can be expressed in circuit form as in figure \ref{circuito}.

\vspace{1cm}

\begin{figure}[h]
    \centering
    \includegraphics[width=0.6\textwidth]{images/Windkessel/Windkessel2Element.png}
    \caption{Circuit form of the two-element Windkessel model.}
    \label{circuito}
\end{figure}

The two-element Windkessel model is characterized by the equation:
\[
\frac{dP}{dt}=\frac{1}{C}(Q_{in}-Q_{out}),
\]
where $C$ is the systemic compliance, $R_1$ and $R_2$ are the proximal and peripheral resistance (of all arteries and arterioles), $Q_{out}$ the outflow from the capacitator.\\
Let us also consider $Q_{out}=\frac{P-P_d}{R_2}$, where $P_d$ is the distal pressure (i.e., subsequent to the peripheral resistance, in the course of the elaboration, when not otherwise specified, is set to $5 mmHg$), and
\begin{equation}\label{Pin}
    P_{in}=P+R_1Q_{in},
\end{equation}
from which we can then write:
\begin{equation}\label{equation}
\frac{dP}{dt}=\frac{1}{C}\left( Q_{in}-\frac{P-P_d}{R_2}\right).
\end{equation}



\subsection{Compliance estimation}
To work with the model equation, it is necessary to know compliance. To do this, you minimize the function
\[
f_C(C)=\sum_{n=1}^M\left( P_{in}(t_n)-\hat{P}_{in}(t_n)\right)^2,
\]
where $\hat{P}_{in}$ is the measured value of pressure at $M$ points $t_n$ and $P_{in}$ is defined as in (\ref{Pin}) and requires solving (\ref{equation}) with initial condition $P(0)=\hat{P}_{in}(t_1)$.\\
For this first approach it is assumed $R_1=0$ and $R_2=R$ where $R$ is the total peripheral resistance. From (\ref{Pin}) it is derived that $P_{in}=P$.\\

\newpage

Since it is necessary to solve the equation (\ref{equation}), it is necessary to define it in Python as a function, as in the code \ref{ODE}. You then define the function $f_C$ in the code \ref{fC}. The graph of the function $f_C$ is shown in figure \ref{plotfC}.

\begin{figure}[h]
    \centering
    \includegraphics[width=0.75\textwidth]{images/Windkessel/f_C.pdf}
    \caption{Graph of $f_C$. Code \ref{plotfC-code}.}
    \label{plotfC}
\end{figure}



So it is now possible to estimate $C$ that minimizes $f_C$ by relying on the Python library \texttt{scipy.optimize}, as done in the code \ref{stimaC}, in this way $C=1.83288 mL/mmHg$ is found.



Now solving the equation (\ref{equation}) with the estimated $C$ yields what is in the figure \ref{soluzioneCapprossimata}.

\begin{figure}[h]
    \centering
    \includegraphics[width=0.75\textwidth]{images/Windkessel/modelloCstimata.pdf}
    \caption{Graph of the approximate solution of the equation (\ref{equation}) with $C$ estimated. Code \ref{plotSoluzioneCstimata}.}
    \label{soluzioneCapprossimata}
\end{figure}


\subsection{Estimation of compliance and resistance}\label{stimaCR}
By changing the definition of the resistances, the approximation can be improved. We now define: $R_1=(1-\alpha)R$ and $R_2=\alpha R$ where $\alpha\in[0,1]$. So now the function $f_C$ also depends on $\alpha$, and it is therefore necessary to update it as shown in the code \ref{fCa}.
In the code \ref{stimaCA} the parameters $C$ and $\alpha$ are estimated. We obtain: $C=2.11579 mL/mmHg$ and $\alpha=0.97134$.
Now solving the equation (\ref{equation}) with the estimated parameters of $C$ and $\alpha$ yields a graph that approximates the actual data with remarkable accuracy, as shown in figure \ref{soluzioneCalphaapprossimata}.

\begin{figure}[h]
    \centering
    \includegraphics[width=0.75\textwidth]{images/Windkessel/modelloCalphastimata.pdf}
    \caption{Graph of the approximate solution of the equation (\ref{equation}) with $C=2,11579mL/mmHg$ and $\alpha=0,97134$. Code \ref{soluzioneCalphastimate}.}
    \label{soluzioneCalphaapprossimata}
\end{figure}


\section{Periodic forcing}
It is a well-known result that given a homogeneous linear system with constant coefficients, i.e.
\begin{equation}\label{sistema omogeneo}
    \bm{\dot{x}}(t)=A\bm{x}(t),
\end{equation}
the stability of the exact solution of that system is determined by the real part of the eigenvalues of the matrix of coefficients $A$. In particular, a necessary and sufficient condition for the system to be asymptotically stable is that all eigenvalues of $A$ have strictly negative real part. In that case there exist positive constants $\alpha, \beta$ such that
\begin{equation}\label{inequation}
||e^{At}||\leq \beta e^{-\alpha t}\quad t\geq 0.    
\end{equation}
It is now shown that when a periodic forcing function $\bm{b}(t)$ is added to the system (\ref{sistema omogeneo}) to obtain the
inhomogeneous system 
\begin{equation}\label{sistema disomogeneo}
    \bm{\dot{x}}(t)=A\bm{x}(t)+\bm{b}(t),
\end{equation}
if the homogeneous part is asymptotically stable, then any solution of the inhomogeneous system (\ref{sistema disomogeneo}) converges to a periodic solution as $t$ increases. The exact solution of the inhomogeneous system, obtained by the method of variation of constants (or Lagrange's method), is:
\[
\bm{x}(t)=e^{At}\bm{x}_0+e^{At}\int_0^t e^{A(t-s)}\bm{b}(s)ds
\]
Where $\bm{x}(0)= \bm{x}_0$. Given $\bm{b}(t)$ periodic forcing function, i.e., $\bm{b}(T_0) = \bm{b}(0)$ for some $T_0 > 0$, it is possible to obtain a periodic solution of the inhomogeneous system (\ref{sistema disomogeneo}), thus satisfying $\bm{x}(T_0) = \bm{x}(0)$, by appropriately choosing the initial condition $\bm{x}_0$. With simple math we obtain that the above initial condition $\bm{x}^P_0$ to obtain a periodic solution is given by
\[
\bm{x}_0^P=(I-e^{AT_0})^{-1} e^{AT_0}\int_0^{T_0} e^{-As}\bm{b}(s)ds.
\]
It is now shown that, starting from an initial condition other than $\bm{x}^P_0$, the solution values converge to the values of the periodic solution as $t$ increases. Let $\bm{x}^P$ be the periodic solution corresponding to the initial condition $\bm{x}^P_0$, let $\bm{x}^{NP}$ be any solution of the inhomogeneous system associated with some initial condition $\bm{x}_0^{NP}$. Then, calculating the norm of the difference between the two solutions and applying (\ref{inequation}), we obtain
\[
||\bm{x}^P(t)-\bm{x}^{NP}(t)||=||e^{At}(\bm{x}_0^P-\bm{x}_0^{NP})||\leq \beta e^{-\alpha t} ||\bm{x}_0^P-\bm{x}_0^{NP}||,
\]
in which $\beta e^{-\alpha t}\rightarrow 0$ for $t\rightarrow +\infty$, that is what was intended to be proved. 

Then if the forcing function $\bm{b}(t)$ is periodic and if the homogeneous part of the system is asymptotically stable, then the exact solution of the inhomogeneous problem will converge to the periodic solution of the inhomogeneous system itself as $t$ increases for any permissible choice of the initial condition.

\subsection{Application to the Windkessel model}
In the Windkessel model, the periodic forcing is the cardiac flux, as observed in equation (\ref{equation}). So, following the above, instead of finding an approximation of the solution using the flow of a single cardiac cycle, an approximation is found over several cycles until the solution is, barring a tolerance, equal to the previous one.

To do this, simply replace the flow function code \ref{flusso} with \ref{flusso periodico} and when solving the differential equation replace \verb|t_span| and of \verb|t_eval| with \verb|t_span = [time[0], numCycles+time[-1]]| and \verb|t_eval = numCycles+time| where \verb|numCycles| is the number of cardiac cycles against which to solve the differential equation. 

\newpage

With these changes to the code we repeat what we saw earlier for the approximation of $\alpha$ and $C$ and find: $C=2.03424$ and $\alpha= 0.97354$. The approximation of the solution after twenty cardiac cycles with these parameters is shown in figure \ref{soluzionePeriodicaCalphaapprossimata}.


\begin{figure}[h]
    \centering
    \includegraphics[width=0.75\textwidth]{images/Windkessel/modelloPeriodicoCalphastimata.pdf}
    \caption{Graph of the solution of the equation (\ref{equation}) approximated after twenty cardiac cycles with $C=2,03424mL/mmHg$ e $\alpha=0,97354$.}
    \label{soluzionePeriodicaCalphaapprossimata}
\end{figure}

Note that in the figure \ref{soluzioneCalphaapprossimata} the error of the solution approximated by the real pressure is in norm infinite $5.005246$ and in norm two $55.584739$, while in the figure \ref{soluzionePeriodicaCalphaapprossimata} in norm infinite is $4.654959$ and in norm two is $46.429220$. So the second approach, the one that takes into account the convergence of each solution to the periodic solution, generates more accurate results at the expense of more heart cycles to consider.


\subsection{Running time} \label{Windkessel: tempo esecuzione}
With the python command \verb+%%timeit -r 20+ (referred to as a \textit{built-in magic command} of jupyter notebooks) it is possible to compute the average time and standard deviation of 20 executions of python code contained in a jupyter notebook cell. When the approximation of the solution to convergence is sought, it is verified to be quite similar to the previous one, unless $5\times 10^{-3}$. As explained in \ref{Risultati training: dataset}, the input dataset consists of 3375 combinations of parameters $C, R_1, R_2$, for each of which the solution to convergence is sought. Over all the combinations we obtain different numbers of cycles required to reach convergence, all strictly less than one hundred. 

\newpage

To give an idea of how long this approach takes, shown in table \ref{tab: tempo windkessel} are the amount of time it takes to find the solution after several cardiac cycles.

\vspace{1cm}

% Please add the following required packages to your document preamble:
% \usepackage{lscape}

\begin{table}[!htb]
\centering
\begin{tabular}{cc}
\hline
\textbf{Cicli cardiaci} & \textbf{\begin{tabular}[c]{@{}c@{}}Tempo esecuzione \\ (media + dev. std.)\end{tabular}} \\ \hline
1   & 17.1 ms + 1.23 ms \\
10  & 90.4 ms ± 3.75 ms \\
20  & 189 ms ± 10.3 ms  \\
30  & 270 ms ± 12.9 ms  \\
40  & 362 ms ± 13.3 ms  \\
50  & 458 ms ± 15.1 ms  \\
60  & 526 ms ± 21.7 ms  \\
70  & 602 ms ± 15.9 ms  \\
80  & 711 ms ± 17.3 ms  \\
90  & 802 ms ± 32.3 ms  \\
100 & 883 ms ± 17.6 ms  \\ \hline
\end{tabular}
\caption{Tempo di esecuzione per trovare l'approssimazione della soluzione del modello Windkessel su diversi cicli cardiaci. }
\label{tab: tempo windkessel}
\end{table}





%%%%%%%%%%%%%%%%%%%%%%%%%%%%%%%%%%
%%%%%%%%% ANALISI SENSITIVITA'
%%%%%%%%%%%%%%%%%%%%%%%%%%%%%%%%
\section{Local sensitivity analysis}\label{sensitività}
In this section, the sensitivity of the model output to parameter changes is investigated by performing a \textit{local sensitivity analysis}. \\
The output of the model (the pressure $P$) is used to calculate four variables:
\begin{itemize}
  \item $\text{MAP}$: Mean Arterial Pressure, is calculated as $\text{mean}(P)$;
  \item $\text{DBP}$: Diastolic Blood Pressure, is calculated as $\text{min}(P)$;
  \item $\text{SBP}$: Systolic Blood Pressure,  is calculated as $\text{max}(P)$;
  \item $\text{PP}$: Pulse Pressure, is calculated as $\text{max}(P)-\text{min}(P)$.
\end{itemize}

\newpage

From the output obtained with the values of $C$ and $\alpha$ estimated in the code \ref{stimaCA}, the values of the variables are reported by comparing them with the actual values (of healthy individuals) in the table \ref{tab:variabili}.

% Please add the following required packages to your document preamble:
% \usepackage{graphicx}
\begin{table}[!htb]
\centering
\resizebox{0.7\textwidth}{!}{%
\begin{tabular}{cccc}
\hline
\textbf{Variabili} & \textbf{Unità misura} & \textbf{Modello} & \textbf{Reale} \\ \hline
MAP                & mmHg                  & 92.91            & 65 - 110       \\
DBP                & mmHg                  & 74.86            & \textless 80   \\
SBP                & mmHg                  & 109.78           & \textless 120  \\
PP                 & mmHg                  & 34.91            & $\sim$ 40       \\ \hline
\end{tabular}%
}
\caption{Valori variabili calcolati con il modello Windkessel e parametri di input ($C, \alpha$) stimati. I valori reali sono ripresi da \cite{AaronsonPhilipI.PhilipIrving2020Tcsa}.}
\label{tab:variabili}
\end{table}

The concept of local sensitivity is formalized as.
\[
S_\mathcal{M}^\mathcal{P}=\frac{\hat{\mathcal{P}}}{\hat{\mathcal{M}}} \frac{\partial \mathcal{M}(\mathcal{P})}{\partial \mathcal{P}},
\]
where $S_\mathcal{M}^\mathcal{P}$ is the sensitivity of the variable $\mathcal{M}$ to the change of the parameter $\mathcal{P}$. The values $\hat{\mathcal{M}}$ is the value calculated from the model output while $\hat{\mathcal{P}}$ is the set value of the parameter.\\

To calculate the partial derivative, use was made of the \textit{centered finite difference method} readjusted to the calculation of local sensitivity, as shown in the code \ref{differenzefinite}. Ten percent of its value was used as the variance for each parameter.

For each parameter $\mathcal{P}=\{C,R_1,R_2,P_d\}$ we then solve the problem at initial values (\ref{equation}) twice: once with the parameter increased by ten percent, once with the parameter decreased by ten percent; the other parameters are left unchanged. The two $P$ outputs obtained (along with the parameter change) are then used in the centered finite difference method for approximating the partial derivative. Finally, the sensitivity is calculated. \\
The example in the case $\mathcal{P}=C$ is shown in the code \ref{Csensitivity}.
Table \ref{tab:local sensitivity} summarizes the sensitivity values obtained in descending order in modulus.

% Please add the following required packages to your document preamble:
% \usepackage{graphicx}
\begin{table}[h]
\centering
\resizebox{\textwidth}{!}{%
\begin{tabular}{ccccc}
\hline
\textbf{Rank} & \textbf{MAP}      & \textbf{DBP}      & \textbf{SBP}      & \textbf{PP}       \\ \hline
1             & $R_2$ ( -0.1907 ) & $R_2$ ( -0.0870 ) & $C$ ( 0.2464 )    & $C$ ( 0.7928 )    \\
2             & $C$ ( 0.1271 )    & $C$ ( -0.0085 )   & $R_2$ ( -0.0871 ) & $R_1$ ( -0.1695 ) \\
3             & $R_1$ ( -0.0285 ) & $R_1$ ( -0.0066 ) & $R_1$ ( -0.0584 )  & $R_2$ ( -0.0871 )  \\
4             & $P_d$ ( -0.0099 )  & $P_d$ ( -0.0006)  & $P_d$ ( -0.0049 )  & $P_d$ ( -0.0141 )  \\ \hline
\end{tabular}%
}
\caption{Sensitività locale delle variabili $\mathcal{M}=\{MAP, DBP, SBP, PP\}$ al variare del valore dei parametri $\mathcal{P}=\{C,R_1,R_2,P_d\}$. I parametri sono inseriti in ordine decrescente (in modulo) di sensitività.}
\label{tab:local sensitivity}
\end{table}

Table \ref{tab:VariazioneParametri-Variabili} shows the variation of variables as a function of parameter variation.

\newpage

% Please add the following required packages to your document preamble:
% \usepackage{multirow}
% \usepackage{lscape}
\begin{landscape}
\begin{table}
\centering
\begin{tabular}{ccccccccccccc}
\hline
\textbf{Variabili} &
  \textbf{$\text{MAP}_S$} &
  \textbf{$\text{MAP}_{+10\%}$} &
  \textbf{$\Delta_{MAP}$} &
  \textbf{$\text{DBP}_S$} &
  \multicolumn{1}{l}{\textbf{$\text{DBP}_{+10\%}$}} &
  \multicolumn{1}{l}{\textbf{$\Delta_{DBP}$}} &
  \multicolumn{1}{l}{\textbf{$\text{SBP}_S$}} &
  \multicolumn{1}{l}{\textbf{$\text{SBP}_{+10\%}$}} &
  \textbf{$\Delta_{SBP}$} &
  \textbf{$\text{PP}_S$} &
  \textbf{$\text{PP}_{+10\%}$} &
  \textbf{$\Delta_{PP}$} \\ \hline
\textbf{$C$} &
  \multirow{4}{*}{92.91} &
  91.81 &
  -1.18\% &
  \multirow{4}{*}{74.86} &
  74.92 &
  +0.08\% &
  \multirow{4}{*}{109.78} &
  107.32 &
  -2.24\% &
  \multirow{4}{*}{34.91} &
  32.40 &
  -7.19\% \\
\textbf{$R_1$} &
   &
  93.18 &
  +0.29\% &
   &
  74.91 &
  +0.07\% &
   &
  110.43 &
  +0.59\% &
   &
  35.52 &
  +1.75\% \\
\textbf{$R_2$} &
   &
  94.54 &
  +1.75\% &
   &
  74.93 &
  +0.09\% &
   &
  110.66 &
  +0.80\% &
   &
  35.73 &
  +2.35\% \\
\textbf{$P_d$} &
   &
  93.01 &
  +0.10\% &
   &
  74.86 &
  +0.00\% &
   &
  109.84 &
  +0.05\% &
   &
  34.98 &
  +0.20\% \\ \hline
\end{tabular}
\caption{Variazione delle variabili all'aumento del dieci percento dei parametri. $\mathcal{M}_S=\{\text{MAP}_S, \text{DBP}_S, \text{SBP}_S, \text{PP}_S\}$ sono le variabili ottenute con parametri standard (stimati precedentemente), $\mathcal{M}_{+10\%}=\{\text{MAP}_{+10\%}, \text{DBP}_{+10\%}, \text{SBP}_{+10\%}, \text{PP}_{+10\%}\}$ sono le variabili ottenute con singoli parametri aumentati del dieci percento, $\Delta=\{\Delta_{MAP}, \Delta_{DBP}, \Delta_{SBP}, \Delta_{PP}\}$ sono le variazioni percentuali della variabile a pedice.}
\label{tab:VariazioneParametri-Variabili}
\end{table}
\end{landscape}

\newpage
From the table \ref{tab:VariazioneParametri-Variabili} it is understood that $P_d$ has low influence in all variables.\\
Local sensitivity thus behaves as expected: the higher this is (in modulus), the greater the percentage change (in modulus) in the variable by changing the parameter.\\
For example in the case $\mathcal{M}=\text{MAP}$, the change in percent is highest by modifying $R_2$, followed by the change obtained by modifying $C$, then by that by modifying $R_1$ and $P_d$. It then follows the same order as the sensitivity of $\text{MAP}$ (in modulus).\\
It can be seen that $\text{DBP}$ depends sparsely on all parameters.\\
In addition $C$ has strong influence in the variation of $\text{SBP}$ and $\text{PP}$ (the latter also influenced not insignificantly by $R_1$ and $R_2$).\\
Overall $R_2$ has greater influence than $R_1$, although the influence of the latter is not negligible.\\

Note that for each parameter the percent change in each variable has inverse sign of sensitivity. This follows from the method of centered finite differences: a negative change would result in a percent change concordant with the sign of sensitivity.
\chapter{Methodology and results training}\label{Capitolo: risultati training}


This chapter explains the regression problem of interest in the work and the training process of the model. The python library used for the training part of the model is introduced by reporting the programming details; finally, the results obtained are reported.


\begin{textblock*}{0.64\textwidth}(3.5cm+0.36\textwidth,18.5cm)
\epigraph{A baby learns to crawl, walk and then run. We are in the crawling stage when it comes to applying machine learning.}{Dave Waters}
\end{textblock*}

\newpage

\section{Regression problem}
The chapter \ref{windkessel} showed how the Windkessel model works: starting with cardiac flow, proximal resistance, peripheral resistance and compliance, the model is able to give an approximation of the pressure trend during the cardiac cycle.\\
The problem facing the paper is to obtain MAP, DBP, SBP, PP without having to solve the differential model equation. To do this, the supervised learning approach is used: a large database of combinations of the input parameters associated with the output found by solving the windkessel model equation is generated (as seen in \ref{windkessel}), the dataset is divided into training, validation, and testing set (as seen in \ref{dataset}), training is performed with Gaussian processes as explained in \ref{neiProcessiGaussiani}.


\section{Details of the training}

\subsection{GPErks}
The python library GPErks, a project developed by Dr. Stefano Longobardi and used in biomedical applications, for example in \cite{doi:10.1098/rsta.2019.0334}, was used for training.\\
PyTorch, a python library very common in machine learning, was mainly used to create the library.\\

Since the paper would take advantage of the GPErks library, the theoretical basis on which it rests has been explained in previous chapters, particularly in the chapter \ref{machineLearning}.\\

In particular, GPErks divides the input database (consisting of input - output pairs of examples) into training set, validation set, and testing set. It defines the Gaussian process with a linear mean function and a squared exponential kernel (these are not mandatory choices, but they are the most common). It defines the marginal likelihood and uses an optimization method for its maximization (you can choose the optimization method; in the case of the paper, Adam's method was chosen). The library has implemented three different earlystoppers useful for terminating training before incurring overfitting. 

GPErks also shows the training results in the form of loss function graphs and also reports the values of \textit{metrics} (as they are called in the library) such as R2Score and Mean Squared Error (the two chosen for the purposes of the paper). Finally, it generates inferential results about the training, showing how the learning results can predict outputs from the input data.

\subsection{Gaussian process}
A Gaussian process is used with mean function $m(\mathbf{x})=\beta_0+\beta_1x_1+\beta_2x_2+\beta_3x_3$ and covariance function the squared exponential kernel with a lengthscale parameter for each dimension (as described in \ref{multidimensionalKernel}).

Specifically, in the elaborate we follow the code structure of example four found on the library's github page. 

\subsection{Hyperparameter optimization}
The marginal likelihood is maximized, following what was said in \ref{neiProcessiGaussiani}. Adams' method (explained in \ref{adam}) is used to do this.

\subsection{Input dataset}\label{Risultati training: dataset}
The input dataset consists of a file with the input parameters: $C$, $R_1$ (proximal resistance), $R_2$ (distal resistance). No $P_d$ is added to the input data because of what was seen in \ref{sensitività}: the local sensitivity of $P_d$ (i.e., the influence on the output parameters) is very low ($<0.02$ in modulus), which means that there is no causal relationship between the distal pressure and the output parameters. Note that leaving $P_d$ in the list of input parameters may worsen the training results because the model would try to learn a causal relationship between $P_d$ and the outputs while there is only a random relationship.\\
Three lists are created to generate it, one for each input parameter; specifically:
\[
C\in [1.4, 2.6] \quad R_1\in [0.01, 0.1] \quad R_2\in [0.6, 1.3]
\]
For each parameter, the list consists of fifteen elements. All parameter combinations are then generated, mixed\footnote{Mixing combinations before splitting the dataset into training, validation, and testing sets avoids overfitting problems.} and placed in a file.\\
For each combination of input parameters, the Windkessel model is then solved, as done in \ref{windkessel}. With the result MAP, DBP, SBP, PP are found and are saved in one file for each output. In this way, four files are obtained with an input for each parameter combination contained in the first generated file.\\
The input dataset is then divided (by GPErks) into training set, validation set, and testing set. The proportion is $60\%$ training set, $20\%$ validation set, $20\%$ testing set.\\

\newpage
Figure \ref{distribuzioneDataset} shows the distribution of the database. This depends on the fact that all possible combinations of input data are constructed.

\begin{figure}[h]
    \centering
    \includegraphics[width=1\textwidth]{images/Training (risultati)/database.png}
    \caption{Distribution of data in the database.}
    \label{distribuzioneDataset}
\end{figure}

\newpage
\section{Training results}


%******************************
%*********** MAP **************
%******************************
\subsection{Mean arterial pressure (MAP)}
$\text{lr}=0.1$ is set and the early stopper \textit{GLEarlyStoppingCriterion} is used with parameters: $\alpha = 5$, $\text{patience}=2$.


% **********
% MAP - loss
% **********
\subsubsection{Training and validation loss}
The training needed fifty-two EPOCHS (each EPOCH consists of evaluating the gradient on each dataset element), concluded with $\text{R2Score}=0.9999$, $\text{MeanSquaredError}=0.0001$. Figure \ref{MAP - loss} shows the training and validation loss trend with MSE and R2Score; early stopper trend in green.
\begin{figure}[h]
    \centering
    \includegraphics[width=1\textwidth]{images/Training (risultati)/MAP/MAP - loss.png}
    \caption{MAP: progress of training and validation loss, early stopper, R2Score e MSE.}
    \label{MAP - loss}
\end{figure}

\newpage



% **********
% MAP - inference
% **********
\subsubsection{Approximation of input data}
Figure \ref{MAP - inference} shows how the predictions approximate the input data. The length of the error bars is $0.0015$, so they are very short indicating high accuracy.

\begin{figure}[!htb]
    \centering
    \includegraphics[width=1\textwidth]{images/Training (risultati)/MAP/MAP - inference.png}
    \caption{MAP: predictions about the input data.}
    \label{MAP - inference}
\end{figure}



% **********
% MAP - C
% **********
\subsubsection{Dependence on $C$}
To study the dependence of MAP on $C$, ninety equidistant $C$ values are taken in the same range used for creating the input database and, fixed the values of $R_1$ and $R_2$ to those found in their approximation in \ref{stimaCR}, a file of $C$, $R_1$ and $R_2$ combinations is generated. For each combination, the pressure approximation with the Windkessel model is found and MAP calculated; with the combination of compliance and strengths, MAP is then estimated using the model already trained. The same is then done on two intervals adjacent to the training interval and broaden the $10\%$ of the training interval. The overall result is shown in figure \ref{MAP - C - full}, the result in the training interval alone in \ref{MAP - C - training}, the result in the individual adjacent intervals in \ref{MAP - C - sx} and \ref{MAP - C - dx}.

\begin{figure}
    \centering
    \includegraphics[width=1\textwidth]{images/Training (risultati)/MAP/MAP - C - full.pdf}
    \caption{Dependence of MAP on $C$ on the training interval and two adjacent intervals.}
    \label{MAP - C - full}
\end{figure}


\begin{figure}
    \centering
    \includegraphics[width=1\textwidth]{images/Training (risultati)/MAP/MAP - C - training.pdf}
    \caption{Dependence of MAP on $C$ over the training interval.}
    \label{MAP - C - training}
\end{figure}


\begin{figure}
    \centering
    \includegraphics[width=1\textwidth]{images/Training (risultati)/MAP/MAP - C - sx.pdf}
    \caption{Dependence of MAP on $C$ on the adjacent interval to the left of the training interval.}
    \label{MAP - C - sx}
\end{figure}



\begin{figure}
    \centering
    \includegraphics[width=1\textwidth]{images/Training (risultati)/MAP/MAP - C - dx.pdf}
    \caption{Dependence of MAP on $C$ on the adjacent interval to the right of the training interval.}
    \label{MAP - C - dx}
\end{figure}




\newpage
% **********
% MAP - R1
% **********
\subsubsection{Dependence on $R_1$}
The same approach is used to study the dependence on $R_1$. The overall result is shown in figure \ref{MAP - R1 - full}, the result in the training interval alone in \ref{MAP - R1 - training}, the result in the individual adjoint intervals in \ref{MAP - R1 - sx} and \ref{MAP - R1 - dx}.

\begin{figure}[!htb]
    \centering
    \includegraphics[width=1\textwidth]{images/Training (risultati)/MAP/MAP - R1 - full.pdf}
    \caption{Dependence of MAP on $R1$ on the training interval and two adjacent intervals.}
    \label{MAP - R1 - full}
\end{figure}

\vspace{1cm}

\begin{figure}[!htb]
    \centering
    \includegraphics[width=1\textwidth]{images/Training (risultati)/MAP/MAP - R1 - training.pdf}
    \caption{Dependence of MAP on $R1$ over the training interval.}
    \label{MAP - R1 - training}
\end{figure}

\begin{figure}
    \centering
    \includegraphics[width=1\textwidth]{images/Training (risultati)/MAP/MAP - R1 - sx.pdf}
    \caption{Dependence of MAP on $R1$ on the adjacent interval to the left of the training interval.}
    \label{MAP - R1 - sx}
\end{figure}


\begin{figure}
    \centering
    \includegraphics[width=1\textwidth]{images/Training (risultati)/MAP/MAP - R1 - dx.pdf}
    \caption{Dependence of MAP on $R1$ on the adjacent interval to the right of the training interval.}
    \label{MAP - R1 - dx}
\end{figure}




\newpage
% **********
% MAP - R2
% **********
\subsubsection{Dependence on $R_2$}
The same approach is used to study the dependence on $R_2$. The overall result is shown in figure \ref{MAP - R2 - full}, the result in the training interval alone in \ref{MAP - R2 - training}, the result in the individual adjoint intervals in \ref{MAP - R2 - sx} and \ref{MAP - R2 - dx}.

\begin{figure}[!htb]
    \centering
    \includegraphics[width=1\textwidth]{images/Training (risultati)/MAP/MAP - R2 - full.pdf}
    \caption{Dependence of MAP on $R2$ over the training interval and two adjacent intervals.}
    \label{MAP - R2 - full}
\end{figure}

\vspace{1cm}

\begin{figure}[!htb]
    \centering
    \includegraphics[width=1\textwidth]{images/Training (risultati)/MAP/MAP - R2 - training.pdf}
    \caption{Dependence of MAP on $R2$ over the training interval.}
    \label{MAP - R2 - training}
\end{figure}

\begin{figure}
    \centering
    \includegraphics[width=1\textwidth]{images/Training (risultati)/MAP/MAP - R2 - sx.pdf}
    \caption{Dependence of MAP on $R2$ on the adjoint interval to the left of the training interval.}
    \label{MAP - R2 - sx}
\end{figure}


\begin{figure}
    \centering
    \includegraphics[width=1\textwidth]{images/Training (risultati)/MAP/MAP - R2 - dx.pdf}
    \caption{Dependence of MAP on $R2$ on the adjacent interval to the right of the training interval.}
    \label{MAP - R2 - dx}
\end{figure}







\newpage
%******************************
%*********** DBP **************
%******************************
\subsection{Diastolic blood pressure (DBP)}
$\text{lr}=0.07$ is imposed and the early stopper \textit{GLEarlyStoppingCriterion} is used with parameters: $\alpha = 2$, $\text{patience}=2$.



% **********
% DBP - loss
% **********
\subsubsection{Training and validation loss}
The training needed one hundred and thirty-one EPOCHS, concluded with $\text{R2Score}=0.9999$, $\text{MeanSquaredError}=0.0001$. Figure \ref{DBP - loss} shows the trend of training and validation loss with MSE and R2Score; in green is the trend of early stopper.
\begin{figure}[h]
    \centering
    \includegraphics[width=1\textwidth]{images/Training (risultati)/DBP/DBP - loss.png}
    \caption{DBP: progress of training and validation loss, early stopper, R2Score e MSE.}
    \label{DBP - loss}
\end{figure}

\vspace{-0.5cm}

% **********
% DBP - inference
% **********
\subsubsection{Approximation of input data}
Figure \ref{DBP - inference} shows how the predictions approximate the input data. The length of the error bars is $0.0028$.

\begin{figure}[h]
    \centering
    \includegraphics[width=1\textwidth]{images/Training (risultati)/DBP/DBP - inference.png}
    \caption{DBP: input data predictions.}
    \label{DBP - inference}
\end{figure}



% **********
% DBP - C
% **********
\subsubsection{Dependence on $C$}
The overall result is shown in figure \ref{DBP - C - full}, the result in the training interval alone in \ref{DBP - C - training}, the result in the individual adjacent intervals in \ref{DBP - C - sx} and \ref{DBP - C - dx}. In all intervals the model succeeds in generating excellent predictions with low uncertainty (represented by the error bars).

\vspace{1cm}

\begin{figure}[!htb]
    \centering
    \includegraphics[width=1\textwidth]{images/Training (risultati)/DBP/DBP - C - full.pdf}
    \caption{Dependence of DBP on $C$ on the training interval and two adjacent intervals.}
    \label{DBP - C - full}
\end{figure}

\vspace{0.32cm}

\begin{figure}[!htb]
    \centering
    \includegraphics[width=1\textwidth]{images/Training (risultati)/DBP/DBP - C - training.pdf}
    \caption{Dependence of DBP on $C$ over the training interval.}
    \label{DBP - C - training}
\end{figure}



\begin{figure}
    \centering
    \includegraphics[width=1\textwidth]{images/Training (risultati)/DBP/DBP - C - sx.pdf}
    \caption{Dependence of DBP on $C$ on the adjacent interval to the left of the training interval.}
    \label{DBP - C - sx}
\end{figure}


\begin{figure}
    \centering
    \includegraphics[width=1\textwidth]{images/Training (risultati)/DBP/DBP - C - dx.pdf}
    \caption{Dependence of DBP on $C$ on the adjacent interval to the right of the training interval.}
    \label{DBP - C - dx}
\end{figure}


\newpage

% **********
% DBP - R1
% **********
\subsubsection{Dependence on $R_1$}
The overall result is shown in figure \ref{DBP - R1 - full}, the result in the training interval alone in \ref{DBP - R1 - training}, the result in the individual adjacent intervals in \ref{DBP - R1 - sx} and \ref{DBP - R1 - dx}. Again, the model is able to make very accurate predictions.

\vspace{1cm}

\begin{figure}[!htb]
    \centering
    \includegraphics[width=1\textwidth]{images/Training (risultati)/DBP/DBP - R1 - full.pdf}
    \caption{Dependence of DBP on $R_1$ on the training interval and two adjacent intervals.}
    \label{DBP - R1 - full}
\end{figure}

\vspace{0.32cm}

\begin{figure}[!htb]
    \centering
    \includegraphics[width=1\textwidth]{images/Training (risultati)/DBP/DBP - R1 - training.pdf}
    \caption{Dependence of DBP on $R_1$ over the training interval.}
    \label{DBP - R1 - training}
\end{figure}

\begin{figure}
    \centering
    \includegraphics[width=1\textwidth]{images/Training (risultati)/DBP/DBP - R1 - sx.pdf}
    \caption{Dependence of DBP on $R_1$ on the adjoint interval to the left of the training interval.}
    \label{DBP - R1 - sx}
\end{figure}



\begin{figure}
    \centering
    \includegraphics[width=1\textwidth]{images/Training (risultati)/DBP/DBP - R1 - dx.pdf}
    \caption{Dependence of DBP on $R_1$ on the adjoint interval to the right of the training interval.}
    \label{DBP - R1 - dx}
\end{figure}



\newpage
% **********
% DBP - R2
% **********
\subsubsection{Dependence on $R_2$}
The overall result is shown in figure \ref{DBP - R2 - full}, the result in the training interval alone in \ref{DBP - R2 - training}, the result in the individual adjacent intervals in \ref{DBP - R2 - sx} and \ref{DBP - R2 - dx}. Again there are very accurate predictions. At first glance it appears that the error bars do not appear; in fact they are very short compared to the scale used. In fact, in graphs where only the adjacent intervals are plotted the error bars reappear and show that the error is less than one unit.

\vspace{1cm}

\begin{figure}[!htb]
    \centering
    \includegraphics[width=1\textwidth]{images/Training (risultati)/DBP/DBP - R2 - full.pdf}
    \caption{Dependence of DBP on $R_2$ on the training interval and two adjacent intervals.}
    \label{DBP - R2 - full}
\end{figure}

\vspace{0.34cm}

\begin{figure}[!htb]
    \centering
    \includegraphics[width=1\textwidth]{images/Training (risultati)/DBP/DBP - R2 - training.pdf}
    \caption{Dependence of DBP on $R_2$ over the training interval.}
    \label{DBP - R2 - training}
\end{figure}

\begin{figure}
    \centering
    \includegraphics[width=1\textwidth]{images/Training (risultati)/DBP/DBP - R2 - sx.pdf}
    \caption{Dependence of DBP on $R_2$ on the adjoint interval to the left of the training interval.}
    \label{DBP - R2 - sx}
\end{figure}


\begin{figure}
    \centering
    \includegraphics[width=1\textwidth]{images/Training (risultati)/DBP/DBP - R2 - dx.pdf}
    \caption{Dependence of DBP on $R_2$ on the adjoint interval to the right of the training interval.}
    \label{DBP - R2 - dx}
\end{figure}







\newpage
%******************************
%*********** PP **************
%******************************
\subsection{Pulse pressure (PP)}
$\text{lr}=0.05$ is imposed and the early stopper \textit{GLEarlyStoppingCriterion} is used with parameters: $\alpha = 5$, $\text{patience}=3$.


% **********
% PP - loss
% **********
\subsubsection{Training and validation loss}
The training needed ninety-three EPOCHS, concluded with $\text{R2Score}=0.9994$, $\text{MeanSquaredError}=0.0006$. Figure \ref{PP - loss} shows the training and validation loss trend with MSE and R2Score; early stopper trend in green.
\begin{figure}[h]
    \centering
    \includegraphics[width=1\textwidth]{images/Training (risultati)/PP/PP - loss.png}
    \caption{PP: progress of training and validation loss, early stopper, R2Score and MSE.}
    \label{PP - loss}
\end{figure}

\vspace{-0.5cm}

% **********
% PP - inference
% **********
\subsubsection{Approximation of input data}
Figure \ref{PP - inference} shows how the predictions approximate the input data. The length of the error bars is $0.0034$.

\begin{figure}[h]
    \centering
    \includegraphics[width=1\textwidth]{images/Training (risultati)/PP/PP - inference.png}
    \caption{PP: predictions about the input data.}
    \label{PP - inference}
\end{figure}



% **********
% PP - C
% **********
\subsubsection{Dependence on $C$}
The overall result is shown in figure \ref{PP - C - full}, the result in the training interval alone in \ref{PP - C - training}, the result in the individual adjacent intervals in \ref{PP - C - sx} and \ref{PP - C - dx}.

\vspace{1cm}

\begin{figure}[!htb]
    \centering
    \includegraphics[width=1\textwidth]{images/Training (risultati)/PP/PP - C - full.pdf}
    \caption{Dependence of PP on $C$ on the training interval and two adjacent intervals.}
    \label{PP - C - full}
\end{figure}

\vspace{0.32cm}

\begin{figure}[!htb]
    \centering
    \includegraphics[width=1\textwidth]{images/Training (risultati)/PP/PP - C - training.pdf}
    \caption{Dependence of PP on $C$ over the training interval.}
    \label{PP - C - training}
\end{figure}

\begin{figure}
    \centering
    \includegraphics[width=1\textwidth]{images/Training (risultati)/PP/PP - C - sx.pdf}
    \caption{Dependence of PP on $C$ on the adjacent interval to the left of the training interval.}
    \label{PP - C - sx}
\end{figure}



\begin{figure}
    \centering
    \includegraphics[width=1\textwidth]{images/Training (risultati)/PP/PP - C - dx.pdf}
    \caption{Dependence of PP on $C$ on the adjacent interval to the right of the training interval.}
    \label{PP - C - dx}
\end{figure}




\newpage
% **********
% PP - R1
% **********
\subsubsection{Dependence on $R_1$}
The overall result is shown in figure \ref{PP - R1 - full}, the result in the training interval alone in \ref{PP - R1 - training}, the result in the individual adjacent intervals in \ref{PP - R1 - sx} and \ref{PP - R1 - dx}.

\vspace{1cm}

\begin{figure}[!htb]
    \centering
    \includegraphics[width=1\textwidth]{images/Training (risultati)/PP/PP - R1 - full.pdf}
    \caption{Dependence of PP on $R1$ on the training interval and two adjacent intervals.}
    \label{PP - R1 - full}
\end{figure}

\vspace{0.32cm}

\begin{figure}[!htb]
    \centering
    \includegraphics[width=1\textwidth]{images/Training (risultati)/PP/PP - R1 - training.pdf}
    \caption{Dependence of PP on $R1$ over the training interval.}
    \label{PP - R1 - training}
\end{figure}

\begin{figure}
    \centering
    \includegraphics[width=1\textwidth]{images/Training (risultati)/PP/PP - R1 - sx.pdf}
    \caption{Dependence of PP on $R1$ on the adjoint interval to the left of the training interval.}
    \label{PP - R1 - sx}
\end{figure}



\begin{figure}
    \centering
    \includegraphics[width=1\textwidth]{images/Training (risultati)/PP/PP - R1 - dx.pdf}
    \caption{Dependence of PP on $R1$ on the adjacent interval to the right of the training interval.}
    \label{PP - R1 - dx}
\end{figure}

\newpage


% **********
% PP - R2
% **********
\subsubsection{Dependence on $R_2$}
The overall result is shown in figure \ref{PP - R2 - full}, the result in the training interval alone in \ref{PP - R2 - training}, the result in the individual adjacent intervals in \ref{PP - R2 - sx} and \ref{PP - R2 - dx}.

\vspace{0.9cm}

\begin{figure}[!htb]
    \centering
    \includegraphics[width=1\textwidth]{images/Training (risultati)/PP/PP - R2 - full.pdf}
    \caption{Dependence of PP on $R2$ on the training interval and two adjacent intervals.}
    \label{PP - R2 - full}
\end{figure}

\vspace{0.32cm}

\begin{figure}[!htb]
    \centering
    \includegraphics[width=1\textwidth]{images/Training (risultati)/PP/PP - R2 - training.pdf}
    \caption{Dependence of PP on $R2$ over the training interval.}
    \label{PP - R2 - training}
\end{figure}

\begin{figure}
    \centering
    \includegraphics[width=1\textwidth]{images/Training (risultati)/PP/PP - R2 - sx.pdf}
    \caption{Dependence of PP on $R2$ on the adjacent interval to the left of the training interval.}
    \label{PP - R2 - sx}
\end{figure}



\begin{figure}
    \centering
    \includegraphics[width=1\textwidth]{images/Training (risultati)/PP/PP - R2 - dx.pdf}
    \caption{Dependence of PP on $R2$ on the adjoint interval to the right of the training interval.}
    \label{PP - R2 - dx}
\end{figure}







\newpage

%******************************
%*********** SBP **************
%******************************
\subsection{Systolic blood pressure (SBP)}
$\text{lr}=0.07$ is imposed and the early stopper \textit{GLEarlyStoppingCriterion} is used with parameters: $\alpha = 5$, $\text{patience}=1$.

% **********
% SBP - loss
% **********
\subsubsection{Training and validation loss}
The training needed one hundred and thirteen EPOCHS, concluded with $\text{R2Score}=0.9999$, $\text{MeanSquaredError}=0.0052$. Figure \ref{SBP - loss} shows the training and validation loss trend with MSE and R2Score; in green is the early stopper trend.
\begin{figure}[h]
    \centering
    \includegraphics[width=1\textwidth]{images/Training (risultati)/SBP/SBP - loss.png}
    \caption{SBP: andamento del training e validation loss, early stopper, R2Score e MSE.}
    \label{SBP - loss}
\end{figure}

\vspace{-0.5cm}

% **********
% SBP - inference
% **********
\subsubsection{Approximation of input data}
Figure \ref{SBP - inference} shows how the predictions approximate the input data. The error bars are $0.0012$ long, so they are not noticeable.

\begin{figure}[h]
    \centering
    \includegraphics[width=1\textwidth]{images/Training (risultati)/SBP/SBP - inference.png}
    \caption{SBP: predictions about the input data.}
    \label{SBP - inference}
\end{figure}



% **********
% SBP - C
% **********
\subsubsection{Dependence on $C$}
The overall result is shown in figure \ref{SBP - C - full}, the result in the training interval alone in \ref{SBP - C - training}, the result in the individual adjacent intervals in \ref{SBP - C - sx} and \ref{SBP - C - dx}.

\vspace{1cm}

\begin{figure}[!htb]
    \centering
    \includegraphics[width=1\textwidth]{images/Training (risultati)/SBP/SBP - C - full.pdf}
    \caption{Dependence of SBP on $C$ on the training interval and two adjacent intervals.}
    \label{SBP - C - full}
\end{figure}

\vspace{0.32cm}

\begin{figure}[!htb]
    \centering
    \includegraphics[width=1\textwidth]{images/Training (risultati)/SBP/SBP - C - training.pdf}
    \caption{Dependence of SBP on $C$ over the training interval.}
    \label{SBP - C - training}
\end{figure}

\begin{figure}
    \centering
    \includegraphics[width=1\textwidth]{images/Training (risultati)/SBP/SBP - C - sx.pdf}
    \caption{Dependence of SBP on $C$ on the adjacent interval to the left of the training interval.}
    \label{SBP - C - sx}
\end{figure}


\begin{figure}
    \centering
    \includegraphics[width=1\textwidth]{images/Training (risultati)/SBP/SBP - C - dx.pdf}
    \caption{Dependence of SBP on $C$ on the adjacent interval to the right of the training interval.}
    \label{SBP - C - dx}
\end{figure}



\newpage

% **********
% SBP - R1
% **********
\subsubsection{Dependence on $R_1$}
The overall result is shown in figure \ref{SBP - R1 - full}, the result in the training interval alone in \ref{SBP - R1 - training}, the result in the individual adjacent intervals in \ref{SBP - R1 - sx} and \ref{SBP - R1 - dx}.

\vspace{0.7cm}


\begin{figure}[!htb]
    \centering
    \includegraphics[width=1\textwidth]{images/Training (risultati)/SBP/SBP - R1 - full.pdf}
    \caption{Dependence of SBP on $R1$ on the training interval and two adjacent intervals.}
    \label{SBP - R1 - full}
\end{figure}

\vspace{0.32cm}

\begin{figure}[!htb]
    \centering
    \includegraphics[width=1\textwidth]{images/Training (risultati)/SBP/SBP - R1 - training.pdf}
    \caption{Dependence of SBP on $R1$ over the training interval.}
    \label{SBP - R1 - training}
\end{figure}

\begin{figure}
    \centering
    \includegraphics[width=1\textwidth]{images/Training (risultati)/SBP/SBP - R1 - sx.pdf}
    \caption{Dependence of SBP on $R1$ on the adjacent interval to the left of the training interval.}
    \label{SBP - R1 - sx}
\end{figure}



\begin{figure}
    \centering
    \includegraphics[width=1\textwidth]{images/Training (risultati)/SBP/SBP - R1 - dx.pdf}
    \caption{Dependence of SBP on $R1$ on the adjacent interval to the right of the training interval.}
    \label{SBP - R1 - dx}
\end{figure}






\newpage
% **********
% SBP - R2
% **********
\subsubsection{Dependence on $R_2$}
The overall result is shown in figure \ref{SBP - R2 - full}, the result in the training interval alone in \ref{SBP - R2 - training}, the result in the individual adjacent intervals in \ref{SBP - R2 - sx} and \ref{SBP - R2 - dx}.

\vspace{1cm}

\begin{figure}[!htb]
    \centering
    \includegraphics[width=1\textwidth]{images/Training (risultati)/SBP/SBP - R2 - full.pdf}
    \caption{Dependence of SBP on $R2$ on the training interval and two adjacent intervals.}
    \label{SBP - R2 - full}
\end{figure}

\vspace{0.32cm}

\begin{figure}[!htb]
    \centering
    \includegraphics[width=1\textwidth]{images/Training (risultati)/SBP/SBP - R2 - training.pdf}
    \caption{Dependence of SBP on $R2$ over the training interval.}
    \label{SBP - R2 - training}
\end{figure}

\begin{figure}
    \centering
    \includegraphics[width=1\textwidth]{images/Training (risultati)/SBP/SBP - R2 - sx.pdf}
    \caption{Dependence of SBP on $R2$ on the adjacent interval to the left of the training interval.}
    \label{SBP - R2 - sx}
\end{figure}



\begin{figure}
    \centering
    \includegraphics[width=1\textwidth]{images/Training (risultati)/SBP/SBP - R2 - dx.pdf}
    \caption{Dependence of SBP on $R2$ on the adjacent interval to the right of the training interval.}
    \label{SBP - R2 - dx}
\end{figure}

\section{Running time: approximation of MAP, DBP, SBP, PP}
As done in \ref{Windkessel: tempo esecuzione}, we now calculate the run time to find the value of MAP, DBP, SBP, PP from the values of $C, R_1, R_2$. Since one training is performed for each parameter, the parameter estimation times for each trained model are reported.

% Please add the following required packages to your document preamble:
% \usepackage{lscape}

\begin{table}[!htb]
\centering
\begin{tabular}{cc}
\hline
\textbf{Estimated parameter} & \textbf{\begin{tabular}[c]{@{}c@{}}Running time\\ (mean + dev. std.)\end{tabular}} \\ \hline
DBP & 5.89 ms ± 1.63 ms \\
SBP & 5.23 ms ± 1.75 ms \\
MAP & 5.6 ms ± 1.64 ms  \\
PP  & 4.03 ms ± 1.19 ms \\ \hline
\end{tabular}
\caption{Run time to find parameter approximation using trained Gaussian processes.}
\label{tab: tempo par}
\end{table}


Since an independent model is trained for each parameter, it is necessary to run the four trained Gaussian processes to obtain the MAP, DBP, SBP and PP values. So the running time needed to obtain the approximations of the four parameters is the sum of the times needed to obtain them one by one. So it can be said that in all, an average of 20.75 ms is required to obtain the four values.
Comparing this quantity to the table \ref{tab: tempo windkessel} it is evident that using Gaussian processes takes much less time than approximating the solution of the Windkessel model already after only ten cardiac cycles (a rather low number if the goal is to find the solution at convergence). \\


Evidently, then, Gaussian processes solve the run-time problem in the Windkessel model by providing a good approximation of the value of MAP, DBP, SBP, PP.

\chapter{Conclusions and future directions}
This chapter summarizes the main results obtained in the course of the work, and proposes \textit{future directions} that could improve what has been studied in both theoretical and practical terms.

\begin{textblock*}{0.64\textwidth}(3.5cm+0.36\textwidth,18.5cm)
\epigraph{A conclusion is the place where you get tired of thinking.}{Arthur Bloch}
\end{textblock*}

\newpage



\section{Conclusions}
It was seen throughout the elaborate what Gaussian processes are and, in particular, explained in what sense they generalize the multivariate Gaussian distribution. The main kernel functions were illustrated, explaining the meaning of the parameters of the functions (called in supervised learning \textit{hyperparameters}). It was then explained how to generalize covariance functions in multiple dimensions, and then apply this generalization to the context of supervised learning.

Then it was explained how to use Gaussian processes for the prediction of observations without noise (or interpolation) and with noise (\textit{noisy} observations), showing how simple in theoretical terms it is to obtain remarkable results.\\

Despite what has been seen, i.e., the simplicity and power of Gaussian processes, it is worth explaining why they are not widely used in the scientific landscape: only recently, in fact, there are some research groups applying them to hemodynamic contexts, such as \cite{doi:10.1098/rsta.2019.0334} and \cite{Yuhn2022.03.10.483573}. The main reasons that \cite{rasmussen_gaussian_2006} identifies to motivate the low scientific interest are two: the first is that the application of Gaussian processes requires the handling of large matrices and in particular the inversion of them, something that has become computationally addressable only in recent decades; the second is that most of the theoretical study has been done using the same covariance functions, with little awareness about them and thus without exploiting the power of this technology. Research like \cite{duvenaud_automatic_2014} and books like \cite{rasmussen_gaussian_2006} have certainly helped the scientific community in this regard. \\

The Windkessel model was then introduced in the paper, leaving extensive coverage of the application part. Although the model is rather simple, being formed only by a differential equation, it provides results of remarkable accuracy. Within the same chapter, the local sensitivity of the variables with respect to the parameters was studied, which allowed to understand how little influence the distal pressure $P_d$ has on MAP, DBP, SBP, PP; for this reason, it was discarded from the supervised learning parameters because, leaving it, one risked worsening the performance of the statistical model both in terms of learning time and accuracy.\\

Finally, the results of supervised learning performed with the GPErks library were reported after explaining its operation from a statistical point of view. These results allow us to conclude that indeed Gaussian processes provide a shortcut to the approximation of MAP, DBP, SBP and PP compared to the use of the Windkessel model. With the right parameters, supervised learning has been seen to provide an acceptable standard deviation, hence error,\footnote{The term \textit{acceptable} refers to the precision that can be accepted in a clinical setting where measurement errors of physiological parameters can be very high.} with fairly short training. Moreover, from the input parameters of the Windkessel model, the trained Gaussian process is able to return the value of MAP, SBP, DBP and PP much faster and with less computational cost than using the Windkessel model, then solving a differential equation to convergence. For these reasons, Gaussian processes have proven to be a tool with high potential and wide uses in applied mathematics.



\section{Future directions}
The in-depth and systematic study of Gaussian processes, the Bayesian theoretical basis on which supervised learning rests and the Windkessel model make it easy to identify, in retrospect, how the approach used can be improved.\\

For example, the squared-exponential kernel and a linear mean function were used in the paper. For the purposes of what was studied they worked perfectly, but the choice of kernel function and mean function deserves to be done with proper care. As seen in \ref{section: mauna loa}, the choice of composite covariance functions may require a lengthy study starting with the knowledge (which therefore needs to be thorough) of the event to be modeled while obtaining, however, a considerable increase in accuracy in the training phase.

The early stopper should be studied extensively and possibly modified\footnote{Early stoppers do not follow precise rules; there is scientific research that suggests their form and provides examples of their implementation. Optimal would be to build them specifically for each problem, but this can be very time-consuming} based on the specific problem to best avoid overfitting and to precisely impose the number of EPOCHS and thus the execution time required at the training phase. In clinical settings, for example, a low execution time at the expense of lower accuracy may be preferable to provide real-time support to the clinician. Indeed, it should be noted that in clinical settings it is unnecessary to require high accuracy from training results because patient measurements may be subject to significant measurement error.

The choice of optimization method must be made based on the problem being addressed. In fact, each method has advantages and disadvantages that must be carefully studied in order to optimize the training. Furthermore, based on the method, it is also necessary to study in depth the parameters that can be set, for example the learning rate, which can modify the speed of training and its performance.


Another improvement that can be made to training is to use real data to verify the model, allowing for more reliable feedback on its functioning. There are several ways to do this, a simple way is to use a validation set of real data, so that during the training phase the model, after each training phase on the training set, validates its functioning on real data. This improvement is not always possible, since sometimes there are no real datasets of the values that interest the training. For example, for the problem addressed in the paper, an adequate database of real data could not be found: the only possibility was to use a database generated by a more complex model than the Windkessel model.

One training approach that would significantly decrease the risk of overfitting is $K$-fold cross validation. This technique consists of dividing the entire dataset into $k$ subsamples of equal size. The training is executed $k$ times where each time one of the $k$ subsamples is used as validation test and the other $k-1$ as training set\footnote{It is possible to use one of the $k-1$ subsamples as testing set. In reality there are different forms of $k$-fold cross validation and they differ in the use of $k$ subsamples; the basic idea is to run multiple tests by partitioning the dataset.}. Then, $k$ results are obtained, which are averaged to generate a single estimate. Changing the validation set and the training set at each training lowers the probability of overfitting.

In a context like the one seen in the paper, the local sensitivity was sufficient to exclude $P_d$ from the parameters to be taken into consideration since it proved to have little influence on the values of MAP, DBP, SBP, PP (low influence also confirmed by the results of training). In much larger situations, with many more input parameters (a real problem can have hundreds and potentially even thousands of parameters), a more in-depth and non-local analysis is that of global sensitivity analysis, which allows us to understand, in a global, what are the parameters that influence the outputs. This can lead to a significant decrease in the number of parameters to consider, speeding up training and increasing performance.

Since independent training is done for each variable MAP, DBP, SBP, PP, starting from the same input parameters, approximations of MAP, DBP, SBP and PP are obtained which are not correlated with each other, in the sense that, probably, there will be $\text{PP}\neq \text{SBP} - \text{DBP}$. Generally this is not a particularly serious problem, remembering, as previously mentioned, that in clinical situations there are large measurement errors. However, simultaneous training of all variables would avoid similar problems, even though it is a rather complicated training technique.









%%%%%%%%%%%%%%%%%%%%%%%%%%%%%%%%%%
%%%%%%%%%%%%%%%%%%%%%%%%%%%%%%%%%
%%%%%%%%%%%%%%%%%%%%%%%%%%%%%%%%%%%
%Capitolo temporaneo
%\chapter{Training}

Seguono le immagini dei risultati. Osservazioni:
\begin{itemize}
    \item Il numero di EPOCHS di alcuni risultati è dettata da fattori "arbitrari". Ad esempio il learning rate di alcuni training è stato settato molto piccolo, dunque il tempo di training è più lungo. Questo è stato fatto principalmente per evitare un bug (che ho già accennato a Christian e che anche lui ha già avuto...) di matrici che NON dipende da me...
    \item Non conosco precisamente il significato del lr (learning rate) nello specifico nell'optimizer usato (Adam)
    \item Non conosco precisamente come scegliere l'optimizer. Quello impostato è di default; è possibile cambiarlo e ho intenzione di valutare quale (tra quelli già in gpytorch...) è da preferire
    \item Non conosco precisamente il funzionamento dell'EarlyStopper, programmato da Stefano Longobardi. Non conosco precisamente il funzionamento di alpha (parametro per l'EarlyStopper) né il funzionamento di patience (altro parametro di EarlyStopper)
    \item Non ho implementato la ripetizione del training per velocizzare la generazione delle immagini e inserirle con facilità nel capitolo "temporaneo". Ovviamente in sede di "bella copia" le implementerò, cosa che permetterà all'emulatore di selezionare il "migliore" training.
    \item Lo strano comportamento del grafici MAP - Pd, SBP - Pd, DBP - Pd, PP - Pd (in realtà abbastanza preciso guardando la scala...) non so bene come giustificarlo... Sono abbastanza sicuro che dipenda dalla bassa sensitività locale, cioè dal fatto che infici molto poco nell'output.
    \item Si noti che i risultati si possono migliorare aumentando la mole del dataset, tenuto abbastanza piccolo per rimanere in tempi veloci per facilitare la stesura di questo capitolo temporaneo... Potrebbe essere oggetto di studio anche la dimensione del dataset per ottenere risultati precisi (imponendosi una tolleranza), ma credo abbia senso farla dopo lo studio delle singole componenti di ML usate dall'API 
    \item Ci sono early stopper che "non hanno funzionato", cioè il training ha finito al numero massimo di EPOCHS impostato. Questo dipende da una componente randomica nello starting del training: a volte un certo early stopper funziona in un certo modo (termina ad una certa EPOCH) altre volte in un altro modo; questo complica la scelta dei parametri dell'early stopper
\end{itemize}

%%%%%%%%%%%%%%%%%%%%%%%%%%%%%
%%%%%%%%%%%%%%%%%%%%%%%%%%%%%
%             MAP
%%%%%%%%%%%%%%%%%%%%5%%%%%%%%%
%%%%%%%%%%%%%%%%%%%%%%%%%%%%
\newpage

\section{MAP}
\begin{figure}[h]
    \centering
    \includegraphics[width=1\textwidth]{images/Training - temp/MAP - dataset.png}
    \caption{Dataset (lo ometto successivamente perché uguale)}
\end{figure}

\newpage

\begin{figure}[h]
    \centering
    \includegraphics[width=1\textwidth]{images/Training - temp/MAP - parametri.png}
    \caption{Plot: MAP vs single parameter. Sotto vengono riproposti i singoli grafici fatti meglio.}
\end{figure}

\newpage

\begin{figure}[h]
    \centering
    \includegraphics[width=1\textwidth]{images/Training - temp/MAP - loss.png}
    \caption{Loss, no early stopper, 100 EPOCHS}
\end{figure}

\newpage

\begin{figure}[h]
    \centering
    \includegraphics[width=1\textwidth]{images/Training - temp/MAP - inference.png}
    \caption{Inference, no early stopper, 100 EPOCHS}
\end{figure}

\newpage

\begin{figure}[h]
    \centering
    \includegraphics[width=1\textwidth]{images/Training - temp/MAP - loss - early stopper.png}
    \caption{Loss, early stopper. Ho riscontrato il bug per cui ho dovuto diminuire di molto il lr, dunque ha richiesto molti più EPOCH di quanto (se non avviene tale bug) si possa raggiungere.}
\end{figure}

\newpage

\begin{figure}[h]
    \centering
    \includegraphics[width=1\textwidth]{images/Training - temp/MAP - inference - early stopper.png}
    \caption{Inference, early stopper.}
\end{figure}

\newpage
Seguono quattro grafici: MAP - C, R1, R2, Pd. Per farli ho generato un dataset (diverso da quello per il training) con solo un parametro in input che variasse, mentre gli altri restavano fissi. (Quindi per MAP - C ho generato un dataset in cui R1, R2 e PD sono fissi e C varia) Ho predetto per ogni entry del dataset la MAP e l'ho controntata con quella che calcolo col Windkessel.


\begin{figure}[h]
    \centering
    \includegraphics[width=1\textwidth]{images/Training - temp/MAP - C.png}
    \caption{MAP - C}
\end{figure}

\newpage

\begin{figure}[h]
    \centering
    \includegraphics[width=1\textwidth]{images/Training - temp/MAP - R2.png}
    \caption{MAP - R2}
\end{figure}

\newpage

\begin{figure}[h]
    \centering
    \includegraphics[width=1\textwidth]{images/Training - temp/MAP _ R1.png}
    \caption{MAP - R1}
\end{figure}

\newpage


\begin{figure}[h]
    \centering
    \includegraphics[width=1\textwidth]{images/Training - temp/MAP - Pd.png}
    \caption{MAP - Pd}
\end{figure}

%%%%%%%%%%%%%%%%%%%%%%%%%%%%%
%%%%%%%%%%%%%%%%%%%%%%%%%%%%%
%             DBP
%%%%%%%%%%%%%%%%%%%%5%%%%%%%%%
%%%%%%%%%%%%%%%%%%%%%%%%%%%%
\newpage
\section{DBP}

\begin{figure}[h]
    \centering
    \includegraphics[width=1\textwidth]{images/Training - temp/DBP - parameters.png}
    \caption{DBP - parameters (meglio sotto)}
\end{figure}

\newpage


\begin{figure}[h]
    \centering
    \includegraphics[width=1\textwidth]{images/Training - temp/DBP - loss.png}
    \caption{Loss, no early stopper, 100 EPOCHS}
\end{figure}


\newpage


\begin{figure}[h]
    \centering
    \includegraphics[width=1\textwidth]{images/Training - temp/DBP - inference.png}
    \caption{Inference, no early stopper, 100 EPOCHS}
\end{figure}



\newpage


\begin{figure}[h]
    \centering
    \includegraphics[width=1\textwidth]{images/Training - temp/DBP - loss - early stopper.png}
    \caption{Loss, early stopper (non ho riscontrato bug, l'early stopper ha funzionato bene in questo caso. Si noti che è stato cambiato anche il lr, cosa che, in combinazione con l'early stopper, ha ridotto notevolmente il numero di EPOCHS.)}
\end{figure}


\newpage


\begin{figure}[h]
    \centering
    \includegraphics[width=1\textwidth]{images/Training - temp/DBP - inference - early stopper.png}
    \caption{Inference, early stopper}
\end{figure}


\newpage
Non sono sicuro se ciò che segue sia colpa mia (codice sbagliato) oppure no. In parte ha senso: il RBF kernel genera funzioni $C^\infty$; mi aspetto che con un altro kernel si raggiunga una maggiore precisione. Ad esempio per DBP - C un linear kernel potrebbe forse fare di meglio...

\begin{figure}[h]
    \centering
    \includegraphics[width=1\textwidth]{images/Training - temp/DBP - C.png}
    \caption{DBP - C}
\end{figure}

\newpage


\begin{figure}[h]
    \centering
    \includegraphics[width=1\textwidth]{images/Training - temp/DBP - R1.png}
    \caption{DBP - R1}
\end{figure}

\newpage


\begin{figure}[h]
    \centering
    \includegraphics[width=1\textwidth]{images/Training - temp/DBP - R2.png}
    \caption{DBP - R2}
\end{figure}

\newpage


\begin{figure}[h]
    \centering
    \includegraphics[width=1\textwidth]{images/Training - temp/DBP - Pd.png}
    \caption{DBP - Pd}
\end{figure}






%%%%%%%%%%%%%%%%%%%%%%%%%%%%%
%%%%%%%%%%%%%%%%%%%%%%%%%%%%%
%             SBP
%%%%%%%%%%%%%%%%%%%%5%%%%%%%%%
%%%%%%%%%%%%%%%%%%%%%%%%%%%%
\newpage
\section{SBP}
\begin{figure}[h]
    \centering
    \includegraphics[width=1\textwidth]{images/Training - temp/SBP - parameters.png}
    \caption{SBP - parameters (meglio sotto)}
\end{figure}

\newpage

\begin{figure}[h]
    \centering
    \includegraphics[width=1\textwidth]{images/Training - temp/SBP - loss.png}
    \caption{Loss, no early stopper, 100 EPOCHS}
\end{figure}


\newpage

\begin{figure}[h]
    \centering
    \includegraphics[width=1\textwidth]{images/Training - temp/SBP - inference.png}
    \caption{Inference, no early stopper, 100 EPOCHS}
\end{figure}


\newpage

\begin{figure}[h]
    \centering
    \includegraphics[width=1\textwidth]{images/Training - temp/SBP - loss - early stopper.png}
    \caption{Loss, early stopper (Early stopper ha funzionato, penso si possa fare di meglio conoscendone bene il funzionamento...)}
\end{figure}


\newpage

\begin{figure}[h]
    \centering
    \includegraphics[width=1\textwidth]{images/Training - temp/SBP - inference - early stopper.png}
    \caption{Inference, early stopper}
\end{figure}

\newpage

\begin{figure}[h]
    \centering
    \includegraphics[width=1\textwidth]{images/Training - temp/SBP - C.png}
    \caption{SBP - C}
\end{figure}

\newpage

\begin{figure}[h]
    \centering
    \includegraphics[width=1\textwidth]{images/Training - temp/SBP - R1.png}
    \caption{SBP - R1}
\end{figure}


\newpage

\begin{figure}[h]
    \centering
    \includegraphics[width=1\textwidth]{images/Training - temp/SBP - R2.png}
    \caption{SBP - R2}
\end{figure}


\newpage

\begin{figure}[h]
    \centering
    \includegraphics[width=1\textwidth]{images/Training - temp/SBP - Pd.png}
    \caption{SBP - Pd}
\end{figure}










%%%%%%%%%%%%%%%%%%%%%%%%%%%%%
%%%%%%%%%%%%%%%%%%%%%%%%%%%%%
%             PP
%%%%%%%%%%%%%%%%%%%%5%%%%%%%%%
%%%%%%%%%%%%%%%%%%%%%%%%%%%%
\newpage
\section{PP}

\begin{figure}[h]
    \centering
    \includegraphics[width=1\textwidth]{images/Training - temp/PP - parametri.png}
    \caption{PP - parametri (meglio sotto)}
\end{figure}


\newpage


\begin{figure}[h]
    \centering
    \includegraphics[width=1\textwidth]{images/Training - temp/PP - loss.png}
    \caption{Loss, no early stopper, 100 EPOCHS}
\end{figure}



\newpage
\begin{figure}[h]
    \centering
    \includegraphics[width=1\textwidth]{images/Training - temp/PP - inference.png}
    \caption{Inference, no early stopper, 100 EPOCHS}
\end{figure}


\newpage
\begin{figure}[h]
    \centering
    \includegraphics[width=1\textwidth]{images/Training - temp/PP - loss - early inference.png}
    \caption{Loss, early stopper (l'early stopper non ha funzionato e ha finito al massimo delle EPOCHS impostate)}
\end{figure}



\newpage
\begin{figure}[h]
    \centering
    \includegraphics[width=1\textwidth]{images/Training - temp/PP - inference - early stopper.png}
    \caption{Inference, early stopper}
\end{figure}


\newpage
\begin{figure}[h]
    \centering
    \includegraphics[width=1\textwidth]{images/Training - temp/PP - C.png}
    \caption{PP - C}
\end{figure}


\newpage
\begin{figure}[h]
    \centering
    \includegraphics[width=1\textwidth]{images/Training - temp/PP - R1.png}
    \caption{PP - R1}
\end{figure}

\newpage
\begin{figure}[h]
    \centering
    \includegraphics[width=1\textwidth]{images/Training - temp/PP - R2.png}
    \caption{PP - R2}
\end{figure}

\newpage
\begin{figure}[h]
    \centering
    \includegraphics[width=1\textwidth]{images/Training - temp/PP - Pd.png}
    \caption{PP - Pd}
\end{figure}


















\newpage
Possibili future directions:
\begin{itemize}
    \item Comprendere e scegliere bene l'EarlyStopper. Dovrebbero essercene anche di già implementati (preferibili, a mio avviso) e esplorerò questo mondo
    \item Comprendere e scegliere il preferito optimizer (quello che ora è Adam) e relativi parametri
    \item Usare un external validation set potrebbe aumentare la precisione. Se avessimo un dataset "reale" (diversi pazienti veri) si potrebbe forse addirittura ottenere dei risultati che superino la precisione del modello Windkessel (not sure...). Mi rendo conto richieda del tempo però e non so se esistano database simili...
    \item Scegliere altre (migliori?) loss function case-specific
    \item Rifare il training senza Pd in input e confrontare i risultati (oppure decidere proprio di toglierla...)
    \item Provare a settare il dataset in modo da migliorare i risultati (variabile) - (parametro), magari impostando come validation set combinazioni di parametri realistiche, fingendo quindi che siano misurati su pazienti; per il training e testing usare combinazioni di dati più estreme e meno realistiche. Non so se possa funzionare...
\end{itemize}

Queste direzioni future peraltro rientrano nei possibili interessi di Christian e potrebbero essere "facilmente" adattate al suo modello. (mi riferisco soprattutto alla scelta dell'optimizer, del lr, dell'EarlyStopper, della scelta del Kernel)\\

Per ora il K-fold cross validation mi sembra "secondario": prima sistemerei i problemi / dubbi su questo tipo di training, poi mi sposterei sulle K-fold cross validation... 


%\input{Capitoli/Capitoli di esempio/capitolo: immagine}
%\input{Capitoli/Capitoli di esempio/capitolo: codice}
%%%%%%%%%%%%%%%%%%%%%%%%%%%%%%%%%%
%%%%%%%%%%%%%%%%%%%%%%%%%%%%%%%%%
%%%%%%%%%%%%%%%%%%%%%%%%%%%%%%%%%%%



%%%%%%%%%%%%%%%%%%%%%%%%%%%%%%%
%% Backmatter       
%%%%%%%%%%%%%%%%%%%%%%%%%%%%%%%
\backmatter

\afterpage{\blankpage}


%%%%%%%%%%%%%%%%%%%%%%%%%%%%%%%
%%%%%%%%% Intestazione per la bibliografia
%%%%%%%%%%%%%%%%%%%%%%%%%%%%%%%
\fancyhf{}
\fancyhead[ROH]{\textbf{\thepage}}
\fancyhead[LEH]{\textbf{\thepage}}
\fancyhead[REH]{Bibliografia}
\fancyhead[LOH]{Bibliografia}

%\renewcommand{\headrulewidth}{1pt}
\renewcommand{\footrulewidth}{0pt}

%% BIBLIOGRAFIA
\printbibliography[heading=bibintoc]





%%%%%%%%%%%%%%%%%%%%%%%%%
% INTESTAZIONE APPENDICE
%%%%%%%%%%%%%%%%
\fancyhf{}
\fancyhead[ROH]{\textbf{\thepage}}
\fancyhead[LEH]{\textbf{\thepage}}
\fancyhead[REH]{Appendice}
\fancyhead[LOH]{Appendice}

\chapter*{Appendice}
\addcontentsline{toc}{chapter}{Appendice}




In questo ultimo capitolo vengono inseriti tutti i codici non precedentemente menzionati per la generazione di immagini e per poter lavorare con il modello Windkessel come proposto nel capitolo \ref{windkessel}.\\
Tutti i codici sono in Python, per la loro creazione sono stati usati: Anaconda (versione 2021.11), conda (versione 4.12.0),  python (versione 3.9.7).\\
Non vengono inseriti i codici completi per la creazione del training di processi gaussiani perché sono una debole modifica dei codici del dottor Stefano Longobardi, presenti su GitHub alla pagina "\href{https://github.com/stelong/GPErks}{GPErks}".



\begin{textblock*}{0.64\textwidth}(3.5cm+0.36\textwidth,18.5cm)
   \epigraph{Appendix usually means "\textit{small outgrowth from large intestine}", but in this case it means "\textit{additional information accompanying main text}". Or are those really the same things? Think carefully before you insult this book.}{Pseudonymous Bosch}
\end{textblock*}


\newpage

%%%%%%%%%%%%%%%%%%%%%%%%%%%%%%%%%%%%%%%%
%%%%%  PROCESSI GAUSSIANI
%%%%%%%%%%%%%%%%%%%%%%%%%%%%%%%%%%%%%%%%
\section*{Processi gaussiani}

\subsection*{Import e codici preliminari}

%%% Import
\lstinputlisting[language=Python, caption = Import necessario per la generazione di immagini dei kernel introdotti nel capitolo \ref{gaussianProcessChapter}, firstnumber=1, stepnumber=1, label={import}]{codes/Processi gaussiani/Import.txt}

%%% Import
\lstinputlisting[language=Python, caption = Import necessario per  la generazione di immagini sulla predizione del capitolo \ref{gaussianProcessChapter}, firstnumber=1, stepnumber=1, label={import2}]{codes/Processi gaussiani/Import2.txt}

%%% Cubed mean
\lstinputlisting[language=Python, caption = Definizione della media cubica, firstnumber=1, stepnumber=1, label={cubedMean}]{codes/Processi gaussiani/cubedMean.txt}

\newpage

%%% Esempio di processo gaussiano
\lstinputlisting[language=Python, caption=Codice per generare la figura \ref{esempioProcessoGaussianoImmagine}, firstnumber=1, stepnumber=1, label={Example}]{codes/Processi gaussiani/Example.txt}



\newpage




\subsection*{Linear kernel}

%%% Plot kernel
\lstinputlisting[language=Python, label={linear kernel code}, caption={Codice per generare la figura \ref{linear kernel}} ,firstnumber=1, stepnumber=1]{codes/Processi gaussiani/Linear kernel.txt}

%%% Sample
\lstinputlisting[language=Python, caption={Codice per generare la figura \ref{10 sample linear kernel zero mean}} ,firstnumber=1, stepnumber=1, label={linear sample}]{codes/Processi gaussiani/Linear sample.txt}

%%% Vario c
\lstinputlisting[language=Python, caption={Codice per generare la figura \ref{10 sample linear modified c}} ,firstnumber=1, stepnumber=1, label={Linear - c}]{codes/Processi gaussiani/Linear - c.txt}


%%% Vario sigmab
\lstinputlisting[language=Python, caption={Codice per generare la figura \ref{10 sample linear modified sigmab}} ,firstnumber=1, stepnumber=1, label={Linear - sigmab}]{codes/Processi gaussiani/Linear - sigmab.txt}


%%% Vario sigmav
\lstinputlisting[language=Python, caption={Codice per generare la figura \ref{10 sample linear modified sigmav}} ,firstnumber=1, stepnumber=1, label={Linear - sigmav}]{codes/Processi gaussiani/Linear - sigmav.txt}


%%% cubedmean
\lstinputlisting[language=Python, caption={Codice per generare la figura \ref{10 sample linear kernel cubed mean}} ,firstnumber=1, stepnumber=1, label={linear cubedmean}]{codes/Processi gaussiani/Linear - cubedmean.txt}







\subsection*{Squared-exponential kernel}

%%% Plot kernel
\lstinputlisting[language=Python, caption={Codice per generare la figura \ref{squared-exponential kernel}} ,firstnumber=1, stepnumber=1, label={squared-exponential}]{codes/Processi gaussiani/RBF kernel.txt}

%%% Sample
\lstinputlisting[language=Python, caption={Codice per generare la figura \ref{10 sample exponential kerne zero mean}} ,firstnumber=1, stepnumber=1, label={RBF sample}]{codes/Processi gaussiani/RBF sample.txt}

\newpage

%%% Vario sigma
\lstinputlisting[language=Python, caption={Codice per generare la figura \ref{10 sample exponential modified sigma}} ,firstnumber=1, stepnumber=1, label={RBF - sigma}]{codes/Processi gaussiani/RBF - sigma.txt}


%%% Vario l
\lstinputlisting[language=Python, caption={Codice per generare la figura \ref{10 sample exponential modified l}} ,firstnumber=1, stepnumber=1, label={codice9}]{codes/Processi gaussiani/RBF - l.txt}

\newpage

%%% cubedmean
\lstinputlisting[language=Python, caption={Codice per generare la figura \ref{10 sample exponential kernel cubed mean}} ,firstnumber=1, stepnumber=1, label={codice10}]{codes/Processi gaussiani/RBF - cubedMean.txt}




\newpage

\subsection*{Periodic kernel}


%%% kernel plot
\lstinputlisting[language=Python, caption={Codice per generare la figura \ref{periodic kernel}} ,firstnumber=1, stepnumber=1, label={periodic Kernel}]{codes/Processi gaussiani/Periodic kernel.txt}

%%% sample
\lstinputlisting[language=Python, caption={Codice per generare la figura \ref{3 sample periodic kerne zero mean}} ,firstnumber=1, stepnumber=1, label={periodic sample}]{codes/Processi gaussiani/Periodic sample.txt}

%%% vario sigma
\lstinputlisting[language=Python, caption={Codice per generare la figura \ref{10 sample periodic modified sigma}} ,firstnumber=1, stepnumber=1, label={Periodic sigma}]{codes/Processi gaussiani/Periodic sigma.txt}

%%% vario p
\lstinputlisting[language=Python, caption={Codice per generare la figura \ref{10 sample periodic modified p}} ,firstnumber=1, stepnumber=1, label={periodic p}]{codes/Processi gaussiani/Periodic p.txt}

%%% vario l
\lstinputlisting[language=Python, caption={Codice per generare la figura \ref{10 sample periodic modified l}} ,firstnumber=1, stepnumber=1, label={periodic l}]{codes/Processi gaussiani/Periodic l.txt}

%%% cubed mean
\lstinputlisting[language=Python, caption={Codice per generare la figura \ref{3 sample periodic kernel cubed mean}} ,firstnumber=1, stepnumber=1, label={priodic cubedmean}]{codes/Processi gaussiani/Periodic - cubedmean.txt}




\subsection*{Composizione di kernel}

%%% RBF + periodic kernel
\lstinputlisting[language=Python, caption={Codice per generare la figura \ref{SE + periodic kernel}} ,firstnumber=1, stepnumber=1, label={RBF + periodic kernel}]{codes/Processi gaussiani/RBF + periodic kernel.txt}

%%% RBF + periodic sample
\lstinputlisting[language=Python, caption={Codice per generare la figura \ref{SE + periodic sample}} ,firstnumber=1, stepnumber=1, label={RBF + periodic sample}]{codes/Processi gaussiani/RBF + periodic sample.txt}


%%% Linear x linear kernel
\lstinputlisting[language=Python, caption={Codice per generare la figura \ref{linear * linear kernel}} ,firstnumber=1, stepnumber=1, label={linear x linear}]{codes/Processi gaussiani/Linear x linear.txt}

%%% Linear x linear sample
\lstinputlisting[language=Python, caption={Codice per generare la figura \ref{linear * linear sample}} ,firstnumber=1, stepnumber=1, label={linear x linear sample}]{codes/Processi gaussiani/Linear x linear sample.txt}




\newpage




\subsection*{Predizione coi processi gaussiani}

%% CODICE 21
\lstinputlisting[caption={Codice per generare la figura \ref{Interpolation}},language=Python, columns=fullflexible, stepnumber=1, firstnumber=1, label={interpolation code}]{codes/Processi gaussiani/Interpolazione.txt}


\newpage

%% CODICE 21
\lstinputlisting[caption={Codice per generare la figura \ref{Interpolation confidence region}},language=Python, stepnumber=1, firstnumber=1, columns=fullflexible, label={interpolation confidence region code}]{codes/Processi gaussiani/InterpolazioneConfidenceRegion.txt}


\newpage

%% CODICE 21
\lstinputlisting[caption={Codice per generare la figura \ref{Noisy}},language=Python, stepnumber=1, firstnumber=1, columns=fullflexible, label={Noise code}]{codes/Processi gaussiani/InterpolazioneConfidenceRegion.txt}


\newpage

%% CODICE 21
\lstinputlisting[caption={Codice per generare la figura \ref{Noisy confidence region}},language=Python, stepnumber=1, firstnumber=1, columns=fullflexible, label={Noise confidence region code}]{codes/Processi gaussiani/NoiseConfidenceRegion.txt}


\newpage

%%%%%%%%%%%%%%%%%%%%%%%%%%%%%%%%%%%%%%%%
%%%%%  WINDKESSEL
%%%%%%%%%%%%%%%%%%%%%%%%%%%%%%%%%%%%%%%%
\section*{Modello Windkessel}
\begin{lstlisting}[language=Python,caption={Codice per importare le librerie necessarie}\label{configurazione1}, columns=fullflexible,firstnumber=1, stepnumber=1]
    import numpy as np
    import scipy as sp
    import matplotlib.pyplot as plt
    import matplotlib.pyplot as plt
    from scipy.integrate import solve_ivp
    from scipy.optimize import minimize_scalar
\end{lstlisting}

\begin{lstlisting}[language=Python,caption={Code for plotting the actual data}\label{datiReali}, columns=fullflexible,firstnumber=1, stepnumber=1]
    # Pressione e flusso dai file
    flow = np.genfromtxt('stergioFlow.dat')
    pressure = np.genfromtxt('stergioPressure.dat')

    # Time grid
    M = 1000
    # array tempo
    time = np.linspace(0.,1.,M)

    # Interpolo i dati nella grid
    flow = np.interp(time,flow[:,0],flow[:,1])
    pressure = np.interp(time,pressure[:,0],pressure[:,1])

    # Plot
    fig,ax=plt.subplots()
    ln1 = ax.plot(time,flow,'b-',label='$Q_{in}$')
    ax.set_xlabel("time [s]")
    ax.set_ylabel("$Q_{in}\,[mL/s]$")
    ax2=ax.twinx()
    ln2 = ax2.plot(time,pressure,'r-',label='P')
    ax2.set_ylabel("$P\,[mmHg]$")
    lns = ln1+ln2
    labs = [l.get_label() for l in lns]
    ax.legend(lns, labs, loc=0)
\end{lstlisting}

\begin{lstlisting}[language=Python,caption={Code for calculating total peripheral resistance}\label{resistenzatotale}, columns=fullflexible,firstnumber=1, stepnumber=1]
    rd = np.average(pressure)/np.average(flow)
    print("Pressione media - %.3f mmHg" % np.average(pressure))
    print("Flusso medio - %.3f mL/s" % np.average(flow))
    print("Resistenza - %.3f mmHg/mL*s" % rd)
\end{lstlisting}

\begin{lstlisting}[language=Python,caption={Codice per la definizione della ODE}\label{ODE}, columns=fullflexible,firstnumber=1, stepnumber=1]
    # Definisco dP/dt
    def dpdt(t,p,args):
        c,r2,qFunc,pd = args
        qin = qFunc(t)
        return (qin-(p-pd)/r2)/c
\end{lstlisting}

\newpage

\begin{lstlisting}[language=Python,caption={Codice per definire la funzione di flusso}\label{flusso}, columns=fullflexible,firstnumber=1, stepnumber=1]
    def qFull(t):
        # Conosco i valori del tempo e del flusso
        q = np.interp(t,time,flow)
        return q
\end{lstlisting}

\begin{lstlisting}[language=Python,caption={Code for the definition of the function $f_C$}\label{fC}, columns=fullflexible,firstnumber=1, stepnumber=1]
    def funC(c,args):
        """
        Objective function for CR Windkessel with compliance
        C unknown
        Input arguments:
        - c: current value for compliance
        - args: list     containing:
            1) time array where data is available
            2) pressure array where data is available (associated to time)
            3) peripehral resistance
            4) flow function, that is called qFunc(t)
            5) ODE function, that is called dydt(t,y,args)
        Output argument:
        - errnorm: the value of objective function (7) for "c"
        """
    
        time, pressure, r, qFunc, dydt= args
    
        argsIVP = [[c,r, qFunc, pd]]
    
        # ODE
        fun = dydt
    
        # Tempo iniziale e finale
        t_span = [time[0],time[-1]]
    
        # Condizione iniziale (lista di un elemento)
        y0=[pressure[0]]
    
        # Metodo (stringa con il nome)
        method='RK45'
   
        # Punti dove valutare la soluzione
        t_eval = time

        # Tolleranza per il metodo numerico
        tol = 1e-6
    
        # solve_ivp
        sol = solve_ivp(fun=fun, t_span=t_span,y0=y0,method=method, t_eval=t_eval,args=argsIVP,rtol=tol)
    
        # Vettore addendo di f_C
        errnorm = (sol.y[0,:] - pressure)**2
    
        # f_C
        errnorm = sum(errnorm)
        return errnorm
\end{lstlisting}

\newpage

\begin{lstlisting}[language=Python,caption={Plot soluzione del modello "semplice"}\label{modelloSemplice}, columns=fullflexible,firstnumber=1, stepnumber=1]
    pPred = flow*rd
    plt.plot(time,pressure,label='Data')
    plt.plot(time,pPred,label='Resistance model')
    plt.xlabel("time [s]")
    plt.ylabel("$P\,[mmHg]$")
    plt.legend()
\end{lstlisting}

\begin{lstlisting}[language=Python,caption={Codice per generare la figura \ref{plotfC}.}\label{plotfC-code}, columns=fullflexible,firstnumber=1, stepnumber=1]
    # Considero i valori di C da 0.6 a 10 per minimizzare la funzione
    M = 100
    C = np.linspace(0.6,10,M)

    function = []
    for i in range(M):
        args = [time, pressure, rd, qFull, dpdt]
        # Valuto la funzione in quel C[i]
        err = funC(C[i],args=args)
        function.append(err)

    # Plot dei valori della f_C
    plt.plot(C,function,'b--',label='$f_C$')
    plt.legend()
    plt.ylabel("$f(C)\,[mmHg]$")
    plt.xlabel("$C\,[mL/mmHg]$")
\end{lstlisting}

\begin{lstlisting}[language=Python,caption={Code for estimating $C$ that minimizes $f_C$}\label{stimaC}, columns=fullflexible,firstnumber=1, stepnumber=1]
    args = [time, pressure, rd, qFull, dpdt]
    
    # Trovo il valore di C che minimizza f_C
    bracket = [0.6,10.]
    # Funzione per minimizzare
    res = minimize_scalar(funC,bracket=bracket, args=args)
    
    # Non funziona sempre
    if res.success:
        C = res.x
        print("C: %.5f" % C)
    else:
        print("ATTENTION: did not succeed in finding C!")
\end{lstlisting}

\newpage

\begin{lstlisting}[language=Python,caption={Codice per generare la figura \ref{soluzioneCapprossimata}}\label{plotSoluzioneCstimata}, columns=fullflexible,firstnumber=1, stepnumber=1]
    # Imposto i valori
    args = [[C, rd, qFull, pd]]
    fun = dpdt
    t_span = [time[0],time[-1]]
    y0 = [p0]
    method='RK45'
    t_eval = time

    # Uso solve_ivp per risolvere ODE
    sol = solve_ivp(fun=fun, t_span=t_span,y0=y0,method=method,
    t_eval=t_eval,args=args,rtol=tol)
    pPred = sol.y[0,:]

    err = pressure-pPred
    print("Error inf-norm is: %.6f" % np.linalg.norm(err,np.inf))
    print("Error 2-norm is: %.6f" % np.linalg.norm(err,2))
    plt.plot(time,pressure,'g-',label='Data')
    plt.plot(time[::10],pPred[::10],'rx',label='Model')
    plt.xlabel("time [s]")
    plt.ylabel("$P\,[mmHg]$")
    plt.legend()
\end{lstlisting}

\begin{lstlisting}[language=Python,caption={Codice per la definizione della funzione  $f_{C,\alpha}$}\label{fCa}, columns=fullflexible,firstnumber=1, stepnumber=1]
    def funCR(x,args):
    """
    Objective function for RCR Windkessel with compliance
    C and alpha unknown. Input arguments:
    - x: list with current value for compliance and alpha
    - args: list containing:
        1) time array where data is available
        2) pressure array where data is available (associated to time)
        3) peripehral resistance
        4) flow function, that is called qFunc(t)
        5) ODE function, that is called dydt(t,y,args)
    Output argument:
    - errnorm: the value of objective function (7) for "c"
    """
    c, alpha = x
    time, pressure, r, qFunc, dydt = args
    
    # Definisco "r1" e "r2" usando "r" e "alpha"
    r1= (1-alpha)*r
    r2 = alpha*r
    
    # *****Risolvo IVP*****
    # Condizionale iniziale p(t0)=pressure[0]
    args = [[c,r2,qFunc, pd]]
    fun = dydt
    t_span = [time[0],time[-1]]
    y0 = [pressure[0]]
    method='RK45'
    t_eval = time
    sol = solve_ivp(fun=fun, t_span=t_span,y0=y0,method=method, t_eval=t_eval,args=args,rtol=tol)
    
    # Valuto f_C,alpha
    pin = sol.y[0,:] + r1*flow
    errnorm = (pin - pressure)**2
    errnorm = sum(errnorm)
    return errnorm
\end{lstlisting}

\begin{lstlisting}[language=Python,caption={Codice per la stima di $C$ e $\alpha$}\label{stimaCA}, columns=fullflexible,firstnumber=1, stepnumber=1]
    args = [time, pressure, rd, qFull, dpdt]
    x0 = [1.7,1.]
    
    # Scelgo il metodo
    #method='Powell'
    #method='CG'     NON FUNZIONA
    #method='BFGS'   NON FUNZIONA
    method='Nelder-Mead'
    
    res = minimize(funCR,x0,args=args,method=method)
    print(res)
    
    if res.success:
        C = res.x[0]
        alpha = res.x[1]
        print("C: %.5f" % C)
        print("alpha: %.5f" % alpha)
    else:
        print("ATTENTION: did not succeed in finding C!")
\end{lstlisting}

\begin{lstlisting}[language=Python,caption={Code to generate the figure \ref{soluzioneCalphaapprossimata}.}\label{soluzioneCalphastimate}, columns=fullflexible,firstnumber=1, stepnumber=1]
    # I parametri sono stati aggiornati nei codici precedenti
    r1= (1-alpha)*rd
    r2 = alpha*rd
    
    args = [[C, r2, qFull, pd]]
    fun = dpdt
    t_span = [time[0],time[-1]]
    y0 = [pressure[0]]
    method='RK45'
    t_eval = time
    tol = 1e-6
    
    # solve_ivp
    sol = solve_ivp(fun=fun, t_span=t_span,y0=y0,method=method, t_eval=t_eval,args=args,rtol=tol)
    pPred = sol.y[0,:] + r1*flow
    
    # Plot
    err = pressure-pPred
    plt.plot(time,pressure,'g-',label='Data')
    plt.plot(time[::10],pPred[::10],'rx',label='Model')
    plt.xlabel("time [s]")
    plt.ylabel("$P\,[mmHg]$")
    plt.legend()
\end{lstlisting}

\newpage

\begin{lstlisting}[language=Python,caption={Code for centered finite difference method repurposed for local sensitivity calculation.}\label{differenzefinite}, columns=fullflexible,firstnumber=1, stepnumber=1]
    def centredFiniteDifference(back, forward, h):
    '''
    ----- Parameters -----
    back: lista Ppred con parametro Pi-h
    forward: lista Ppred con parametro Pi+h
    h : variazione del parametro Pi
    ----- Returns -----
    lista differenze finite centrate: [map', dbp', sbp', pp']        
    '''
    
    # MAP
    mapb = np.average(back)            # MAP(Pi-hi)
    mapf = np.average(forward)         # MAP(Pi+hi)
    mapDerivative = (mapf-mapb)/(2*h)
    
    # DBP
    dbpb = np.min(back)                # DBP(Pi-hi)
    dbpf = np.min(forward)             # DBP(Pi+hi)
    dbpDerivative = (dbpf-dbpb)/(2*h)
    
    # SBP
    sbpb = np.max(back)                # SBP(Pi-hi)
    sbpf = np.max(forward)             # SBP(Pi+hi)
    sbpDerivative = (sbpf-sbpb)/(2*h)
    
    # PP
    ppb = sbpb - dbpb                  # PP(Pi-hi)
    ppf = sbpf - dbpf                  # PP(Pi+hi)
    ppDerivative = (ppf-ppb)/(2*h)
    
    return (mapDerivative, dbpDerivative, sbpDerivative, ppDerivative)
\end{lstlisting}

%\begin{lstlisting}[language=Python,caption={Codice per il calcolo delle variabili}\label{variabili}, columns=fullflexible,firstnumber=1, stepnumber=1, style=mystyle]
    map = np.average(pPred)    # MAP
    dbp = np.min(pPred)        # DBP
    sbp = np.max(pPred)        # SBP 
    pp = sbp - dbp             # PP
\end{lstlisting}

% keywordstyle=\bfseries\color{black}

\begin{lstlisting}[language=Python,caption={Codice per il calcolo della sensitività di $C$}\label{Csensitivity}, columns=fullflexible,firstnumber=1, stepnumber=1, style=mystyle]
    #************ +10%  ****************
    cost = 1.1
    args = [[C*cost, r2, qFull, pd]]
    sol_C_1 = solve_ivp(fun=fun, t_span=t_span,y0=y0,method=method, t_eval=t_eval,args=args,rtol=tol)
    pPred_C_1 = sol_C_1.y[0,:] + r1*flow
    
    #************ -10%  ****************
    cost = 0.9
    args = [[C*cost, r2, qFull, pd]]
    sol_C_2 = solve_ivp(fun=fun, t_span=t_span,y0=y0,method=method, t_eval=t_eval,args=args,rtol=tol)
    pPred_C_2 = sol_C_2.y[0,:] + r1*flow
    
    # Derivate parziali
    partialDerivativeC = centredFiniteDifference(pPred_C_1, pPred_C_2, C*(1-cost))
    
    # Sensitivita
    S_map_C = (C/map) * partialDerivativeC[0]
    S_dbp_C = (C/dbp) * partialDerivativeC[1]
    S_sbp_C = (C/sbp) * partialDerivativeC[2]
    S_pp_C = (C/pp) * partialDerivativeC[3]
\end{lstlisting}

\newpage

\begin{lstlisting}[language=Python,caption={Codice per il calcolo della sensitività di $R_1$}\label{R1sensitivity}, columns=fullflexible,firstnumber=1, stepnumber=1, style=mystyle]
    #************ +10%  ****************
    cost = 1.1
    args = [[C, r2, qFull, pd]]
    sol_R1_1 = solve_ivp(fun=fun, t_span=t_span,y0=y0,method=method, t_eval=t_eval,args=args,rtol=tol)
    pPred_R1_1 = sol.y[0,:] + r1*cost*flow
    
    #************ -10%  ****************
    cost = 0.9
    args = [[C, r2, qFull, pd]]
    sol_R1_2 = solve_ivp(fun=fun, t_span=t_span,y0=y0,method=method, t_eval=t_eval,args=args,rtol=tol)
    pPred_R1_2 = sol.y[0,:] + r1*cost*flow
    
    # Derivate parziali
    partialDerivativeR1 = centredFiniteDifference(pPred_R1_1, pPred_R1_2, r1*(1-cost))
    
    # Sensitivita
    S_map_R1 = (r1/map) * partialDerivativeR1[0]
    S_dbp_R1 = (r1/dbp) * partialDerivativeR1[1]
    S_sbp_R1 = (r1/sbp) * partialDerivativeR1[2]
    S_pp_R1 = (r1/pp) * partialDerivativeR1[3]
\end{lstlisting}

\begin{lstlisting}[language=Python,caption={Code for calculating the sensitivity of $R_2$}\label{R2sensitivity}, columns=fullflexible,firstnumber=1, stepnumber=1, style=mystyle]
   #************ +10%  ****************
    cost = 1.1
    args = [[C, r2*cost, qFull, pd]]
    sol_R2_1 = solve_ivp(fun=fun, t_span=t_span,y0=y0,method=method, t_eval=t_eval,args=args,rtol=tol)
    pPred_R2_1 = sol_R2_1.y[0,:] + r1*flow
    
    #************ -10%  ****************
    cost = 0.9
    args = [[C, r2*cost, qFull, pd]]
    sol_R2_2 = solve_ivp(fun=fun, t_span=t_span,y0=y0,method=method, t_eval=t_eval,args=args,rtol=tol)
    pPred_R2_2 = sol_R2_2.y[0,:] + r1*flow
    
    # Derivate parziali
    partialDerivativeR2 = centredFiniteDifference(pPred_R2_1, pPred_R2_2, r2*(1-cost))
    
    # Sensitivita
    S_map_R2 = (r2/map) * partialDerivativeR2[0]
    S_dbp_R2 = (r2/dbp) * partialDerivativeR2[1]
    S_sbp_R2 = (r2/sbp) * partialDerivativeR2[2]
    S_pp_R2 = (r2/pp) * partialDerivativeR2[3]
\end{lstlisting}

\newpage

\begin{lstlisting}[language=Python,caption={Codice per il calcolo della sensitività di $P_d$}\label{Pdsensitivity}, columns=fullflexible,firstnumber=1, stepnumber=1, style=mystyle]
   #************ +10%  ****************
    cost = 1.1
    args = [[C, r2, qFull, pd*cost]]
    sol_Pd_1 = solve_ivp(fun=fun, t_span=t_span,y0=y0,method=method, t_eval=t_eval,args=args,rtol=tol)
    pPred_Pd_1 = sol_Pd_1.y[0,:] + r1*flow
    
    #************ -10%  ****************
    cost = 0.9
    args = [[C, r2, qFull, pd*cost]]
    sol_Pd_2 = solve_ivp(fun=fun, t_span=t_span,y0=y0,method=method, t_eval=t_eval,args=args,rtol=tol)
    pPred_Pd_2 = sol_Pd_2.y[0,:] + r1*flow
    
    # Derivate parziali
    partialDerivativePd = centredFiniteDifference(pPred_Pd_1, pPred_Pd_2, pd*(1-cost))
    
    # Sensitivita
    S_map_Pd = (pd/map) * partialDerivativePd[0]
    S_dbp_Pd = (pd/dbp) * partialDerivativePd[1]
    S_sbp_Pd = (pd/sbp) * partialDerivativePd[2]
    S_pp_Pd = (pd/pp) * partialDerivativePd[3]
\end{lstlisting}

\begin{lstlisting}[language=Python,caption={Codice per definire la funzione di flusso per il modello Windkessel ciclico}\label{flusso periodico}, columns=fullflexible,firstnumber=1, stepnumber=1]
    def qFull(t):
        if type(t)==float:
            tLoc = t-int(t)*1.
        else:
            tLoc = t-t.astype(int)*1.
        q = np.interp(tLoc,time,flow)
        return q
\end{lstlisting}



\end{document}